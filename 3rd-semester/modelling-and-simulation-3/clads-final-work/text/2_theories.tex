\section{Important Concepts}
\begin{itemize}
\item Ponzi Scheme: Is a fraudulent type of investment scheme that uses later investments to provided quick, high returns to early investors. These schemes focus on attracting new investors rather than engaging in any legitimate investment. In the 1920s, Charles Ponzi initially bought a small number of international mail coupons in support of his scheme, but quickly switched to using incoming funds from new investors to pay returns to earlier investors. While dealing with international exchange rates, postal organizations and foreign currency kept him from producing actual revenue, the scheme did allow him to brag and advertise about the investment opportunity. In a few months, he managed to convince hundreds of people to invest in his business; Ponzi used the funds to buy a mansion and deposited cash in banks all across New England (today UK) \cite{ponzi} \cite{ezubao}.

\item Foreign Exchange Market: Global online network where traders buy or sell currencies, its main objective is to set the exchange rate for currencies \cite{foreignExchange}. The basic concept behind the foreign exchange (or forex) market is for trading currencies,
one pair against another. In 2010, it was the world’s largest market, consisting of almost \$2 trillion in daily volume and is growing rapidly. The price of each currency within the pair is determined by a number of factors, such as changes in political leadership, economic booms or busts or even natural disasters \cite{forex}. 

\item ICO (Initial coin offering): It is a method in which a cryptocurrency startup firm sells a number of its cryptocurrency to companies and investors to back their project up; it is similar to A IPO (initial public offering), where new companies sell shares to other companies or investors. Early investors in the operation are usually motivated to buy the cryptocoins in the hope that the plan becomes successful after it launches which could translate to a higher cryptocurrency value than what they purchased it for before the project was initiated. Since these fund-raising operations are not regulated by financial authorities, although there are successful ICOs, there are ICOs and crowd-sales campaigns that are fraudulent. Funds that are lost due to fraudulent activities may never be recovered \cite{ICO}.
\item Blockchain Protocol: It is used to serialize transactions of the currency among its users, it is maintained by a replicated state machine that keeps user's transactions and balances. This state machine is managed by nodes, called miners. Cryptographic methods are used to ensure security on each transaction, the miners commit the transactions into a global-decentralized log called blockchain. This is the protocol that several cryptocurrencies use for their trades and transactions \cite{blockchain}.
\end{itemize}

