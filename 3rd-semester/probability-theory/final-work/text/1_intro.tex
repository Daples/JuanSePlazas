\section{Introduction}
In the 1980s, Edward Jaynes wrote the book ``Probability as Logic'' stating a methodology to handle uncertainty in computational models such as the models we find in Artificial Intelligence (AI); in that time, it was of the utmost importance to find answers to these new challenges to handle real problems that are inherently unpredictable. This approach proposed by Jaynes is called Bayesian programming. The new paradigm is an approach to solve problems that concern decision making based in real data, using probability theory, specially using Bayes theorem.

This perspective is very useful for solving any type of issue that involves decision making in any level. For example, this type of approach has solved problems such as automatic driving cars \cite{coue2003using}, robot navigation with topology learning \cite{tapus2004topology} and so forth. In this manner, this paradigm can be useful in a variety of problems specially ones related to video games.

The biggest problem back in the day in the video games industry was programming bots that felt real to players; the issue was that the bots have to be real enough that the player doesn't feel cheated by the game but, challenging enough to be fun. The main solution was using finite state machines, that for the most part accomplished the job; the principal issue found with this approach, was that it very tedious to type it out, it doesn't imitate really well a ``human conduct'' and it is really hard to translate between problems as we will show in Section \ref{sec:theory}. On the other hand, Bayesian programming allows it to generalize in a more simpler way and, makes the work easier for the programmer; in the same way, it allows the character of the video game to learn a desired behavior without too much of a problem \cite{le2004teaching}.

On the other hand, the gaming market has become seriously relevant in the past decade, in 2016 it was around USD 97.80 billion and it is expected to grow to USD 171.96 billion by 2025 \cite{gamingMarket}. It is important to highlight that the success of this franchise is, in part, due to the user experience and nowadays it's each time harder to achieve a more enjoyable experience, since video games are becoming more realistic and gamers (users) more capable of completing levels and are developing skills at games in general. This is the main reason to address this problem: game developers need to find new ways of challenging the user or making more realistic and unpredictable AI for games.

On this paper, it will be more focused of handling the mathematics using bayesian statistical analysis rather than focusing on bayesian networks and programming.