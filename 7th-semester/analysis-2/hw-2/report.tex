\documentclass[11pt]{article}
\usepackage[english]{babel}
\usepackage{geometry}
\usepackage{amsmath}
\usepackage{amsthm}
\usepackage{graphicx}
\usepackage[utf8]{inputenc}

%%%%%%%% MARGIN
\geometry{verbose, letterpaper, tmargin=3cm,
  bmargin=3cm,lmargin=2.5cm,rmargin=2.5cm}

%%%%%%%% PARAGRAPH SETTINGS
% https://tex.stackexchange.com/questions/27802/set-noindent-for-entire-file
\setlength\parindent{0pt}

% https://tex.stackexchange.com/questions/49188/how-to-insert-vertical-space-between-paragraphs
\setlength{\parskip}{5pt}

%%%%%%%% SUB-FIGURE PACKAGE
\usepackage{subcaption}

%%%%%%%% HYPERREF PACKAGE
\usepackage{hyperref}
\hypersetup{linkcolor=blue}
\hypersetup{citecolor=blue}
\hypersetup{urlcolor=blue}
\hypersetup{colorlinks=true}

%%%%%%%% DEFINITION AND THEOREM DEFINITIONS
\theoremstyle{definition}
\newtheorem{definition}{Definition}[section]

\theoremstyle{remark}
\newtheorem{remark}{Remark}

\theoremstyle{remark}
\newtheorem{question}{Question}

\newtheorem{theorem}{Theorem}[section]

%%%%%%%% MULTI-COLUMNS PACKAGE
\usepackage{multicol}

%%%%%%%% PERSONAL COMMANDS
\usepackage{amssymb}

%%%% Important sets
\renewcommand{\O}{\mathbb{O}}
\newcommand{\N}{\mathbb{N}}
\newcommand{\Z}{{\mathbb{Z}}}
\newcommand{\Q}{{\mathbb{Q}}}
\newcommand{\R}{{\mathbb{R}}}

%%%% Usual operations
\newcommand{\pow}[2]{#1^{#2}}
\newcommand{\expp}[1]{e^{#1}}
\newcommand{\fst}{\mathrm{fst}}
\newcommand{\snd}{\mathrm{snd}}

%%%% Lambda Calculus
\newcommand{\dneq}{\,\, \# \,\,}
\renewcommand{\S}{\pmb{\mathrm{S}}}
\newcommand{\I}{\pmb{\mathrm{I}}}
\newcommand{\K}{\pmb{\mathrm{K}}}
\newcommand{\ch}[1]{\ulcorner #1 \urcorner}

%%%% Ordinal Lambda Calculus
\newcommand{\ordAlph}{\Sigma_{\text{Ord}}}
\newcommand{\termOrd}{\text{Term}_\text{Ord}}
\newcommand{\fl}{\mathrm{fl}}
\newcommand{\sk}{\mathrm{sk}}

%% Superscript to the left
% https://latex.org/forum/viewtopic.php?t=455
\usepackage{tensor}
\newcommand{\app}[3]{\tensor*[^{#1}]{\left(#2, #3\right)}{}}

%%%% Make optional parameter
% https://tex.stackexchange.com/questions/217757/special-behavior-if-optional-argument-is-not-passed
\usepackage{xparse}

%%%% Statistics
\NewDocumentCommand{\E}{o m}{
  \IfNoValueTF{#1}
  {\mathbb{E}\left[#2\right]}
  {\mathbb{E}^{#1}\left[ #2\right]}
}
\NewDocumentCommand{\V}{o m}{
  \IfNoValueTF{#1}
  {\mathrm{Var}\left[#2\right]}
  {\mathrm{Var}^{#1}\left[ #2\right]}
}
\RenewDocumentCommand{\P}{o o m}{
  \IfNoValueTF{#1}
  {\IfNoValueTF{#2}
    {\mathrm{P}\left(#3\right)}
    {\mathrm{P}^{#2}\left(#3\right)}}
  {\IfNoValueTF{#2}
    {\mathrm{P}_{#1}\left(#3\right)}
    {\mathrm{P}_{#1}^{#2} \left(#3\right)}}
}

%%%% Lambda Calculus
\NewDocumentCommand{\cx}{o}{
  \IfNoValueTF{#1}
  {\left[\quad\right]}
  {\left[\, #1 \,\right]}
}

%%%% Create absolute value function
% https://tex.stackexchange.com/questions/43008/absolute-value-symbols
\usepackage{mathtools}
\DeclarePairedDelimiter\abs{\lvert}{\rvert}%
\DeclarePairedDelimiter\norm{\lVert}{\rVert}%
\makeatletter
\let\oldabs\abs
\def\abs{\@ifstar{\oldabs}{\oldabs*}}
%
\let\oldnorm\norm
\def\norm{\@ifstar{\oldnorm}{\oldnorm*}}
\makeatother

%%%%%%%% LOGIC TREES
\usepackage{prftree}

%%%%%%%% SPLIT EQUATIONS
% https://tex.stackexchange.com/questions/51682/is-it-possible-to-pagebreak-aligned-equations
\allowdisplaybreaks

%%%%%%%% FLOAT SPECIFIER
% https://www.overleaf.com/learn/latex/Errors/LaTeX_Error:_Unknown_float_option_%60H%27
\usepackage{float}

%%%%%%%% TO USE SHORT COMMANDS FOR VECTOR LINES
\usepackage{esvect}

%%%%%%%% ENUMERATE LABEL
% https://www.latex-tutorial.com/tutorials/lists/
\usepackage{enumitem}

%%%%%%%% CODE RENDERING !!! UNCOMMENT IF NEEDED !!!
% Compile with flag -shell-escape
%\usepackage{minted}

%%%%%%%% START DOCUMENT

\title{Workshop Analysis II}
\author{Juan Sebasti\'an C\'ardenas-Rodríguez \\
  \scalebox{0.7}{Mathematical Engineering, Universidad EAFIT}}
\date{\today}


\begin{document}
\maketitle
\begin{question}
  Two players, $A$ and $B$, throw a dice. Describe the probability
  space associated to this experiment and each one of the following
  random variables:
  \begin{enumerate}[label=\alph*)]
  \item $X$: Score obtained by player $A$.
  \item $Y$: Biggest score.
  \item $Z$: Earnings of the player $A$ if the player that obtains the
    biggest score receives a coin from the other player.
  \item $U$: Earnings of the player $A$ if the player with least score
    pays to the adversary the difference of the scores.
  \end{enumerate}
\end{question}

\begin{proof}
  Let $x$ be the score obtained by player $A$ and $y$ the score
  obtained by player $B$ in the tuple $(x, y)$. Let
  $\Gamma = \{1,2,3,4,5,6\}$. Then:
  \begin{equation*}
    \Omega = \{(x, y) \mid x,y \in \Gamma\} \quad
    \mathcal{A} = \mathcal{P}(\Omega)
  \end{equation*}
  with $\mathcal{P}(\cdot)$ the power set function. Let's define the
  function $\mathrm{P}(\cdot)$ in the following way:
  \begin{align*}
    \mathrm{P}: \mathcal{A}& \rightarrow [0, 1] \\
    A& \mapsto \P{\omega} = \frac{|A|}{36}
  \end{align*}
  with $|\cdot|$ the cardinality of a set. It is clear that
  $\mathcal{A}$ is a $\sigma$-algebra of $\Omega$ as the power set
  always is. Hence, let's prove that $(\Omega, \mathcal{A},\mathrm{P})$ is a
  measurable space.
  \begin{itemize}
  \item It is easy to see that:
    \begin{equation*}
      \P{\emptyset} = \frac{\lvert \emptyset \rvert}{36} = 0
    \end{equation*}
  \item Let $\{E_i\}_{i \in \N}$ such that $E_i \cap E_j = \emptyset$
    for $\forall i \ne j$. Hence, it is obtained that:
    \begin{align*}
      \P{\bigcup_{i \in \N} E_i} &= \frac{\abs{\bigcup\limits_{i \in \N} E_i}}{36} \\
                                 &= \dfrac{\sum\limits_{i \in \N} \lvert E_i \rvert}{36} \\
                                 &= \sum_{i \in \N}\frac{\abs{E_i}}{36} \\
                                 &= \sum_{i \in \N} \P{E_i}
    \end{align*}
  \end{itemize}
  Therefore, $(\Omega, \mathcal{A}, \mathrm{P})$ is a measurable
  space.

  On the other hand, it is clear that $\abs{\Omega} = 36$. Hence, we
  obtain that $\P{\Omega} = 1$. Then, the selected space is also a
  probability space.
  \begin{enumerate}[label=\alph*)]
  \item As the tuples contain the information about the score of the
    first player in the first dimension, let's define the function
    $\fst(\cdot)$ as the projection of the tuple in the first
    dimension. Hence, the random variables is defined as:
    \begin{align*}
      X : \Omega & \rightarrow \R \\
      \omega& \mapsto X(\omega) = \fst(\omega)
    \end{align*}

  \item Let $\max(\cdot)$ denote the function of the maximum value of
    a tuple. Then, the random variable is defined as:
    \begin{align*}
      Y : \Omega & \rightarrow \R \\
      \omega& \mapsto Y(\omega) = \max(\omega)
    \end{align*}

  \item Let $\snd(\cdot)$ denote the function of the projection of the
    tuple in the second dimension. Hence, the random variable, with
    the type $Z : \Omega \rightarrow \R$, is defined as:
    \begin{equation*}
      Z(\omega) =
      \begin{cases}
        1, &\fst(\omega) > \snd(\omega) \\
        0, &\fst(\omega) = \snd(\omega) \\
        -1, &\text{otherwise}
      \end{cases}
    \end{equation*}

  \item The random variable is defined as:
    \begin{align*}
      U : \Omega & \rightarrow \R \\
      \omega& \mapsto U(\omega) = \fst(\omega) - \snd(\omega)
    \end{align*}

    It is important to notice that each of the r.v. defined have to be
    $\Omega-\mathcal{B}$-measurable as $\mathcal{A}$ has all the
    possible subsets of $\Omega$.
  \end{enumerate}
\end{proof}

\begin{question}
  For the random variables $X$ and $Y$ of the previous question,
  calculate the expectancy, the variance and the cumulative
  distribution function.
\end{question}
\begin{proof}
  Let's find the expectancy for the random variable $X$. It is clearly
  seen that this random variable is discrete with $x_i = i$, for
  $i \in \Gamma$ (recall that $\Gamma$ is defined in the first
  proof). Furthermore, using the theorem proved in the teacher notes:
  \begin{equation*}
    \E{f \circ X} = \sum_i f(x_i)\P[X]{\{x_i\}}
  \end{equation*}

  With the following function:
  \begin{align*}
    \mathrm{I} : \R& \rightarrow \R \\
                 x& \mapsto \mathrm{I}(x) = x
  \end{align*}

  One can prove that the expectancy of the random variable can be
  written as:
  \begin{align*}
    \E{X} &= \E{\mathrm{I} \circ X} \\
          &= \sum_i \mathrm{I}(x_i) \P[X]{\{x_i\}} \\
          &= \sum_i x_i \P[X]{\{x_i\}}
  \end{align*}

  Furthermore it is easily seen that for a value $\alpha \in \Gamma$:
  \begin{align*}
    \P[X]{\{\alpha\}} &= \P{\omega \in \Omega \mid X(\omega) \in \{\alpha\}} \\
                      &=\P{\{(\alpha, y) \mid y \in \Gamma\}} \\
                      &=\frac{\abs{\{(\alpha, y) \mid y \in \Gamma\}}}{36} \\
                      &= \frac{6}{36} \\
                      &= \frac{1}{6}
  \end{align*}

  Hence the expectancy for this r.v. is:
  \begin{align*}
    \E{X} &= \sum_{i=1}^6 x_i \P[X]{\{x_i\}} \\
          &= \sum_{i=1}^6 \frac{i}{6} \\
          &= \frac{1}{6} \cdot\frac{6(6 + 1)}{2} \\
          &= \frac{7}{2} = 3.5
  \end{align*}

  To calculate the variance, the following formula proved in the
  teacher notes will be used:
  \begin{equation*}
    \V{X} = \E{X^2} - \E{X}^2
  \end{equation*}

  Hence, using a similar reasoning with the following function:
  \begin{align*}
    h : \R& \rightarrow \R \\
    x& \mapsto h(x) = x^2
  \end{align*}

  Then:
  \begin{align*}
    \E{X^2} &= \E{h \circ X} \\
            &= \sum_{i=1}^6 h(x_i) \P[X]{\{x_i\}} \\
            &= \sum_{i=1}^6 i^2 \cdot \frac{1}{6} \\
            &= \frac{1}{6} \cdot \frac{6 (6 + 1) (2\cdot 6 + 1)}{6} \\
            &= \frac{7 \cdot 13}{6} = \frac{91}{6}
  \end{align*}

  Finally:
  \begin{equation*}
    \V{X} = \frac{91}{6} - \frac{49}{4} = \frac{35}{12}
  \end{equation*}

  Lastly, let's define the cumulative distribution function. It is
  clearly seen as it is discrete the jump between each $x_i$ has a
  length of $\frac{1}{6}$. Hence:
  \begin{align*}
    F_X(t) &= \P[X]{[-\infty, t]} \\
           &= \P{\omega \in \Omega \mid X(\omega) \in [-\infty, t]} \\
    &=
      \begin{cases}
        0, &t < 1 \\
        \dfrac{\lfloor t \rfloor}{6}, &1 \le t < 7 \\
        1, &t \ge 7
      \end{cases}
  \end{align*}

  Now, let's repeat the process above for the random variable $Y$. In
  first place, it is easily seen that this r.v. is also discrete and
  has the same values as the previous one $y_i = i$. Hence, for
  $\alpha \in \Gamma$:
  \begin{align*}
    \P[Y]{\{\alpha\}} &= \P{\{\omega \in \Omega \mid Y(\omega) \in \{\alpha\}\}} \\
                      &= \P{\{(\alpha, y) \mid y \le \alpha\} \cup
                        \{(x, \alpha) \mid x < \alpha\}} \\
                      &= \P{\{(\alpha, y) \mid y \le \alpha\}} +
                        \P{\{(x, \alpha) \mid x < \alpha\}} \\
                      &= \frac{\abs{\{(\alpha, y) \mid y \le \alpha\}}}{36} +
                        \frac{\abs{\{(x, \alpha) \mid x < \alpha\}}}{36} \\
                      &= \frac{\alpha}{36} + \frac{\alpha - 1}{36} \\
                      &= \frac{2\alpha - 1}{36}
  \end{align*}

  Then, the expectancy can be calculated as:
  \begin{align*}
    \E{Y} &= \sum_{i=1}^6 y_i \P[Y]{\{y_i\}} \\
          &= \sum_{i=1}^6 i \cdot \frac{2i - 1}{36} \\
          &= \frac{1}{18}\sum_{i=1}^6i^2 - \frac{1}{36} \sum_{i=1}^6 i \\
          &= \frac{1}{18}\frac{6(6 + 1)(2\cdot 6 + 1)}{6} - \frac{1}{36}\frac{6(6 + 1)}{2} \\
          &= \frac{91}{18} - \frac{7}{12} = \frac{161}{36}
  \end{align*}

  Then, the second moment can be calculated similarly as the last
  proof. Hence:
  \begin{align*}
    \E{Y^ 2} &= \sum_{i=1}^6 y_i^2 \P[Y]{\{y_i\}} \\
             &= \sum_{i=1}^6 i^2 \cdot\frac{2i-1}{36} \\
             &= \frac{1}{18} \sum_{i=1}^6 i^3 - \frac{1}{36} \sum_{i=1}^6 i^2 \\
             &= \frac{1}{18} \left[\frac{6(6 + 1)}{2} \right]^2 -
               \frac{1}{36} \frac{6(6+1)(2\cdot 6 + 1)}{6} \\
             &= \frac{49}{2} - \frac{91}{36} = \frac{791}{36}
  \end{align*}

  Then, the variance is:
  \begin{equation*}
    \V{Y} = \frac{791}{36} - \frac{161^2}{36^2} = \frac{2555}{1296}
  \end{equation*}

  Lastly, the cumulative distribution function can be found taking
  into account the property that:
  \begin{equation*}
    \sum_{i=1}^n (2i - 1) = 2\sum_{i=1}^ni - \sum_{i=1}^n 1 = n(n+1) - n = n^2
  \end{equation*}
  Hence:
  \begin{equation*}
    F_Y(t) =
    \begin{cases}
      0, &t < 1 \\
      \dfrac{\lfloor t \rfloor^2}{36}, &1 \le t < 7 \\
      1, &t \ge 7
    \end{cases}
  \end{equation*}
\end{proof}

\begin{question}
  For the random variable $Z$ of the first question, calculate
  \begin{equation*}
    \int_A Z d\mathrm{P}, \quad \text{where } A = \{(1,2), (4,3)\}
  \end{equation*}
\end{question}

\begin{proof}
  It is clear that the r.v. $Z$ is a simple function. The standard
  representation for this random variable is:
  \begin{equation*}
    Z(\omega) = \I_{A_1}(\omega) - \I_{A_2}(\omega)
  \end{equation*}
  where
  $A_1 = \{\omega \in \Omega \mid \fst(\omega) > \snd(\omega)\}$,
  $A_2 = \{\omega \in \Omega \mid \fst(\omega) < \snd(\omega)\}$ and:
  \begin{equation*}
    \I_\chi(\omega) =
    \begin{cases}
      1, &\omega \in \chi \\
      0, &\text{otherwise}
    \end{cases}
  \end{equation*}

  As this is not a non-negative simple function, the integral will be
  calculated by using:
  \begin{equation*}
    \int_A Z d\mathrm{P} = \int_A Z^+ d\mathrm{P} - \int_A Z^- d\mathrm{P}
  \end{equation*}

  where
  \begin{align*}
    Z^+(\omega) &= \I_{A_1}(\omega) \\
    Z^-(\omega) &= -\I_{A_2}(\omega)
  \end{align*}

  Hence, using the definition of the integral of a simple function it
  is obtained:
  \begin{align*}
    \int_A Z d\mathrm{P} &= \int_A Z^+ d\mathrm{P} - \int_A Z^- d\mathrm{P} \\
                         &= \P{A\cap A_1} + \P{A\cap A_2} \\
                         &= \P{\{(4,3)\}} + \P{\{(1,2)\}} \\
                         &= \frac{\abs{\{(4,3)\}}}{36} + \frac{\abs{\{(1,2)\}}}{36} \\
                         &= \frac{1}{36} + \frac{1}{36} = \frac{1}{18}
  \end{align*}
\end{proof}

\begin{question}
  A point rotates around a circumference of unitary radius and stops
  at a point $P$ randomly with any type of preference. Define a random
  variable $\Theta$ that the range values are the real numbers that
  represent the angle, in radians, formed by the radius, (from the
  center) of the circumference to $P$, and a fixed selected
  diameter.
  \begin{enumerate}[label=\alph*)]
  \item Find the probability space
    $(\Omega, \mathcal{A}, \mathrm{P})$, such that $\Theta$ is a
    random variable.
  \item Calculate the expectancy $\E{\Theta}$.
  \item Find the cumulative distribution function $F_\Theta$.
  \item Analyze the existence of the probability density function
    $f_\Theta$.
  \end{enumerate}
\end{question}
\begin{proof}
  \begin{enumerate}[label=\alph*)]
  \item Let's define the probability space as:
    \begin{equation*}
      \Omega = [0, 2\pi), \quad \mathcal{A} = \{\beta \in \mathcal{B} \mid
        \beta \subseteq \Omega\}, \quad \P{A} = \frac{\lambda(A)}{2\pi}
      \end{equation*}
      with $\lambda(\cdot)$ the Lebesgue measure and $\mathcal{B}$ the
      Borel algebra. Let's prove that $\mathcal{A}$ is a
      $\sigma$-algebra of $\Omega$.
      \begin{itemize}
      \item It is clear that $\Omega \in \mathcal{A}$ as $\Omega$ is a
        Borel set and a subset of itself.

      \item Let $A \in \mathcal{A}$. Let's see that
        $A^c \in \mathcal{A}$. It is clear that the complement can be
        written as:
        \begin{equation*}
          A^c = [0, 2\pi) - A
        \end{equation*}
        Hence, it is easily seen that $A^c$ is a subset of
        $\Omega$. Furthermore, as $A^c$ is a difference of two Borel
        sets hence it is also a Borel set. Therefore,
        $A^c \in \mathcal{A}$ as it is a Borel set subset of $\Omega$.

      \item Let $\{A_i\}_{i \in \N}$ be disjointed sets of
        $\mathcal{A}$. Let's show that the union of those sets is also
        in the $\sigma$-algebra. It is clear that:
        \begin{equation*}
          \bigcup_{i \in \N} A_i \in \mathcal{B}
        \end{equation*}
        as $A_i$ are Borel sets. Furthermore, as $A_i \in \mathcal{A}$
        then it is clear that $A_i \subseteq [0, 2\pi)$. Hence,
        applying union to both sides it is obtained:
        \begin{equation*}
          \bigcup_{i \in \N} A_i \subseteq [0, 2\pi)
        \end{equation*}
        Then, as $\bigcup\limits_{i \in \N} A_i$ is a Borel set subset of
        $\Omega$, then $\bigcup\limits_{i \in \N} A_i \in \mathcal{A}$.

        In conclusion, $\mathcal{A}$ is a $\sigma$-algebra of
        $\Omega$.
      \end{itemize}

      Now, let's show that the space
      $(\Omega, \mathcal{A}, \mathrm{P})$ is a measurable space.
      \begin{itemize}
      \item It is easily seen that:
        \begin{equation*}
          \P{\emptyset} = \frac{\lambda(\emptyset)}{2 \pi} = 0
        \end{equation*}

      \item Let $\{A_i\}_{i \in \N}$ be disjointed sets of the
        $\sigma$-algebra, then:
        \begin{equation*}
          \P{\bigcup_{i \in \N} A_i} = \frac{1}{2\pi} \lambda\left(\bigcup_{i \in \N} A_i\right)
        \end{equation*}
        As $A_i$ are Lebesgue-measurable (as they are Borel sets) one
        can conclude that:
        \begin{align*}
          \P{\bigcup_{i \in \N} A_i} &= \frac{1}{2\pi}\sum_{i \in \N} \lambda(A_i) \\
                                     &= \sum_{i \in \N} \frac{\lambda(A_i)}{2\pi} \\
                                     &= \sum_{i \in \N} \P{A_i}
        \end{align*}

        In conclusion, the space $(\Omega, \mathcal{A}, \mathrm{P})$
        is measurable.
      \end{itemize}

      Finally, it is seen that:
      \begin{equation*}
        \P{\Omega} = \frac{\lambda([0, 2\pi))}{2\pi} = 1
      \end{equation*}

      Hence, the space $(\Omega, \mathcal{A}, \mathrm{P})$ is a
      probability space.

      Finally, the random variable $\Theta$ can be defined by:
      \begin{align*}
        \Theta : \Omega& \rightarrow \R \\
        \omega& \mapsto \Theta(\omega) = \omega
      \end{align*}

      It is clear that this function is
      $\Omega - \mathcal{B}$-measurable as the inverse of a Borel set
      will return another Borel set subset of $\Omega$.

    \item In the item \ref{item:acc4}, the cumulative distribution function is
      proved. Using it to calculate the expectancy it is found that:
      \begin{align*}
        \E{\Theta} &= \int_0^\infty [1 - F_\Theta(t)] dt -
                     \int_{-\infty}^0 F_\Theta(t) dt \\
                   &= \int_0^{2\pi} [1 - F_\Theta(t)] dt
                     + \int_{2\pi}^\infty [1 - F_\Theta(t)]dt \\
                   &= \int_0^{2\pi} \left[1 - \frac{t}{2\pi}\right]dt \\
                   &= t - \frac{t^2}{4\pi} \Bigg\rvert_0^{2\pi} =
                     2\pi - \frac{4\pi^2}{4\pi} = \pi
      \end{align*}

    \item \label{item:acc4} By definition, it is obtained that:
      \begin{equation*}
        F_\Theta(t) = \P{\{\omega \in \Omega \mid \Theta(\omega) \le t\}}
        =
        \begin{cases}
          0, &t < 0 \\
          \P{\{\omega \in \Omega \mid \omega \le t\}}, &0 \le t < 2\pi \\
          1, &t \ge 2\pi
        \end{cases}
      \end{equation*}
      Hence:
      \begin{align*}
        \P{\{\omega \in \Omega \mid \omega \le t\}} &= \frac{\lambda([0, t])}{2\pi} \\
                                                    &= \frac{t}{2\pi}
      \end{align*}

      This is the cumulative distribution function. It clearly makes
      sense, as it corresponds to a uniform distribution in the range
      $[0, 2\pi)$.

    \item To analyze the existence of the probability density
      function, let's find $\mathrm{P}_\Theta$.
      \begin{align*}
        \P[\Theta]{A} &= \P{\{\omega \in \Omega \mid \Theta(\omega) \in A\}} \\
                      &= \P{\{\omega \in \Omega \mid \omega \in A\}} \\
                      &= \P{[0, 2\pi)\cap A} \\
                      &= \frac{\lambda([0, 2\pi)\cap A)}{2\pi}
      \end{align*}
      Now, let's see that $\mathrm{P}_\Theta \ll \lambda$ (absolute
      continuity respect a the Lebesgue measure). Let
      $A \in \mathcal{A}$ such that $\lambda(A) = 0$. It can be seen
      that:
      \begin{align*}
        A \cap [0, 2\pi) &\subseteq A \\
        \lambda(A \cap [0, 2\pi)) &\le \lambda(A) \\
        \lambda(A \cap [0, 2\pi)) &\le 0
      \end{align*}

      As $\lambda(\cdot)$ is a measure, then it is positive. Hence:
      \begin{align*}
        \lambda(A \cap [0, 2\pi)) &= 0 \\
        \frac{\lambda(A \cap [0, 2\pi))}{2\pi} &= 0 \\
        \P[\Theta]{A} &= 0
      \end{align*}

      In conclusion, it is obtained that
      $\mathrm{P}_\Theta \ll \lambda$. Then by Radon-Nikodym's theorem
      it exists a positive function $f_\Theta : \Omega \rightarrow \R$
      such that for all $A \in \mathcal{A}$
      \begin{equation*}
        \P[\Theta]{A} = \int_A f_\Theta d\lambda
      \end{equation*}

      the function $f_\Theta$ is the density probability function.
  \end{enumerate}
\end{proof}

\begin{question}
  Given the real numbers $a < b$, consider the function
  $Y : [a, b] \rightarrow \R$, whose graph is the segment that joins
  the points $(a, Y(a))$ with $(b, Y(b))$. Suppose the function $Y$
  defines a random variable that describes the election of a value $y$
  between $Y(a)$ and $Y(b)$, randomly and with no preference.
  \begin{enumerate}[label=\alph*)]
  \item Determine the probability space
    $(\Omega, \mathcal{A}, \mathrm{P})$, such that $Y$ is a random variable
  \item Determine the expected value $\E{Y}$.
  \item Determine the variance $\V{Y}$.
  \item Define the cumulative distribution function $F_Y$.
  \item Analyze the existence of the density probability function
    $f_Y$.
  \item If another function $Z$, such that $Y=Z$ almost everywhere
    respect to the Lebesgue measure do the answers change of the last
    items for this new function? Furthermore, if the probability space
    and the density probability function is known for a random
    variable, it is possible to determine explicitly the random
    variable?
  \end{enumerate}
\end{question}

\begin{proof}
  Let's solve this exercise similarly as the Question 4. Furthermore,
  to facilitate having to separate between cases it will be supposed
  that $0 < Y(a)$; this will allow to solve the integrals easier
  without having to separate into cases (the other case is easily done
  using the same logic). Furthermore, it will be supposed that
  $Y(a) < Y(b)$ as the other case is easily done with the same
  procedure.
  \begin{enumerate}[label=\alph*)]
  \item Let's define the probability space as:
    \begin{equation*}
      \Omega = [a, b], \quad \mathcal{A} = \{\beta \in \mathcal{B} \mid
      \beta \subseteq \Omega\}, \quad
      \P{A} = \frac{\lambda(\{Y(x) \mid x \in A\})}{Y(b) - Y(a)}
      \end{equation*}
      with $\lambda(\cdot)$ the Lebesgue measure and $\mathcal{B}$ the
      Borel algebra. Let's prove that $\mathcal{A}$ is a
      $\sigma$-algebra of $\Omega$.
      \begin{itemize}
      \item It is clear that $\Omega \in \mathcal{A}$ as $\Omega$ is a
        Borel set and a subset of itself.

      \item Let $A \in \mathcal{A}$. Let's see that
        $A^c \in \mathcal{A}$. It is clear that the complement can be
        written as:
        \begin{equation*}
          A^c = [a, b] - A
        \end{equation*}
        Hence, it is easily seen that $A^c$ is a subset of
        $\Omega$. Furthermore, as $A^c$ is a difference of two Borel
        sets hence it is also a Borel set. Therefore,
        $A^c \in \mathcal{A}$ as it is a Borel set subset of $\Omega$.

      \item Let $\{A_i\}_{i \in \N}$ be sets of $\mathcal{A}$. Let's
        show that the union of those sets is also in the
        $\sigma$-algebra. It is clear that:
        \begin{equation*}
          \bigcup_{i \in \N} A_i \in \mathcal{B}
        \end{equation*}
        as $A_i$ are Borel sets. Furthermore, as $A_i \in \mathcal{A}$
        then it is clear that $A_i \subseteq [a, b]$. Hence,
        applying union to both sides it is obtained:
        \begin{equation*}
          \bigcup_{i \in \N} A_i \subseteq [a, b]
        \end{equation*}
        Then, as $\bigcup\limits_{i \in \N} A_i$ is a Borel set subset of
        $\Omega$, then $\bigcup\limits_{i \in \N} A_i \in \mathcal{A}$.

        In conclusion, $\mathcal{A}$ is a $\sigma$-algebra of
        $\Omega$.
      \end{itemize}

      Now, let's show that the space
      $(\Omega, \mathcal{A}, \mathrm{P})$ is a measurable space.
      \begin{itemize}
      \item It is easily seen that:
        \begin{equation*}
          \P{\emptyset} = \frac{\lambda(\{Y(x) \mid x \in \emptyset\})}{Y(b) - Y(a)}
          = \frac{\lambda(\emptyset)}{(Y(b) - Y(a))} = 0
        \end{equation*}

      \item Let $\{A_i\}_{i \in \N}$ be disjointed sets of the
        $\sigma$-algebra, then:
        \begin{align*}
          \P{\bigcup_{i \in \N} A_i}
          &= \frac{1}{Y(b) - Y(a)}\lambda\left(
            \left\{Y(x) \mid x \in \bigcup_{i \in \N} A_i\right\}\right) \\
          &= \frac{1}{Y(b) - Y(a)} \lambda\left(\bigcup_{i \in \N}\{Y(x) \mid x \in A_i\}\right)
        \end{align*}
        As $A_i$ are Lebesgue-measurable (as they are Borel sets),
        $Y(\cdot)$ is injective, one can conclude that:
        \begin{align*}
          \P{\bigcup_{i \in \N} A_i}
          &= \frac{1}{Y(b) - Y(a)}\sum_{i \in \N} \lambda(\{Y(x) \mid x \in A_i\}) \\
          &= \sum_{i \in \N} \frac{\lambda(\{Y(x) \mid x \in A_i\})}{Y(b) - Y(a)} \\
          &= \sum_{i \in \N} \P{A_i}
        \end{align*}

        In conclusion, the space $(\Omega, \mathcal{A}, \mathrm{P})$
        is measurable.
      \end{itemize}

      Finally, it is seen that:
      \begin{align*}
        \P{\Omega} &= \frac{\lambda(\{Y(x) \mid x \in [a, b]\})}{Y(b) - Y(a)} \\
                   &= \frac{\lambda([Y(a), Y(b)])}{Y(b) - Y(a)} \\
                   &= 1
      \end{align*}

      Hence, the space $(\Omega, \mathcal{A}, \mathrm{P})$ is a
      probability space.

    \item In item \ref{item:acc5}, the cumulative distribution function is
      proved. Using it to calculate the expectancy it is found that:
      \begin{align*}
        \E{Y} &= \int_0^\infty [1 - F_Y(t)] dt -
                \int_{-\infty}^0 F_Y(t) dt \\
              &= \int_0^{Y(a)}[1 - F_Y (t)] dt + \int_{Y(a)}^{Y(b)} [1 - F_Y(t)] dt
                + \int_{Y(b)}^\infty [1 - F_Y(t)]dt \\
              &= \int_0^{Y(a)}dt +
                \int_{Y(a)}^{Y(b)} \left[1 - \frac{t - Y(a)}{Y(b) - Y(a)}\right]dt \\
              &= t \Bigg\rvert_0^{Y(a)} + \frac{1}{Y(b) - Y(a)} \int_{Y(a)}^{Y(b)}  [Y(b) - t] dt \\
              &= Y(a) + \frac{1}{Y(b)-Y(a)} \left(Y(b)t - \frac{1}{2}t^2\right)
                \Bigg\rvert_{Y(a)}^{Y(b)} \\
              &= Y(a) + \frac{1}{Y(b)-Y(a)} \left(Y(b)^2 - \frac{1}{2}Y(b)^2 -
                Y(a)Y(b) + \frac{Y(a)^2}{2}\right) \\
              &= Y(a) + \frac{Y(b) - Y(a)}{2} = \frac{Y(b) + Y(a)}{2}
      \end{align*}

    \item To calculate the second moment, let's first define a new
      random variable $Y^* = Y^2$. Let's find the cumulative
      distribution function for this variable:
      \begin{align*}
        F_{Y^*}(t) &= \P{\{\omega \in \Omega \mid Y^*(\omega) \le t\}} \\
                   &= \P{\{\omega \in \Omega \mid Y(\omega)^2 \le t\}} \\
                   &\text{If } t \ge 0 \text{ then} \\
                   &=\P{\{\omega \in \Omega \mid Y(\omega) \le \sqrt{t}\}} \\
                   &= F_Y(\sqrt{t})
      \end{align*}
      Hence, using the result in item \ref{item:acc5}, it is obtained
      that:
      \begin{equation*}
        F_{Y^2}(t) =
        \begin{cases}
          0, &t \le Y(a)^2 \\
          \dfrac{\sqrt{t} - Y(a)}{Y(b) - Y(a)}, &Y(a)^2 \le t \le Y(b)^2 \\
          1, &t \ge Y(b)^2
        \end{cases}
      \end{equation*}

      Hence, calculating the expectancy:
      \begin{align*}
        \E{Y^2} &= \int_0^\infty [1 - F_{Y^*}(t)]dt
                  - \int_{-\infty}^0 F_{Y^*}(t) \\
                &= \int_0^{Y(a)^2}dt + \int_{Y(a)^2}^{Y(b)^2}
                  \left[1 - \frac{\sqrt{t} - Y(a)}{Y(b) - Y(a)}\right] dt \\
                &= t \Bigg\rvert_0^{Y(a)^2} +
                  \frac{1}{Y(b) - Y(a)}\int_{Y(a)^2}^{Y(b)^2}\left[Y(b) - \sqrt{t}\right] dt \\
                &= Y(a)^2 + \frac{1}{Y(b) - Y(a)} \left(Y(b)t - \frac{2}{3} t^{3/2} \right)
                  \Bigg\rvert_{Y(a)^2}^{Y(b)^2} \\
                &= Y(a)^2 + \frac{1}{Y(b)-Y(a)} \left(Y(b)^3 - \frac{2}{3} Y(b)^3
                  - Y(a)^2Y(b) + \frac{2}{3}Y(a)^3\right) \\
                &= Y(a)^2 + \frac{1}{3(Y(b) - Y(a))} \left(Y(b)^3 - 3Y(a)^2Y(b) + 2Y(a)^3\right) \\
                &= Y(a)^2 + \frac{1}{3(Y(b) - Y(a))}(Y(b) - Y(a))^2(2Y(a) + Y(b)) \\
                &= Y(a)^2 + \frac{Y(b)^2 + Y(a)Y(b) - 2Y(a)^2}{3} \\
                &= \frac{Y(b)^2 + Y(a)Y(b) + Y(a)^2}{3}
      \end{align*}

      Then, the variance is calculated as:
      \begin{align*}
        \V{Y}
        &= \frac{Y(b)^2 + Y(a)Y(b) + Y(a)^2}{3} - \left(\frac{Y(b) + Y(a)}{2}\right)^2 \\
        &= \frac{Y(b)^2 + Y(a)Y(b) + Y(a)^2}{3} - \frac{Y(b)^2 + 2Y(a)Y(b) + Y(a)^2}{4} \\
        &= \frac{Y(b)^2 -2Y(a)Y(b) + Y(a)^2}{12} \\
        &= \frac{(Y(b) - Y(a))^2}{12}
      \end{align*}

    \item \label{item:acc5} By definition, it is obtained that:
      \begin{equation*}
        F_Y(t) = \P{\{\omega \in \Omega \mid Y(\omega) \le t\}} =
        \begin{cases}
          0, &t < Y(a) \\
          \P{\{\omega \in \Omega \mid Y(\omega) \le t\}}, &Y(a) \le t < Y(b) \\
          1, &t \ge Y(b)
        \end{cases}
      \end{equation*}
      Hence:
      \begin{align*}
        \P{\{\omega \in \Omega \mid Y(\omega) \le t\}}
        &= \frac{\lambda(\{ Y(x) \mid x \in [a, Y^{-1}(\{t\})]\})}{Y(b) - Y(a)} \\
        &= \frac{\lambda([Y(a), t])}{Y(b) - Y(a)} \\
        &= \frac{t - Y(a)}{Y(b) - Y(a)}
      \end{align*}

      This is the cumulative distribution function.

    \item To analyze the existence of the probability density
      function, let's find $\mathrm{P}_Y$.
      \begin{align*}
        \P[Y]{A} &= \P{\{\omega \in \Omega \mid Y(\omega) \in A\}} \\
                 &= \P{\{[a, b]\cap Y^{-1}(A)\}} \\
                 &= \frac{\lambda(\{Y(x) \mid x \in [a,b]\cap Y^{-1}(A)\})}{Y(b) - Y(a)} \\
                 &= \frac{\lambda([Y(a), Y(b)] \cap A)}{Y(b) - Y(a)}
      \end{align*}
      Now, let's see that $\mathrm{P}_Y \ll \lambda$ (absolute
      continuity respect a the Lebesgue measure). Let
      $A \in \mathcal{A}$ such that $\lambda(A) = 0$. It can be seen
      that:
      \begin{align*}
        A \cap [Y(a), Y(b)) &\subseteq A \\
        \lambda(A \cap  [Y(a), Y(b)]) &\le \lambda(A) \\
        \lambda(A \cap [Y(a), Y(b)]) &\le 0
      \end{align*}

      As $\lambda(\cdot)$ is a measure, then it is positive. Hence:
      \begin{align*}
        \lambda(A \cap [Y(a), Y(b)]) &= 0 \\
        \frac{\lambda(A \cap [Y(a), Y(b)])}{Y(b) - Y(a)} &= 0 \\
        \P[Y]{A} &= 0
      \end{align*}

      In conclusion, it is obtained that
      $\mathrm{P}_Y \ll \lambda$. Then by Radon-Nikodym's theorem
      it exists a positive function $f_Y : \Omega \rightarrow \R$
      such that for all $A \in \mathcal{A}$
      \begin{equation*}
        \P[Y]{A} = \int_A f_Y d\lambda
      \end{equation*}

      the function $f_Y$ is the density probability function.

    \item It is important to notice that:
      \begin{equation*}
        E[X] = \int_\Omega X dP \quad E[X^2] = \int_\Omega X^2 dP
      \end{equation*}
      Hence, as $Z = Y$ almost everywhere respect to the
      Lebesgue-measure and $P \ll \lambda$ then $Z=Y$ almost
      everywhere respect to the $P$-measure.

      In these manners, as an integral respect a measure does not
      differentiate between functions that are equal almost everywhere
      the integrals of both moments are equal for $Y$ and $Z$. Hence,
      the expectancy and variance of both r.v. are equal.

      Furthermore, as the distribution is build up using the Lebesgue
      measure the probability measure will not distinguish between $Z$
      or $Y$ because, Lebesgue-measure will not change the value for
      sets that include sets of measure 0. Hence, all results hold up.

      In this manner, it is impossible to determine exactly a random
      variable by it's density and probability space as it cannot know
      exactly where are some ``spots'' that have a measure of 0.
  \end{enumerate}
\end{proof}
\end{document}
