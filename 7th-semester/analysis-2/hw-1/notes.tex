\documentclass[11pt]{article}
\usepackage[english]{babel}
\usepackage{geometry}
\usepackage{amsmath}
\usepackage{amsthm}
\usepackage{graphicx}
\usepackage[utf8]{inputenc}

%%%%%%%% MARGIN
\geometry{verbose, letterpaper, tmargin=3cm,
  bmargin=3cm,lmargin=2.5cm,rmargin=2.5cm}

%%%%%%%% NO PARAGRAPH INDENT
% https://tex.stackexchange.com/questions/27802/set-noindent-for-entire-file
\setlength\parindent{0pt}

%%%%%%%% SUB-FIGURE PACKAGE
\usepackage{subcaption}

%%%%%%%% HYPERREF PACKAGE
\usepackage{hyperref}
\hypersetup{linkcolor=blue}
\hypersetup{citecolor=blue}
\hypersetup{urlcolor=blue}
\hypersetup{colorlinks=true}

%%%%%%%% DEFINITION AND THEOREM DEFINITIONS
\theoremstyle{definition}
\newtheorem{definition}{Definition}[section]

\theoremstyle{remark}
\newtheorem{remark}{Remark}

\theoremstyle{remark}
\newtheorem{question}{Question}

\newtheorem{theorem}{Theorem}[section]

%%%%%%%% MULTI-COLUMNS PACKAGE
\usepackage{multicol}

%%%%%%%% SETS DEFINITIONS
\usepackage{amssymb}
%%%% Important sets
\renewcommand{\O}{\mathbb{O}}
\newcommand{\N}{\mathbb{N}}
\newcommand{\Z}{{\mathbb{Z}}}
\newcommand{\Q}{{\mathbb{Q}}}
\newcommand{\R}{{\mathbb{R}}}

%%%% Statistics
\newcommand{\E}[1]{\mathbb{E}\left[#1 \right]}
\newcommand{\V}[1]{\mathrm{Var}\left[#1 \right]}

%%%% Lambda Calculus Symbols
\newcommand{\dneq}{\,\, \# \,\,}
\renewcommand{\S}{\pmb{\mathrm{S}}}
\newcommand{\I}{\pmb{\mathrm{I}}}
\newcommand{\K}{\pmb{\mathrm{K}}}
\newcommand{\ch}[1]{\ulcorner #1 \urcorner}

%%%% Make optional parameter
% https://tex.stackexchange.com/questions/217757/special-behavior-if-optional-argument-is-not-passed
\usepackage{xparse}
\NewDocumentCommand{\cx}{o}{
  \IfNoValueTF{#1}
  {\left[\quad\right]}
  {\left[\, #1 \,\right]}
}

%%%%%%%% LOGIC TREES
\usepackage{prftree}

%%%%%%%% SPLIT EQUATIONS
% https://tex.stackexchange.com/questions/51682/is-it-possible-to-pagebreak-aligned-equations
\allowdisplaybreaks

%%%%%%%% START DOCUMENT

\title{Analysis II - Notes}
\author{Juan Sebasti\'an C\'ardenas-Rodríguez \\
  \scalebox{0.7}{Mathematical Engineering, Universidad EAFIT}}
\date{\today}


\begin{document}
\maketitle
\begin{question}
  If $I \subseteq \R$ is an interval, then $\lambda^*(I) = L(I)$.
  \begin{equation*}
    L(I) =
    \begin{cases}
      b - a, &\quad I=[a,b], [a, b), (a, b], (a, b) \\
      \infty, &\quad \text{otherwise}
    \end{cases}
  \end{equation*}
\end{question}
\begin{proof}
  \begin{itemize}
  \item $[a,b]$
  \item $(a, b], [a, b), (a, b)$.  It is possible proof that for each
    interval $I$ constructed as above, it exists a closed interval
    $J \subset I$ such that for an $\epsilon > 0$:
    \begin{equation*}
      L(I) - \epsilon< L(J) \quad
      J =
      \begin{cases}
        \left[\frac{2a + b}{3}, \frac{2b + a}{3}\right], &\quad \epsilon \ge L^*(I) \\
        \left[a + \frac{\epsilon}{4}, b - \frac{\epsilon}{4}\right], &\quad\text{otherwise}
      \end{cases}
    \end{equation*}
    Hence, we would obtain that:
    \begin{align*}
      L(I) - \epsilon &\le L(J) \\
                      &= \lambda^*(J) \\
                      &\le \lambda^*(I) \\
                      &\le \lambda^*([a, b]) \\
                      &= b - a \\
                      &= L(I)
    \end{align*}
    Then, we conclude that as:
    \begin{equation*}
      L(I) - \epsilon < \lambda^*(J) \le L(I) \rightarrow \lambda^*(I) = L(I)
    \end{equation*}
  \end{itemize}
\end{proof}

\begin{question}
  Let $E \subseteq \R$ and $x \in \R$. Then, $E \in \mathcal{M}$ if
  and only if $E + x \in \mathcal{M}$.
\end{question}
\begin{proof}
  ($\rightarrow$). Let's suppose that $E \in \mathcal{M}$ and let
  $A \subseteq \R$. It is easily seen that for each set it exists another
  one $B \subseteq \R$ such that:
  \begin{equation*}
    A = B + x \quad B = \{y - x \mid y \in A\}
  \end{equation*}
  Furthermore, it is easily seen that:
  \begin{equation*}
    (B + x)\cap(E + x) = B\cap E + x \quad (B+x) \cap (E+x)^c = B\cap E^c + x
  \end{equation*}
  Hence:
  \begin{align*}
    \lambda^*(A \cap (E+x)) + \lambda^*(A \cap (E+x)^c)
    &= \lambda^*((B + x)\cap(E + x)) + \lambda^*((B+x) \cap (E + x)^c) \\
    &= \lambda^*(B\cap E + x) + \lambda^*(B\cap E^c + x) \\
    &= \lambda^*(B\cap E) + \lambda^*(B \cap E^c) \\
    &= \lambda^*(B) \\
    &= \lambda^*(B + x) \\
    &= \lambda^*(A)
  \end{align*}
\end{proof}

\begin{question}
  Let $E \subseteq \R$. If $\lambda^*(E) = 0$, then $E \in \mathcal{M}$.
\end{question}
\begin{proof}
  Let $A \subseteq \R$. Hence, it is clear that:
  \begin{equation*}
    A \cap E \subseteq E
  \end{equation*}
  Then, as the Lebesgue measure is a outer measure one can affirm:
  \begin{equation*}
    \lambda^*(A \cap E) \le \lambda^*(E)
  \end{equation*}
  Hence, $ \lambda^*(A \cap E) = 0$. Lastly:
  \begin{align*}
    \lambda^*(A \cap E^c) &\le \lambda^*(A) \\
    \lambda^*(A \cap E) + \lambda^*(A \cap E^c) &\le \lambda^*(A)
  \end{align*}
  Then, $E \in \mathcal{M}$
\end{proof}

\end{document}
