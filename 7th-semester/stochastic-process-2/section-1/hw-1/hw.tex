\documentclass[11pt]{article}
\usepackage[spanish]{babel}
\usepackage{geometry}
\usepackage{amsmath}
\usepackage{amsthm}
\usepackage{graphicx}
\usepackage[utf8]{inputenc}

%%%%%%%% MARGIN
\geometry{verbose,letterpaper,tmargin=3cm,bmargin=3cm,lmargin=2.5cm,rmargin=2.5cm}

%%%%%%%% SUB-FIGURE PACKAGE
\usepackage{subcaption}

%%%%%%%% HYPERREF PACKAGE
\usepackage{hyperref}
\hypersetup{linkcolor=blue}
\hypersetup{citecolor=blue}
\hypersetup{urlcolor=blue}
\hypersetup{colorlinks=true}

%%%%%%%% DEFINITION AND THEOREM DEFINITIONS
\theoremstyle{definition}
\newtheorem{definition}{Definición}[section]

\theoremstyle{remark}
\newtheorem{remark}{Anotación}

\theoremstyle{remark}
\newtheorem{question}{Pregunta}

\newtheorem{theorem}{Teorema}[section]

%%%%%%%% MULTI-COLUMNS PACKAGE
\usepackage{multicol}

%%%%%%%% SETS DEFINITIONS
\usepackage{amssymb}
\renewcommand{\O}{\mathbb{O}}
\newcommand{\N}{\mathbb{N}}
\newcommand{\Z}{{\mathbb{Z}}}
\newcommand{\Q}{{\mathbb{Q}}}
\newcommand{\R}{{\mathbb{R}}}

\newcommand{\ri}{\rightarrow}
\newcommand{\fOrd}{\O \ri \O}

%%%%%%%% START DOCUMENT

\title{Tarea Procesos Estocásticos}
\author{Juan Sebasti\'an C\'ardena Rodríguez \\ \scalebox{0.7}{Ingeniería Matemática, Universidad EAFIT} \\ \scalebox{0.7}{201710008101}}
\date{\today}


\begin{document}
\maketitle
\section*{Primera Bonificación}
\begin{question}
  Si $X$ es una variable aleatoria la cual ocurre que: $X \sim N(\mu, \sigma^2)$, probar que su
  función característica es:

  \begin{equation*}
    \Phi_X(\theta) = e^{i\mu\theta - \frac{\sigma^2\theta^2}{2}}
  \end{equation*}
\end{question}

\begin{proof}
  Tenemos que:
  \begin{equation*}
    \Phi_X(\theta) = E\left[e^{i\theta X}\right] = \int_{-\infty}^\infty
    e^{\i \theta x} \frac{1}{\sqrt{2\pi\sigma^2}} e^{-\frac{(x-\mu)^2}
      {2\sigma^2}}dx = \frac{1}{\sqrt{2\pi\sigma^2}}
    \int_{-\infty}^\infty e^{-\frac{(x-\mu)^2}{2\sigma^2} + i\theta x}dx
  \end{equation*}
  Utilizando solamente el exponente, podemos organizarlo de esta
  manera:
  \begin{equation*}
    \begin{split}
      -\frac{(x - \mu)^2}{2\sigma^2} + i\theta x &= \frac{2i\theta
        \sigma^2x - x^2 + 2x\mu - \mu^2}{2\sigma^2} \\
      &= \frac{-x^2 + 2(\mu + i\theta\sigma^2)x - \mu^2}{2\sigma^2} \\
      &= \frac{-(x^2 - 2(\mu + i\theta\sigma^2)x + (\mu + i\theta
        \sigma^2)^2) + (\mu + i\theta\sigma^2)^2 - \mu^2}{2\sigma^2}
      \\
      &= \frac{-(x - (\mu + i\theta\sigma^2))^2 + 2i\mu\theta\sigma^2
      - \theta^2\sigma^4}{2\sigma^2}
    \end{split}
  \end{equation*}
  De esta manera, la integral se puede reescribir como:
  \begin{equation*}
    = \frac{1}{\sqrt{2\pi\sigma^2}}e^{i\mu\theta -\frac{\theta^2
        \sigma^2}{2}}\int_{-\infty}^\infty e^{-\frac{(x - \alpha)^2}
      {2\sigma^2}}dx
  \end{equation*}
  Con $\alpha = \mu + i\theta\sigma^2$. Ahora, si sustituimos
  haciendo:
  \begin{equation*}
    z = \frac{x - \alpha}{\sqrt{2}\sigma} \quad dz = \frac{dx}
    {\sqrt{2}\sigma}
  \end{equation*}
  Lo cual, obtenemos:
  \begin{equation*}
    = \frac{e^{i\mu\theta -\frac{\theta^2\sigma^2}{2}}}{\sqrt{\pi}}
    \int_{-\infty}^\infty e^{-z^2}dz
  \end{equation*}

  Lo cual, es claro que la integral de la derecha es igual a
  $\sqrt{\pi}$; esta es una integral muy conocida. Así, obtenemos
  que:

  \begin{equation*}
    \Phi_X(\theta) = e^{i\mu\theta -\frac{\theta^2\sigma^2}{2}}
  \end{equation*}
\end{proof}

\begin{question}
  Probar que si $X$ es una variable aleatoria que distribuye
  $N(0, t)$, entonces ocurre que:
  \begin{equation*}
    E\left[X^{2k}\right] = \frac{(2k)!}{2^kk!}t^k, \text{ with }
    k \in \N
  \end{equation*}
\end{question}

\begin{proof}
  Se tiene que:
  \begin{equation*}
    E\left[X^{2k}\right] = \int_{-\infty}^{\infty}x^{2k}\frac{1}
    {\sqrt{2\pi t}}e^{-\frac{x^2}{2t}}dx
  \end{equation*}
  Si sustituimos de tal forma que:
  \begin{equation*}
    z = \frac{x}{\sqrt{2t}} \quad dz = \frac{dx}{\sqrt{2t}}
  \end{equation*}
  Obtenemos que:
  \begin{equation*}
    = \frac{(2t)^k}{\sqrt{\pi}} \int_{-\infty}^\infty z^{2k}e^{-z^2}dz
  \end{equation*}
  Lo cual, podemos integrar por partes así:
  \begin{equation*}
    \begin{split}
      &u = z^{2k - 1} \quad dv = ze^{-z^2}dz \\
      &du = (2k - 1)z^{2k-2}dz \quad v = \frac{-e^{-z^2}}{2} \\
      &= \frac{(2t)^k}{\sqrt{\pi}} \left(-z^{2k - 1}\frac{-e^{-z^2}}
        {2}\Bigg|_{-\infty}^\infty + \int_{-\infty}^\infty
        \frac{-e^{-z^2}}{2}(2k - 1)z^{2k-2}dz\right)
    \end{split}
  \end{equation*}
  Es claro, que el término que queda es igual a 0. Lo cual obtenemos:
  \begin{equation*}
    = \frac{(2t)^k}{2\sqrt{\pi}}(2k-1)\int_{-\infty}^{\infty}z^{2(k-1)}
    e^{-z^2}dz
  \end{equation*}
  Claramente, la integral restante es semejante a la integral inicial. Así, si aplicamos el mismo proceso $k$ veces llegaríamos a:
  \begin{equation*}
    = \frac{(2t)^k}{2^k\sqrt{\pi}}(2k - 1)\cdot(2k - 3) \dots 3
    \cdot 1 \int_{-\infty}^{\infty}e^{-z^2}dz
  \end{equation*}
  Lo cual, como establecí anteriormente, la integral restante es
  $\sqrt{pi}$. Por otro lado, la multiplicación de los números
  impares se conoce como el doble factorial. Y este tiene que:
  \begin{equation*}
    (2k - 1)!! = \frac{(2k)!}{2^kk!}
  \end{equation*}
  De esta manera:
  \begin{equation*}
    E\left[X^{2k}\right]= \frac{(2k)!}{2^kk!}t^k
  \end{equation*}
\end{proof}
\begin{remark}
  La simulación se encuentra en el código adjunto.
\end{remark}
\end{document}
