\maketitle
\section{}
Sea $x=0.d_1d_2...d_kd_{k+1}d_{k+2}...\times2^n$ un número en notación punto flotante en base 2. Se desea hallar un valor máximo para el error relativo, dada una aproximación con k cifras significativas. Se sabe que el valor se ve afectado por el tipo de redondeo, así que se hará un análisis con redondeo simétrico y redondeo por corte.
\begin{enumerate}
    \item Redondeo simétrico.\\Si $d_{k+1} = 0$:\\
$$\hat{x}=0.d_1d_2...d_k\times2^n$$
\begin{align*}
|\varepsilon| &= \left|\frac{x - \hat{x}}{x}\right|\\
&= \left|\frac{0.d_1d_2...d_kd_{k+1}d_{k+2}...\times2^n - 0.d_1d_2...d_k\times2^n}{0.d_1d_2...d_kd_{k+1}d_{k+2}...\times2^n}\right|\\
&= \left|\frac{0.d_1d_2...d_kd_{k+1}d_{k+2}... - 0.d_1d_2...d_k}{0.d_1d_2...d_kd_{k+1}d_{k+2}...} \right|\\
&=\left| \frac{0.d_{k+1}d_{k+2}...\times2^{-k}}{0.d_1d_2...d_kd_{k+1}d_{k+2}...} \right|\\
&=\left| \frac{0.0d_{k+2}...\times2^{-k}}{0.d_1d_2...d_kd_{k+1}d_{k+2}...} \right|\\
&=\left| \frac{0.d_{k+2}...\times2^{-k-1}}{0.d_1d_2...d_kd_{k+1}d_{k+2}...} \right|
\end{align*}
Ahora, para escoger una cota superior para el error relativo, se busca el mayor numero posible, el cual se obtiene con el mayor numerador y el menor numerador. Para el numerador, ese numero es $0.11111...$ el cual es equivalente a $1$. Para el denominador, ese numero es $0.1000...$ dado que al estar en notación punto flotante, $d_1\neq0$. Por lo tanto,
\begin{align*}
    |\varepsilon| &< \left| \frac{1\times2^{-k-1}}{0.1} \right|\\
    |\varepsilon| &< 2^{-k}
\end{align*}
Si $d_{k+1} = 1$:
$$ \hat{x} = 0.d_1d_2...(d_k+1)\times2^n$$
\begin{align*}
|\varepsilon| &= \left|\frac{x - \hat{x}}{x}\right|\\
&= \left|\frac{0.d_1d_2...d_kd_{k+1}d_{k+2}...\times2^n - 0.d_1d_2...(d_k+1)\times2^n}{0.d_1d_2...d_kd_{k+1}d_{k+2}...\times2^n}\right|\\
&= \left|\frac{0.d_1d_2...d_kd_{k+1}d_{k+2}... - 0.d_1d_2...(d_k+1)}{0.d_1d_2...d_kd_{k+1}d_{k+2}...}\right|\\
&= \left|\frac{0.d_1d_2...d_kd_{k+1}d_{k+2}... - 0.d_1d_2...d_k -2^{-k} }{0.d_1d_2...d_kd_{k+1}d_{k+2}...}\right|\\ 
&=\left| \frac{0.d_{k+1}d_{k+2}...\times2^{-k}-2^{-k}}{0.d_1d_2...d_kd_{k+1}d_{k+2}...} \right|\\
&=\left| \frac{(0.d_{k+1}d_{k+2}...-1)\times2^{-k}}{0.d_1d_2...d_kd_{k+1}d_{k+2}...} \right|\\
&=\left| \frac{(1-0.d_{k+1}d_{k+2}...)\times2^{-k}}{0.d_1d_2...d_kd_{k+1}d_{k+2}...} \right|\\
&=\left| \frac{(1-0.1d_{k+2}...)\times2^{-k}}{0.d_1d_2...d_k1d_{k+2}...} \right|
\end{align*}
Nuevamente, buscamos una cota superior buscando el mayor numerador y el menor denominador. Para el numerador, se obtiene buscando el valor mínimo para $0.1d_{k+2}...$, el cual es $0.1000...$ . Para el denominador, ese valor es $0.00...0100...$, por lo que nos sirve $0.1$ para la cota, puesto que es menor. Por lo tanto,
\begin{align*}
    |\varepsilon| &< \left| \frac{(1-0.1)\times2^{-k}}{0.1} \right|\\
    |\varepsilon| &< \left| \frac{0.1\times2^{-k}}{0.1} \right|\\
    |\varepsilon| &< 2^{-k}
\end{align*}

\item Redondeo por corte.\\
$$\hat{x}=0.d_1d_2...d_k\times2^n$$
\begin{align*}
|\varepsilon| &= \left|\frac{x - \hat{x}}{x}\right|\\
&= \left|\frac{0.d_1d_2...d_kd_{k+1}d_{k+2}...\times2^n - 0.d_1d_2...d_k\times2^n}{0.d_1d_2...d_kd_{k+1}d_{k+2}...\times2^n}\right|\\
&= \left|\frac{0.d_1d_2...d_kd_{k+1}d_{k+2}... - 0.d_1d_2...d_k}{0.d_1d_2...d_kd_{k+1}d_{k+2}...}\right|\\
&=\left| \frac{0.d_{k+1}d_{k+2}...\times2^{-k}}{0.d_1d_2...d_kd_{k+1}d_{k+2}...} \right|\\
\end{align*}
De la misma manera, buscamos una cota superior, hallando un valor máximo en el numerador y un valor mínimo en el denominador. Para el numerador, ese valor es $0.111...$, que es equivalente a $1$. Para el denominador, ese valor es $0.1000...$, puesto que $x$ esta en notación punto flotante.  
\begin{align*}
    |\varepsilon| &< \left| \frac{1\times2^{-k}}{0.1} \right|\\
    |\varepsilon| &< 2^{-k+1}
\end{align*}
\end{enumerate}

\section{}

\begin{equation*}
x=\frac{4}{5}=0.8
\end{equation*}
\begin{align*}
0.8\times2&=\boxed{1}.6\\
0.6\times2&=\boxed{1}.2\\
0.2\times2&=\boxed{0}.4\\
0.4\times2&=\boxed{0}.8\\
\vdots
\end{align*}
\begin{equation*}
x=\frac{4}{5}=0.8_{10}=0.\overline{1100}_{2}
\end{equation*}
Para un computador de 32 bits, supongamos que la distribución de los 32 bits se realiza de la siguiente forma (como se mostró en la presentación):
\begin{equation*}
    \pm 0.d_1\underset {\text{23 bits físicos}}{\underbrace{d_2d_3...d_24}}\pm \underset {\text{7 bits exponente}}{\underbrace{e_1e_2...e_7}}
\end{equation*}
Primero, escribamos el número en notación punto flotante normalizada:
\begin{align*}
    x=\left(\frac{4}{5}\right)_{10}=\left(+0.11001100...\times2^{+0}\right)_2
\end{align*}
Por lo tanto, la representación de este número en el computador de 32 bits sería
\begin{equation*}
    \underset {\text{signo}}{\underbrace{1}}\underset {\text{mantisa}}{\underbrace{10011001100110011001100}}\underset{\text{signo exponente}}{\underbrace{1}}\underset{\text{ exponente}}{\underbrace{0000000}}\rightarrow 11001100110011001100110010000000
\end{equation*}
Para calcular el error absoluto, se determinará el número representado por el computador y se calculará la diferencia entre éste y $0.8$. Luego,
\begin{align*}
    11001100110011001100110010000000&\longrightarrow0.110011001100110011001100\\
    \hat{x}=(0.110011001100110011001100)_2&=(1\times2^{-1}+1\times2^{-2}+0\times2^{-3}+...+0\times2^{-24})_{10}\\
    &=0.799999952316284
\end{align*}
Ahora
\begin{align*}
    E &=x-\Hat{x}= 0.8-0.799999952316284=4.768371608676603\times10^{-8}\\\varepsilon&=\frac{E}{x}=\frac{4.768371608676603\times10^{-8}}{0.8}=5.960464510845753\times10^{-8}
\end{align*}
\section{}
Junto a este documento está el código en Matlab para calcular el épsilon y el número más grande del computador, junto con un toolbox \cite{vpi} que fue utilizado para simplificar los cálculos. Se puede instalar directamente.
