%-------------------------------------------------------------
% Language
% Use the option "language=EN" to set the beamer theme in English. Use
% the option "language=ES" to set the beamer theme in Spanish.

% Colors
% Use the option "color=white" to set the background in white and the
% bottom bar in blue. Use the option "color=blue" to set the
% background in blue and the bottom bar in white. Use the option
% "color=blue2" to set the background in blue and the bottom bar in
% blue.

% Font Color
% Use the option "fontc=black" to set the font color in black. If this
% argument is not given the default color is set depending of the
% color scheme selected.

% Notes:
% Do not use \large inside multicols
% Enumerate require \justifying command
% Tables captions bellow the tabular
% With citations, note = {} may only work for @misc

% Credits: https://github.com/alejogm0520 & Samuel Plazas Escudero
%-------------------------------------------------------------

%--Principal packages
\documentclass[xcolor=table, aspectratio=43,8pt]{beamer} % 4:3; can be 16:9; [...,8pt,t] in order to start text of all frames on the upper part; add: draft to not compile figures.
\usetheme[language=ES, color=white]{EAFIT}
\usepackage[spanish]{babel}
\usepackage[utf8]{inputenc}

\usepackage{amsmath,amsfonts,amssymb,cancel}
\newcommand{\overbar}[1]{\mkern 1.5mu\overline{\mkern-0.5mu#1\mkern-0.5mu}\mkern 1.5mu}

% Equations; physics is optional and sometimes problematic!
\usepackage{verbatim} % Environments, \begin{comment}
%--Arial
\usepackage{helvet}\renewcommand{\familydefault}{\sfdefault} % It's ok
%--David Plazas recommended
%\usepackage{libertine} % Normal
%--Carlos Cuartas 
%\usepackage[T1]{fontenc}\usepackage{lmodern} % Best
%--Beamer packages
\usepackage{tikz} % For making vectorized figures, arrows
\usepackage{ifthen} % For specifying conditionals for sections
\usepackage{ragged2e}\justifying % Whole text justified, except enumerate: add \justifying
\usepackage{multicol} % Multiple columns in one frame
%--Tables-Figures


\usepackage{booktabs,multirow} % Bookstyle tables
\usepackage{array} % Custom width and centered
\newcolumntype{P}[1]{>{\centering\arraybackslash}p{#1}} % horizontal centering but use custom width
\newcolumntype{M}[1]{>{\centering\arraybackslash}m{#1}} % horizontal and vertical centering but use custom width
%-Figure label
\usepackage[labelsep=period,justification=justified,format=plain]{caption} % Dot instead of colon and justified caption
%--Figure
\usepackage{graphicx,subcaption} % Figures and subfigures
\usepackage{media9} % video and audio
%-Figure-Table on top
\usepackage{float} % Allows to put H instead of ht
\setbeamertemplate{caption}[numbered] % Numbered captions
%---------TOC
\setbeamertemplate{section in toc}[sections numbered]
\setbeamertemplate{subsection in toc}[subsections numbered]
\setbeamerfont{section in toc}{size=\small}
\setbeamerfont{subsection in toc}{size=\footnotesize}
\setbeamertemplate{subsection in toc}{\leavevmode\leftskip=3.2em\rlap{\hskip-2em\inserttocsectionnumber.\inserttocsubsectionnumber}\inserttocsubsection\par} % Indented subsection
\setcounter{tocdepth}{2} % Toc depth, put 1 for only showing there the sections and 2 to include sections
%---------Cite
\usepackage{bibentry} % Full cite foot
\nobibliography* % Full cite foot
\setbeamertemplate{bibliography item}[triangle]% [online][book][article][triangle][text]; Or: \setbeamertemplate{bibliography item}{\insertbiblabel} 
\usepackage{etoolbox} % Package for using justified bibliography 
\apptocmd{\thebibliography}{\justifying}{}{} % Justified bibliography 
%---------Footnotes
\setbeamercolor{footnote}{fg=white} % Footnote white 
\setbeamercolor{footnote mark}{fg=.} % Takes the color depending on the circumpstance
\setbeamercolor{bibliography entry author}{fg=white} % Allows to have white footnote bibs
\setbeamertemplate{footnote}
{
  \hspace*{-1cm} % Horizontal movement
  \vspace*{-3.12cm} % Vertical movement
  \parbox[c][3.64cm]{10.6cm}{\tiny\noindent\insertfootnotemark\insertfootnotetext} % b: bottom, height: 3.3cm, horizontal length: 10.6cm (max horizontal)
% If there are problems, put \vspace*{-2.87cm} and \parbox[c][3.3cm]
% or \vspace*{-2.88cm} and \parbox[c][3.4cm]
% or \vspace*{-3.05cm} and \parbox[c][3.6cm]
% or \vspace*{-3.12cm} and \parbox[c][3.64cm]
}
\renewcommand{\footnoterule}{\kern -3pt \hrule width \textwidth height 0pt\kern 3pt} % No footnoterule
\usepackage{perpage}\MakePerPage{footnote} % Footnote numbered per frame
\renewcommand{\thefootnote}{\Roman{footnote}} % Roman number in footnote
                                              % Cutom: \fnsymbol{footnote}
%------------------------------------
%---------Numbered Slides and Sections
\setbox0=\hbox{\subsecname\unskip}\ifdim\wd0=0pt\else%
 ~--~\insertsubsectionhead
\fi
%------Numbering section: title in bold, centered and with a line
\newcommand{\numb} 
{
  \setbeamertemplate{frametitle}
  {
    \ifx\insertsubsection\empty % No subsection
         \bfseries\thesection.~\insertframetitle~\color{black}\par\vskip-5pt\hrulefill % \centering
    \else % subsection
         \bfseries\thesection.~\insertframetitle~\color{black}\par\vskip-9pt\hrulefill\par\vskip3pt{\large\thesection.\thesubsection~\insertframesubtitle} % Subsection with smaller size;
    \fi
  }
}
%------No numbering section: title in bold, centered and with a line
\newcommand{\nonumb}
{
  \setbeamertemplate{frametitle}{\bfseries\color{black}\centering\insertframetitle\par\vskip-6pt\hrulefill}
}
%------------------------------------
%--No hyphenation on text
\tolerance=1
\emergencystretch=\maxdimen
\hyphenpenalty=10000
\hbadness=10000
%------------------------
%---------Itemize justified in beamer
\makeatletter
\renewcommand{\itemize}[1][]{
  \beamer@ifempty{#1}{}{\def\beamer@defaultospec{#1}}
  \ifnum \@itemdepth >2\relax\@toodeep\else
    \advance\@itemdepth\@ne
    \beamer@computepref\@itemdepth % Sets \beameritemnestingprefix
    \usebeamerfont{itemize/enumerate \beameritemnestingprefix body}
    \usebeamercolor[fg]{itemize/enumerate \beameritemnestingprefix body}
    \usebeamertemplate{itemize/enumerate \beameritemnestingprefix body begin}
    \list
      {\usebeamertemplate{itemize \beameritemnestingprefix item}}
      {\def\makelabel##1{
          {
            \hss\llap{{
                \usebeamerfont*{itemize \beameritemnestingprefix item}
                \usebeamercolor[fg]{itemize \beameritemnestingprefix item}##1}}
          }
        }
      }
  \fi
  \beamer@cramped
  \justifying % Justified itemize
  \beamer@firstlineitemizeunskip
}
\makeatother
%------------------------
%---------get current section name for showing it at its begining
\usepackage{nameref}
\makeatletter
\newcommand*{\currentname}{\@currentlabelname}
\makeatother
%---------Shows in which section we are at the begining of each one
\begin{comment}
\AtBeginSection[]
{
\begin{frame}[plain,noframenumbering]
  \begin{beamercolorbox}[ht=\paperheight,wd=\paperwidth, center]{Portada}
    \begin{center}\textbf{\LARGE \currentname}\end{center} % Leave the next space mandatorily

    \vspace{0.44\paperheight}
  \end{beamercolorbox}
\end{frame}
}
\end{comment}

%-------------------(CONSTANTLY BEING EDITED)------------------
%---------TEXTBLOCKS-GRID 
\usepackage[absolute,overlay,showboxes]{textpos}
%\usepackage[texcoord,grid,gridunit=mm,gridcolor=red!10,subgridcolor=green!10]{eso-pic} % Helping grids, comment when publishing
%---------NOTES IN BEAMER
\AtBeginNote{\Huge}\newcommand{\notei}[1]{\note[item]{\Huge{\textcolor{blue}{#1}}}} % Use \notei{text} everywhere % [1] means one parameter located in #1 (input). 
\setbeamertemplate{note page}[plain] % Plain style for notes page
\setbeameroption{show notes} % {show notes} or {hide notes}
% \setbeameroption{show notes on second screen=right}
% as well you can use \documentclass[notes=only] at the beginning of the code
%-----------More elaborated notes
%\setbeamercolor{note page}{bg=white!90!black, fg=black}
%\setbeamercolor{note title}{bg=white!30!red, fg=black}
%\setbeamercolor{note date}{parent=note title}
%---------Itemize, enumberate and lists inside them
%\setbeamertemplate{itemize/enumerate body begin}{\LARGE} % Body
\setbeamertemplate{itemize/enumerate subbody begin}{\Large} % Subbody
%---------COLOR DEFINITIONS
\definecolor{azure(colorwheel)}{rgb}{0.0, 0.5, 1.0} % Define colors here
\definecolor{blue(ryb)}{rgb}{0.01, 0.28, 1.0}

\usepackage[normalem]{ulem}
\useunder{\uline}{\ul}{}


\usepackage{mathtools}
\DeclarePairedDelimiter{\ceil}{\lceil}{\rceil}
%%%%%%%%%%%%%%%%%%%%%%%
%Start of the Document%
%%%%%%%%%%%%%%%%%%%%%%%

%---------COVER PAGE
  \title{ANÁLISIS DE CLASIFICACIÓN}
\author{\normalfont\texorpdfstring{Presentado por:\\Mateo Restrepo S.\\Juan S. Cárdenas R. \\David Plazas E.\\[1ex]Prof.:\\ Francisco I. Zuluaga D.}{}}

\def\eafit{Universidad EAFIT}
\def\materia{Estadística II}
\def\fecha{2019} % or put the exact date
% to add more def, search for "Dirección" in beamerthemeEAFIT.sty

%\includeonly{Slides/0_cover_title,ex_beamer,Slides/refs_thanks}
\begin{document}
\nonumb % Not numbered titles
\begin{frame}
% Portada Inspira Crea Transforma
\end{frame}
%%%%%%%%%%%%%%%%%%%%%%%%%%%%%%%%%%%%%%%%%%%%%%%%%%%%%%%%%%%%%%%%%%%%%%%%%%%%
\begin{frame}
\begin{center}
  \titlepage % Cover page
\end{center}
\end{frame}
%%%%%%%%%%%%%%%%%%%%%%%%%%%%%%%%%%%%%%%%%%%%%%%%%%%%%%%%%%%%%%%%%%%%%%%%%%%%
\begin{frame}{CONTENIDO}
\begin{multicols}{2}
  \tableofcontents
\end{multicols}
\end{frame}
\numb % Numbered titles
%\begin{comment}

\section{INTRODUCCIÓN}
\begin{frame}{INTRODUCCIÓN}
    \begin{itemize}
        \item Ingeniería y computación $\rightarrow$ \textit{Reconocimiento de patrones.}
        \item \textit{Análisis de Clusters.}
        \item Clasificar nuevas observaciones en grupos ya establecidos.
    \end{itemize}
    Se tienen $k$ grupos, de donde se extraen muestras para obtener $\mathbf{\overbar{y}_1,\,\overbar{y}_2,\,\dots,\overbar{y}_k}$. Se pretender acomodar una observación $\mathbf{y}$ en alguno de los $k$ grupos; una aproximación intuitiva es compararla con las medias $\mathbf{\overbar{y}_i}$, mirando cuál es la más cercana.
    \\\vspace{5mm}
    \textbf{Ejemplos:}
    \begin{itemize}
        \item Clasificación de aplicantes a una Universidad, entre los que desertarán y los que no.
        \item Orientación vocacional basada en tests de aptitudes.
        \item Clasificación de ciudades violentas (estudio previo de indicadores).
    \end{itemize}
\end{frame}

\section{CLASIFICACIÓN EN DOS GRUPOS}
\subsection{Método de Fisher}
\begin{frame}{CLASIFICACIÓN EN DOS GRUPOS}
    \framesubtitle{Método de Fisher}
    \begin{itemize}
        \item Se tienen dos grupos $G_1$ y $G_2$.
        \item Se requiere que $\Sigma_1=\Sigma_2$.
        \item No requiere que $\mathbf{y}_i\sim N_p(\pmb{\mu},\Sigma)$.
        \item Se calcula $\mathbf{\overbar{y}}_1$, $\mathbf{\overbar{y}}_2$.
        \item Se basa en la función discriminante:\begin{equation}
            z=\mathbf{a}'\mathbf{y}=\left(\mathbf{\overbar{y}_1}-\mathbf{\overbar{y}_2}\right)'\mathbf{S}_{\text{pl}}^{-1}\mathbf{y}
        \end{equation}
        \item Se calculan $\mathbf{\overbar{z}}_1$ y $\mathbf{\overbar{z}}_2$ para determinar si $\mathbf{y}\in G_1$ o $\mathbf{y}\in G_2$.
    \end{itemize}
    \begin{align*}
        \left[z>\dfrac{1}{2}\left(\mathbf{\overbar{z}}_1+\mathbf{\overbar{z}}_2\right)\right]\implies \mathbf{y}\in G_1\\
        \left[z<\dfrac{1}{2}\left(\mathbf{\overbar{z}}_1+\mathbf{\overbar{z}}_2\right)\right]\implies \mathbf{y}\in G_2
    \end{align*}
\end{frame}

\subsection{Regla Óptima}
\begin{frame}{CLASIFICACIÓN EN DOS GRUPOS}
\framesubtitle{Regla Óptima}
Sean dos grupos $G_1$ y $G_2$ tal que $\Sigma_1=\Sigma_2=\Sigma$. Si se conocen las probabilidades $p_1$ y $p_2$ asociadas a las poblaciones y las respectivas funciones de densidad $f_{G_1}(\mathbf{y})$ y $f_{G_2}(\mathbf{y})$, se puede aprovechar esta información para una mejor clasificación.

\begin{itemize}
    \item Minimizar el error de clasificación. \item El criterio de asignación óptima de $\mathbf{y}$ en $G_1$ es:
    \begin{equation}
        p_1f_{G_1}(\mathbf{y})>p_2f_{G_2}(\mathbf{y})
    \end{equation}
    \item Si se cumple que $f_{G_1}(\mathbf{y})=N_p(\pmb{\mu}_1,\Sigma)$ y $f_{G_2}(\mathbf{y})=N_p(\pmb{\mu}_2,\Sigma)$, entonces la regla se transforma en:
    \begin{align*}
        \left[z>\dfrac{1}{2}\left(\mathbf{\overbar{z}}_1+\mathbf{\overbar{z}}_2\right)+\ln\left(\dfrac{p_2}{p_1}\right)\right]\implies \mathbf{y}\in G_1\\
        \left[z<\dfrac{1}{2}\left(\mathbf{\overbar{z}}_1+\mathbf{\overbar{z}}_2\right)+\ln\left(\dfrac{p_2}{p_1}\right)\right]\implies \mathbf{y}\in G_2
    \end{align*}
    \item Criterio asintóticamente óptimo. \item Si $p_1=p_2\rightarrow$ Fisher.
    
\end{itemize}

\end{frame}

\section{CLASIFICACIÓN EN VARIOS GRUPOS}
\subsection{Grupos con Igual Covarianza Poblacional}
\begin{frame}{CLASIFICACIÓN EN VARIOS GRUPOS}
\framesubtitle{Grupos con Igual Covarianza Poblacional}
    \begin{itemize}
        \item Se tienen $k$ grupos tales que $\Sigma_1=\Sigma_2=\dots=\Sigma_k=\Sigma$, con vectores de medias $\mathbf{\overbar{y}}_1,\,\dots,\mathbf{\overbar{y}}_k$.
        \item Se utiliza una función de distancia para encontrar el vector de medias más cercano y asignarlo a su población.
        \item Se estima la matriz $\Sigma$ con:
        \begin{equation}
         \mathbf{S}_{\text{pl}}=\dfrac{1}{N-k}\sum_{i=1}^k(n_i-1)\mathbf{S}_i
        \end{equation}
        \item Utiliza la función lineal de clasificación:
        \begin{equation}
            L_i(\mathbf{y})=\mathbf{\overbar{y}}_i'\mathbf{S}_{\text{pl}}^{-1}\left(\mathbf{y}-\dfrac{1}{2}\mathbf{\overbar{y}}_i\right)
        \end{equation}
        \item Se asigna $\mathbf{y}$ al grupo $G_{j}$ tal que $j$ cumple
        \begin{equation}
            L_j=\max_{i}\{L_i(\mathbf{y})\}
        \end{equation}
    \end{itemize}
\end{frame}

\begin{frame}{CLASIFICACIÓN EN VARIOS GRUPOS}
    \framesubtitle{Grupos con Igual Covarianza Poblacional}
    De igual forma que en clasificación de dos grupos, si se conocen las probabilidades $p_i$ y las funciones de densidad $f_{G_i}(\mathbf{y})$ para cada uno de los $k$ grupos, se tiene la siguiente regla óptima:
    \begin{center}
        Asigne $\mathbf{y}$ al grupo en el cual $p_i\,f_{G_i}(\mathbf{y})$ es máximo.
    \end{center}
    Este criterio minimiza el error de asignación. Asumiendo normalidad, $f_{G_i}(\mathbf{y})=N_p(\pmb{\mu}_i,\Sigma)$, la función lineal de clasificación se transforma en:
    \begin{equation}
        L^*_i(\mathbf{y})=\ln p_i + L_i(\mathbf{y})
    \end{equation}
    Si $p_1=p_2=\dots=p_k$
    \begin{equation}
        \max_{i}\{L_i(\mathbf{y})\} \equiv \max_{i}\{L_i^*(\mathbf{y})\}
    \end{equation}
\end{frame}

\subsection{Grupos con Diferente Covarianza Poblacional}
\begin{frame}{CLASIFICACIÓN EN VARIOS GRUPOS}
\framesubtitle{Grupos con Diferente Covarianza Poblacional}
    \begin{itemize}
        \item En general es difícil que $\Sigma_1=\Sigma_2=\dots=\Sigma_k=\Sigma$.
        \item En este caso, no es posible reducir a una función lineal de clasificación $\rightarrow$ función de clasificación cuadrática.
        \item Este método sí requiere normalidad.
        \item Función de clasificación cuadrática:
        \begin{equation}
            Q_{i}(\mathbf{y})=\ln p_i -\dfrac{1}{2}\ln|\mathbf{S}_i|-\dfrac{1}{2}\left(\mathbf{y}-\mathbf{\overbar{y}}_i\right)'\mathbf{S}_i^{-1}\left(\mathbf{y}-\mathbf{\overbar{y}}_i\right)
        \end{equation}
        \item Criterio de clasificación: 
        \begin{center}
            Asigne $\mathbf{y}$ al grupo que proporcione el máximo $Q_i(\mathbf{y})$
        \end{center}
        \item Este método requiere que $(\forall i=1,\dots,k)(n_i>p)\rightarrow$ existencia $\mathbf{S}_i^{-1}$.
    \end{itemize}
\end{frame}

\section{ERROR DE ESTIMACIÓN}
\subsection{Matriz de Confusión}
\begin{frame}{ERROR DE ESTIMACIÓN}
    \framesubtitle{Matriz de Confusión}
    \begin{itemize}
        \item El error de estimación $\epsilon$ se define como la probabilidad que una función de clasificación se equivoque asignando $\mathbf{y}$. Este se puede estimar con la matriz de confusión.
        \item La matriz de confusión $M$, es una matriz de orden $(k \times k)$. Esta se calcula usando la observación de cada uno de los grupos utilizados y la función de clasificación, contando a qué grupo asigna las observaciones. 
        \item Si $n_{ij}$ es el numero de veces que la función de clasificación asignó una observación del grupo $i$ al grupo $j$ y $N_i$ es la cantidad de observaciones hechas en el grupo $i$, entonces el error de estimación se calcula como:
        \begin{equation}
            \epsilon = 1 - \dfrac{\sum_{i = 1}^{k} n_{ii}}{\sum_{i = 1}^k N_i} = 1 - \dfrac{\text{tr}(M)}{\sum_{i = 1}^k N_i} 
        \end{equation}
    \end{itemize}
\end{frame}
\section{EJEMPLOS}
\begin{frame}{EJEMPLOS}

\textbf{Clasificación entre hombres y mujeres midiendo algunos rasgos psicológicos.} \\\vspace{0.5cm}
Se tienen 32 observaciones de cuatro diferentes factores psicológicos medidos para hombres y mujeres. Diga si la siguiente observación pertenece al grupo masculino o femenino:
    \begin{equation}
        \mathbf{y}=\begin{bmatrix}
            11\\
            17\\
            15\\
            23
        \end{bmatrix}
    \end{equation}

Implementación en R.
\end{frame}

\begin{frame}{EJEMPLOS}
\textbf{Medidas de la cabeza de jugadores de fútbol americano.} \\\vspace{0.5cm}
Se tienen 30 medidas de jugadores de fútbol americano de bachillerato, de universidad y de personas que no juegan fútbol americano sobre 6 aspectos de la cabeza tales como ancho, longitud y más. Construya una función que clasifique a que grupo pertenece una observación nueva y halle su matriz de clasificación.

Implementación en R.
\end{frame}

\nonumb % Not numbered titles
%\addcontentsline{toc}{section}{\small\protect\numberline{}{REFERENCIAS BIBLIOGRÁFICAS}} % Separated from other contents, for small number of contents
\addcontentsline{toc}{section}{\small REFERENCES} % Closer from other contents, for large number of contents
\nocite{*} % All citations showed (take care with fraud!)
%%%%%%%%%%%%%%%%%%%%%%%%%%%%%%%%%%%%%%%%%%%%%%%%%%%%%%%%%%%%%%%%%%%%%%%%%%%%
\section*{REFERENCES}
\begin{frame}[allowframebreaks]{REFERENCES} %  and put before {REFEREN...}
\begingroup % Group for changing the color
\renewcommand{\color}[1]{} % Allows to have black bibs and white footnote bibs
\small{\bibliographystyle{IEEEtran}} % Size of text; acm or gatech-thesis or ieeetr or ieeetran or icontec or iso690
\bibliography{ref}
\endgroup % Group for changing the color
% pdflatex -> bibtex -> pdflatex -> pdflatex
\end{frame}
%%%%%%%%%%%%%%%%%%%%%%%%%%%%%%%%%%%%%%%%%%%%%%%%%%%%%%%%%%%%%%%%%%%%%%%%%%%%
% Thank-slide
\begin{frame}[plain,noframenumbering] % No frame number
	\begin{beamercolorbox}[ht=\paperheight,wd=\paperwidth, center]{Portada}
		\begin{center}\Huge\textbf{Thank you}\end{center} % Or Thanks; leave the next space mandatorily
		
		\vspace{0.44\paperheight}
    \end{beamercolorbox}
\end{frame}
%%%%%%%%%%%%%%%%%%%%%%%%%%%%%%%%%%%%%%%%%%%%%%%%%%%%%%%%%%%%%%%%%%%%%%%%%%%%


\end{document}