\section{INTRODUCTION}
Diabetes is defined as ``Diabetes is a chronic, metabolic disease characterized by elevated levels of blood glucose (or blood sugar)'' \cite{whodiabetes}. It can affect any kind of person, independent from his gender or age. Some studies suggest that the prevalence of diabetes will increase by 35\% between 1995 and 2025. Additionally, the World Health Organization (WHO) and the World Bank consider diabetes as a public health threat \cite{altamirano2001epidemiology}. Furthermore, \cite{atlas2015international} shows that diabetes can generate more complicated diseases, reducing life expectancy. 
On the other hand,  regarding obesity:

\begin{quote}
\textit{``Overweight and obesity are defined as abnormal or excessive fat accumulation that presents a risk to health. A crude population measure of obesity is the body mass index (BMI), a person’s weight (in kilograms) divided by the square of his or her height (in metres). A person with a BMI of 30 or more is generally considered obese. A person with a BMI equal to or more than 25 is considered overweight.
Overweight and obesity are major risk factors for a number of chronic diseases, including diabetes, cardiovascular diseases and cancer. Once considered a problem only in high income countries, overweight and obesity are now dramatically on the rise in low- and middle-income countries, particularly in urban settings.''}\cite{whoobesity}
\end{quote}

The goal of this work is to extend the model proposed in \cite{netlogomodel}, enhancing it with interpersonal relationships. The original model characterizes the dynamics of diabetes patients in the U.S. given their eating habits and the influence that they have on the glucose level. The population is classified into three types: healthy, risky and diabetic; individuals can change their state at any moment.

\cite{JCI10842} affirms that there is an association between obesity and diabetes since obese individuals develop resistance to insulin. Motivated by the above statement,  we propose an extension that consists of how social interactions affect eating habits and therefore diabetes.  To achieve this, \cite{jeffery2002cross} was studied for linking marriage and levels of BMI; moreover, in \cite{bays2007relationship} the association of BMI and diabetes is presented. 