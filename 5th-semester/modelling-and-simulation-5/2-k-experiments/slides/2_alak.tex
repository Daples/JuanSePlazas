\section{Factorial Design $\pmb{2^k}$}
\subsection{Yates Notation}
\begin{frame}{Factorial Design $\pmb{2^k}$}
\framesubtitle{Yates Notation}
    This notation can be extended for any value of $k$.
\begin{table}[H]
\begin{tabular}{lccc}
\hline
      & Factor $A$ & Factor $B$ & Factor $C$ \\ \hline
$(1)$ & Low        & Low        & Low        \\
$a$   & High       & Low        & Low        \\
$b$   & Low        & High       & Low        \\
$c$   & Low        & Low        & High       \\
$ab$  & High       & High       & Low        \\
$ac$  & High       & Low        & High       \\
$bc$  & Low        & High       & High       \\
$abc$ & High       & High       & High       \\ \hline
\end{tabular}
\end{table}
    Normally we are interested in treatment with a singular or double factor.
\end{frame}

\subsection{The Best ANOVA}
\begin{frame}{Factorial Design $\pmb{2^k}$}
    \framesubtitle{The Best ANOVA}
It's necessary to make first ANOVA to select which factors don't have a significant impact.\\ \pause

Let $\chi^*$ be the selected factors and $k^*$ the number of selected factor. Then, the \textbf{best ANOVA} test is made:

\begin{align*}
    T_{0\chi^*} &= \dfrac{\text{Effect } \chi^*}{\sqrt{\dfrac{\text{AS}_e}{n2^{k^*-2}}}}, \quad T \sim t_{(\alpha,2^{k^*}(n-1))}\\ \vspace{0.5cm}
    P_{0\chi^*} &= \mathrm{P}(T>T_{0\chi^*})\\\vspace{0.5cm}
\end{align*}
The null hypothesis ($H_0$) is rejected if $P_{0\chi^*} < \alpha$

\end{frame}

\begin{frame}{Factorial Design $\pmb{2^k}$}
\framesubtitle{Non-Replicated $\pmb{2^k}$ Factorial Design}
\begin{itemize}
    \item For $k\geq5$, the experimentation can be unmanageable (e.g. $2\times2^5=64$ runs).
    \item Only one replica is desirable for $k\geq4$.
    \item Make fractions of the complete experiments, analyze them separately and decide if it is necessary to continue the experimentation.
    \item Take samples to estimate the error and apply an approximated ANOVA procedure.
\end{itemize}
\begin{table}[]
\centering
\begin{tabular}{ccc}
\hline
\textbf{Design} & \textbf{Replicas}                        & \textbf{Runs} \\ \hline
$2^2$           & 3 or 4                                   & 12, 16        \\
$2^3$           & 2                                        & 16            \\
$2^4$           & 1 or 2                                   & 16, 32        \\
$2^5$           & fraction $2^{5-1}$ or 1                  & 16, 32        \\
$2^6$           & fraction $2^{6-2}$ or fraction $2^{6-1}$ & 16, 32        \\
$2^7$           & fraction $2^{7-3}$ or fraction $2^{7-2}$ & 16, 32        \\ \hline
\end{tabular}
\end{table}
\end{frame}
