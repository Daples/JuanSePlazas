\documentclass[11pt]{article}
\usepackage[spanish]{babel}
\usepackage{geometry}
\usepackage{amsmath}
\usepackage{amsthm}
\usepackage{graphicx}
\usepackage[utf8]{inputenc}

%%%%%%%% HYPERREF PACKAGE
\usepackage{hyperref}
\hypersetup{citecolor=blue}
\hypersetup{urlcolor=blue}
\hypersetup{colorlinks=true}

%%%%%%%% DEFINITION AND THEOREM DEFINITIONS
\theoremstyle{definition}
\newtheorem{definition}{Definition}[section]

\theoremstyle{remark}
\newtheorem{remark}{Remark}

\theoremstyle{remark}
\newtheorem{question}{Question}

%%%%%%%% MULTI-COLUMNS PACKAGE
\usepackage{multicol}

%%%%%%%% START DOCUMENT

\title{Procesos Estocásticos}
\author{Juan Sebastián Cárdenas}
\date{\today}


\begin{document}
\maketitle
\section{Matriz de transición}
La estimación de la matriz de transición es muy sencilla, lo que se
hace es calcular cuantas personas tomaron cierta bebida el dia
anterior y al siguiente día tomaron cierta bebida al dia siguiente. De
esa manera, uno construye la matriz:

\begin{equation}
  \label{eq:1}
  \hat{N} =
  \begin{bmatrix}
    \hat{N}(\text{PS}_{i-1} \cap \text{PS}_i) && \hat{N}(\text{PS}_{i-1} \cap \text{CC}_i) \\
    \hat{N}(\text{CC}_{i-1} \cap \text{PS}_i) && \hat{N}(\text{CC}_{i-1} \cap \text{CC}_i) \\
  \end{bmatrix}
\end{equation}

Y, ya que tenemos esta matriz sumamos por fila y dividimos a cada
elemento de la respectiva fila. Esto se hace en pro de hacer una
aproximación de las probabilidades condicionales, dado que por ejemplo
en la fila 1 obtenemos:
\begin{equation}
  \label{eq:2}
  \hat{N_1} = \hat{N}(\text{PS}_{i-1} \cap \text{PS}_i) + \hat{N}(\text{PS}_{i-1} \cap \text{CC}_i) = \hat{N}(\text{PS}_{i-1})
\end{equation}

De esta forma, cuando dividimos por fila claramente obtenemos una
aproximación de la matriz de transición. Así, haciendo esto nos da
que:

\begin{equation}
  \label{eq:3}
  \hat{P} =
  \begin{bmatrix}
    0.596 && 0.404 \\
    0.198 && 0.802
  \end{bmatrix}
\end{equation}

\subsection{Vector $\pi$}
En primer lugar, es claro que la cadena es aperiódica dado que el
conjunto de todos los estados es cerrado e irreducible. De esta
manera, se asegura la existencia de este valor. Al mismo tiempo,
resolviendo el siguiente sistema de ecuaciones:

\begin{equation}
  \label{eq:4}
  \pi \hat{P} = \pi
\end{equation}

Con la restricción que $\pi_1 + \pi_2 = 1$. Así, hallamos los valores
lo cual nos da:
\begin{equation}
  \label{eq:5}
  \pi =
  \begin{bmatrix}
    0.329 && 0.671
  \end{bmatrix}
\end{equation}

\section{Cercanía a estado estacionario}
Dado que ya tenemos una aproximación de la matriz de transición y, se
sabe que los datos son tomados para 100 familias en cuestión de 365
días es posible evidenciar que tan cerca está por medio de calcular el
valor de la matriz de transición cuando se toman 365 pasos.

De esta manera, si calculamos esta matriz obtenemos:
\begin{equation}
  \label{eq:6}
  \hat{P}^{365} =
  \begin{bmatrix}
    0.329 && 0.671 \\
    0.329 && 0.671
  \end{bmatrix}
\end{equation}

Así, notamos que la cadena debe estar muy cerca al estado
estacionario, dado que esta matriz ya es independiente del estado
inicial (como se observa que independiente de si está tomando Coca
Cola o Pepsi tiene la misma probabilidad de cambiar al mismo
estado). Por lo tanto, \textbf{la cadena está cerca al estado
  estacionario}.

\section{Problema aplicado}
Supongamos dos casos, el caso 1 es cuando Pepsi no decide contratar a
la empresa y el caso 2 es cuando Pepsi decide contratarla. Para el
resto de los cálculos considérese que $n = 100 \cdot 10^6$, $g = 1 \$$
y $c = 500 \cdot 10^6$.

En el primer caso, tenemos entonces que la utilidad esperada sería:

\begin{equation}
  \label{eq:7}
  E(W_1) = 365ng \pi_1 = 1.2015258843 \cdot 10^{10}
\end{equation}

En el segundo caso, si la empresa que estos contratan cumple su
promesa tendríamos una nueva matriz de transición la cual sería:
\begin{equation}
  \label{eq:8}
  \dot{P} =
  \begin{bmatrix}
    0.646 && 0.354 \\
    0.198 && 0.802
  \end{bmatrix}
\end{equation}

Esta nueva matriz de transición, tiene una nueva distribución
estacionaria la cual es:

\begin{equation}
  \label{eq:9}
  \dot{\pi} =
  \begin{bmatrix}
    0.359 && 0.641
  \end{bmatrix}
\end{equation}

Lo cual, esto nos da una nueva utilidad esperada de:
\begin{equation}
  \label{eq:10}
  E(W_2) = 365ng \dot{\pi}_1 - c = 1.3103869495 \cdot 10^{10}
\end{equation}

Como obtuvimos que la utilidad esperada en el caso 2 es mayor a la
utilidad esperada en el caso 1 entonces concluimos que Pepsi debería
contratar a esta compañía.
\end{document}
