\documentclass[10pt,twoside]{article}
\usepackage[utf8]{inputenc}
\usepackage{natbib}
\usepackage{graphicx}
\usepackage{booktabs}
\usepackage{adjustbox}
\usepackage[table,xcdraw]{xcolor}
\usepackage{amsmath}
\usepackage{mathtools}
\usepackage[]{todonotes}
\usepackage{fancyhdr}
\usepackage{multirow}

% Idioma
\usepackage[english]{babel}

\usepackage[none]{hyphenat}

\usepackage{algorithm}
\usepackage{algpseudocode}

\newcommand{\ignore}[1]{}


\title{Taboo Search and Variable-Neighborhood Search for Heterogeneous Cumulatitive Capacitated Vehicle Routing Problem}
\author{\emph{Juan Sebastián Cárdenas}\\
\vspace{0.3cm}
\small{\tt{jscardenar@eafit.edu.co}}\\
Mathematical Science Department\\
School of Science\\
Universidad EAFIT\\
Medellín -- Colombia}
\date{\today}

\usepackage{anysize}
\marginsize{3cm}{2cm}{2cm}{3cm}

% Configurar encabezado y pie de página
\pagestyle{fancy}
\lfoot[\date{\today}]{\date{\today}}
\rfoot[\thepage]{\thepage}
\cfoot[]{}
\rhead[Cárdenas (2019)]{Heterogeneous Taboo Search and VNS}
\lhead[]{}

\fancypagestyle{firststyle}
{
   \fancyhf{}
   \fancyfoot[R]{Last update: August of 2019}
}

\sloppy



\begin{document}

\maketitle

\thispagestyle{firststyle}

\section{Solution Algorithms}\label{sec_alg}
For this work, different algorithms where used to solve the
problem. In first place, an adaptation of taboo search (TS)
\citep{glover1977} was used. Lastly, a variable-neighborhood search
(VNS) algorithm was used with three different types of operators:
2-opt between routes \citep{croes1958}, 2h-opt in the same route
\citep{bentley1992} and swap.

The initial solution for the implemented algorithms was a modification
of the \cite{clarke1964} algorithm. This code was implemented in a
previous worked. Lastly, the lower bound used was supposing that there
was an infinite amount of the fastest vehicle with unlimited capacity.

The 2h-opt operator uses two routes and two consecutive nodes of
each routes. Let that the nodes in the first route are $(i, i+1)$ and
the nodes in the second route are $(j, j + 1)$. Therefore, this
operator draws an edge between nodes $(i, j + 1)$ and $(i + 1, j)$;
additionally, it tries an other move drawing an edge between $(i, j)$
and $(i + 1, j + 1)$.

As for the 2-opt neighborhood two routes are needed, the centroid of
each route is calculated and for every route it is chosen the route
with the closest centroid. This was due to executing every possible
combination of two routes generated long execution times; furthermore,
this change still creates good solutions that improves the objective
function.

The 2.5-opt operator picks two pairs of consecutive nodes between the
same route and makes 3 possible moves to improve the objective
function. Using the same notation as before, the first move is trying
drawing an edge between $(i, j)$ and $(i + 1, j + 1)$. In second
place, it attempts to insert the node $j$ between $(i, i+1)$. Finally,
it attempts to insert the node $i + 1$ between $(j, j + 1)$.

For this operator, it is picked every pair of nodes that does not
include the base depot and it is compared with every pair of nodes
that are at least 4 positions away. This assures that moves that the
swap is going to execute, this operator does not repeat it so
execution time can be diminish.

Finally, the swap operator picks two pair of nodes and swaps
them. This operator does every possible pair of swaps, as it does not
present a mayor computational effort and, it is assured with this that
the best solution can be picked using this operation.

In this manner, taboo search takes advantage of the operations defined
above. In first place, it finds the best solution in the neighborhood
using the 2-opt operation as explained above. Furthermore, as pair of
routes were compared in this operation, it is tagged as taboo using
these pair of routes for 2 iterations. In this manner, Lastly, to intensify the
search for better solutions the swap operation is done
afterwards. This search stops when the algorithm has run for 15
iterations and it has not found a better solution than the best
solution found.

Finally, the VNS algorithm is easily done. In first place, it calls
the 2-opt operator and finds the best solution. This is done because,
is the only operator which operates between routes; therefore, doing
it first and then applying the other operators should yield a better
solution. In second place, it does the other two operators but
changing order of this operations. It starts by using swap first and
then 2h-opt, then it switches the order every iteration. This
algorithm stops as soon as it makes 5 iterations; this number was
chosen to reduce execution time.

\section{Computational Results}\label{sec_results}

In table \ref{tab:lsvns} it can be found a comparison of execution
times and objective value functions using the VNS and TS algorithm;
the best and worst case scenario for both algorithms are shown in bold
in the table.. It is important to notice that there is not a
predominant one, despite VNS having a lower objective function for
most of the cases.

Furthermore, although VNS for instances 16, 17 had a bigger execution
time the difference could be considered negligible as for every other
case this algorithm has the lowest execution time by far.

\begin{table}[H]
\centering
\begin{tabular}{llllll}
\hline
Instance & Nodes & Z Taboo    & Z VNS      & Time Taboo (s) & Time VNS (s) \\ \hline
1        & 50    & 48617.603  & 46848.122  & 1.23           & 0.83         \\
2        & 50    & \textbf{8079.274}   & 8079.274   & 1.1            & 0.87         \\
3        & 50    & 8188.127   & \textbf{7994.871}   & 1.71           & 1.16         \\
4        & 75    & 64610.015  & 62322.676  & 0.65           & 0.39         \\
5        & 75    & 11095.207  & 11348.977  & 0.71           & 0.46         \\
6        & 75    & 10572.525  & 10337.824  & 0.96           & 0.54         \\
7        & 100   & 85850.941  & 80977.093  & 5.47           & 3.51         \\
8        & 100   & 16481.176  & 15577.674  & 5.56           & 3.62         \\
9        & 100   & 16289.528  & 16217.551  & 10.26          & 4.81         \\
10       & 150   & 102838.607 & 107380.047 & 7.86           & 5.13         \\
11       & 150   & 22953.536  & 22835.773  & 9.31           & 5.58         \\
12       & 150   & \textbf{24909.707}  & 24149.597  & 11.15          & 6.36         \\
13       & 199   & 243441.659 & \textbf{231977.566} & 3.43           & 2.57         \\
14       & 199   & 151022.868 & 144220.052 & 3.57           & 2.97         \\
15       & 199   & 23132.397  & 23842.019  & 2.86           & 1.52         \\
16       & 120   & 148092.648 & 141177.565 & \textbf{19.41}          & \textbf{20.76}        \\
17       & 120   & 26888.726  & 26289.352  & 5.35           & 6.34         \\
18       & 120   & 25483.606  & 25666.333  & 4.04           & 4.15         \\
19       & 100   & 85931.899  & 88359.89   & 5.04           & 1.94         \\
20       & 100   & 11800.365  & 13133.084  & 2.59           & 1.57         \\
21       & 100   & 12938.639  & 13646.083  & 2.76           & 1.5          \\ \hline
\end{tabular}
\caption{Comparison of objective function and time of execution for TS and VNS.}
\label{tab:lsvns}
\end{table}

In this manner, it could be partially considered that the implemented
VNS algorithm works better as for the it generally outperformed the TS
algorithm. Furthermore, it is important to find a bigger data set to
test this as this is not sufficient data to make any meaningful
conclusions.

\subsection{Constructive Algorithm Comparison}
In table \ref{tab:cons} it can be found a comparison of the constructive algorithm and how much did the solution change with both algorithms. The percentages in the last two columns where calculated by using the equation:
\begin{equation}
  \begin{split}
    \text{ZTC} &= (Z_\text{c} - Z_{\text{TS}})/Z_\text{c} * 100\% \\
    \text{ZVC} &= (Z_\text{c} - Z_{\text{VNS}})/Z_\text{c} * 100\%
  \end{split}
\end{equation}

It is important to notice than, as expected, the solution given by
both algorithms is smaller than the one that the constructive
algorithm gave. This confirms that the algorithm was successfully
created the manner that at least solutions are getting better.

\begin{table}[]
\centering
\begin{tabular}{llll}
\hline
Instance & Z Constructive & ZTC    & ZVC    \\ \hline
1        & 59332.81       & 18.1\% & 21.0\% \\
2        & 9523.71        & 15.2\% & 15.2\% \\
3        & 8689.384       & 5.8\%  & 8.0\%  \\
4        & 71435.922      & 9.6\%  & 12.8\% \\
5        & 12930.801      & 14.2\% & 12.2\% \\
6        & 12762.878      & 17.2\% & 19.0\% \\
7        & 100106.215     & 14.2\% & 19.1\% \\
8        & 19829.366      & 16.9\% & 21.4\% \\
9        & 20341.601      & 19.9\% & 20.3\% \\
10       & 131668.171     & 21.9\% & 18.4\% \\
11       & 27959.171      & 17.9\% & 18.3\% \\
12       & 29764.186      & 16.3\% & 18.9\% \\
13       & 304886.044     & 20.2\% & 23.9\% \\
14       & 185074.361     & 18.4\% & 22.1\% \\
15       & 29353.426      & 21.2\% & 18.8\% \\
16       & 159004.997     & 6.9\%  & 11.2\% \\
17       & 28566.496      & 5.9\%  & 8.0\%  \\
18       & 28401.365      & 10.3\% & 9.6\%  \\
19       & 113912.299     & 24.6\% & 22.4\% \\
20       & 15417.25       & 23.5\% & 14.8\% \\
21       & 15755.619      & 17.9\% & 13.4\% \\ \hline
\end{tabular}
\caption{Comparison constructive algorithm with TS and VNS.}
\label{tab:cons}
\end{table}

\subsection{Lower Bound Comparison}\label{sec_lbc}
In table \ref{tab:lbc} can be found the comparison to the lower bound explained in the algorithm solutions. The percentages where calculated similarly to the previous comparison.

\begin{table}[]
\centering
\begin{tabular}{llll}
\hline
Instance & Z LB      & ZTC     & ZVC     \\ \hline
1        & 18103.263 & 168.6\% & 158.8\% \\
2        & 3354.104  & 140.9\% & 140.9\% \\
3        & 3354.104  & 144.1\% & 138.4\% \\
4        & 32178.306 & 100.8\% & 93.7\%  \\
5        & 5452.104  & 103.5\% & 108.2\% \\
6        & 5452.104  & 93.9\%  & 89.6\%  \\
7        & 36910.475 & 132.6\% & 119.4\% \\
8        & 6927.287  & 137.9\% & 124.9\% \\
9        & 6927.287  & 135.2\% & 134.1\% \\
10       & 54911.404 & 87.3\%  & 95.6\%  \\
11       & 11174.822 & 105.4\% & 104.4\% \\
12       & 11174.822 & 122.9\% & 116.1\% \\
13       & 76505.005 & 218.2\% & 203.2\% \\
14       & 76505.005 & 97.4\%  & 88.5\%  \\
15       & 13501.271 & 71.3\%  & 76.6\%  \\
16       & 71620.463 & 106.8\% & 97.1\%  \\
17       & 15813.791 & 70.0\%  & 66.2\%  \\
18       & 15813.791 & 61.1\%  & 62.3\%  \\
19       & 55021.459 & 56.2\%  & 60.6\%  \\
20       & 7514.282  & 57.0\%  & 74.8\%  \\
21       & 7514.282  & 72.2\%  & 81.6\%  \\ \hline
\end{tabular}
\caption{Comparison of lower bound with TS and VNS.}
\label{tab:lbc}
\end{table}

It is important to remark that although the lower bound chosen for
this problem is really small compared to the best possible value it
was possible to construct solutions below the 100\% mark. These means,
that his algorithms reduced the lower bound gap in a noticeable way.

\section{Conclusions}\label{sec_conclusions}
It was successfully implemented a taboo search and
variable-neighborhood search algorithm to solve the heterogeneous
capacitated cumulative vehicle routing problem. This algorithms obtain
a better solution, given an initial solution. Furthermore, it is
hypothesized that the VNS works better but a bigger data set is needed
to verify this claim.

{\small
\bibliographystyle{authordate1}

\bibliography{ref.bib}}


\end{document}
