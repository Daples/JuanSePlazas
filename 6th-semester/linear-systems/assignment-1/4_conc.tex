\section{Conclusions}\label{sec:conc}
    In this work, the Rössler system was analyzed based on the circuit proposed by Canals \textit{et al.}, simulated using Simulink and then the results were interpreted and justified. Regarding the goal of this research, values for both $R_a$ and $R_c$ which the system does not show desired behavior were successfully found, analyzing the system response to changes in these parameters, with the aid of bisection method. Regarding what was hypothesized, the first conjecture was wrong, as it was shown in section \ref{subsubsec:varparaA} and \ref{subsubsec:varparaC} and discussed in \ref{sec:resultAn}, since the response of the system remains chaotic for smaller values of the resistors. In contrast, the second conjecture was proven correct due to the fact that the step input, going from $0V$ to $1000V$, showed that the system stabilizes shortly after the input was applied, as shown in section \ref{subsubsec:step} and discussed in \ref{subsubsec:stepAn}.
    
    On the other hand, as the main objective was to find intervals where the system preserves its chaotic behavior, the procedure followed only allowed us to find an enclosing interval where the exact intervals must be, as this ``critical'' points are only near to values where the system loses its comprehensive behavior: for $R_a$, the enclosing interval shall be $[259.098131084k\Omega,\,10000k\Omega]$ and for $R_c$ is $[0.08400331675k\Omega,\,88.38868k\Omega]$; in order to find an effective interval, more values for $R_a$ and $R_c$ should be tested and constantly checking for positive Lyapunov exponents \cite{lyapunov}, until a negative exponent is found since it yields the non-chaotic behavior.
    