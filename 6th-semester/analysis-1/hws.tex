\documentclass[11pt]{article}
\usepackage[english]{babel}
\usepackage{geometry}
\usepackage{amsmath}
\usepackage{amsthm}
\usepackage{graphicx}
\usepackage[utf8]{inputenc}

%%%%%%%% HYPERREF PACKAGE
\usepackage{hyperref}
\hypersetup{linkcolor=blue}
\hypersetup{citecolor=blue}
\hypersetup{urlcolor=blue}
\hypersetup{colorlinks=true}

%%%%%%%% DEFINITION AND THEOREM DEFINITIONS
\theoremstyle{definition}
\newtheorem{definition}{Definition}[section]

\theoremstyle{remark}
\newtheorem{remark}{Remark}

\theoremstyle{remark}
\newtheorem{question}{Question}

\newtheorem{theorem}{Theorem}[section]

%%%%%%%% MULTI-COLUMNS PACKAGE
\usepackage{multicol}

%%%%%%%% SETS DEFINITIONS
\usepackage{amssymb}
\renewcommand{\O}{\mathbb{O}}
\newcommand{\N}{\mathbb{N}}
\newcommand{\Z}{{\mathbb{Z}}}
\newcommand{\Q}{{\mathbb{Q}}}
\newcommand{\R}{{\mathbb{R}}}

\newcommand{\ri}{\rightarrow}
\newcommand{\fOrd}{\O \ri \O}

%%%%%%%% START DOCUMENT

\title{Notes Real Analysis}
\author{Juan Sebasti\'an C\'ardenas-Rodríguez \\ \scalebox{0.7}{Mathematical Engineering, Universidad EAFIT}}
\date{\today}


\begin{document}
\maketitle

\section{Homework 3}
\subsection{Problem 27}
\begin{question}
  Let $S$ be a non-empty subset of R bounded from above that has no
  greatest element. Prove that the $\sup S$ is a cluster point
  of $S$.
\end{question}

\begin{proof}
  Let's suppose that $s = \sup S$ is not a cluster point of that
  set. Then, it happens that it exists a $\epsilon > 0$ such that:

  \begin{equation}
    |B(s, \epsilon) \cap S| < \infty
  \end{equation}

  There are two possibilities, that $|B(s, \epsilon) \cap S| = 0$ or
  $|B(s, \epsilon) \cap S| > 0$.

  \begin{itemize}
  \item Let's suppose that $|B(s, \epsilon) \cap S| = 0$
    In this manner, as $\epsilon > 0$ then it happens that:
    \begin{equation}
      s - \epsilon < s
    \end{equation}

    Then, because $s$ is the supremum of the set it happens that
    there exists a $x \in S$ such that:

    \begin{equation}
      s - \epsilon < x \le s < s + \epsilon
    \end{equation}

    Then, it is clear that $x \in B(s, \epsilon)$. But
    $|B(s, \epsilon) \cap S| = 0$. Contradiction.

  \item $|B(s, \epsilon) \cap S| > 0$
    Let's define the set:

    \begin{equation}
      A = \{|s - x| \text{ such that } x \in B(s, \epsilon) \cap S\}
    \end{equation}

    It is clear, that this set is different from empty, as the
    condition stated. And, furthermore, every element in $A$ is
    bigger than zero; as the absolute value is positive and
    $s \notin A$.

    Hence, let's define $r = (\min A)/2$. As $r > 0$, then by construction it is clear that:

    \begin{equation}
      |B(s, r) \cap S| = 0
    \end{equation}

    But, as shown before, this leads to a contradiction.
  \end{itemize}

  Then, $\sup S$ is a cluster point of $S$.
\end{proof}

\subsection{Problem 28}
\begin{question}
  Prove that a subset of a metric space is closed if and only if it
  contains all its cluster points.
\end{question}

\begin{proof}
  ($\rightarrow$) Let's suppose that $S$ is a closed subset of a metric
  space that does not contain all of its cluster points. Then, there
  exists a cluster point $x$ of $S$ such that $x \in S^c$. Also,
  as $S$ is a closed subset then $S^c$ is a open set.

  Furthermore, as $S^c$ is open it must occur that it exists a
  $\epsilon > 0$ such that:

  \begin{equation}
    B(x, \epsilon) \subset S^c
  \end{equation}

  Which implies that this ball does not contain any element of
  $S$. Then, $x$ is not a cluster point of $S$. Contradiction.

  ($\leftarrow$) Let's suppose that a set contains all of its cluster
  points and it is a open subset of a metric space. As $S$ is a open
  subset then for a point $x \in S$ it exists $\epsilon > 0$ that:

  \begin{equation}
    B(x, \epsilon) \subset S
  \end{equation}

  Then, let's suppose that $x$ is a cluster point of $S$. Now it is
  clear that $S^c \ne \emptyset$ because $S$ is a subset of the
  metric space, therefore it can not be the entire metric
  space. Furthermore, let $y \in S^c$ and let's define:

  \begin{equation}
    r = d(x, y) + \epsilon
  \end{equation}

  Now it is clear that $B(x, \epsilon) \subset B(y, r)$. This happens
  because for every $x' \in B(x, \epsilon)$ it happens that
  $d(x,x') < \epsilon$. Furthermore, it also happens that
  $d(y,x') \le d(y, x) + d(x,x')$. Then,
  $d(y,x') < d(y,x) + \epsilon$ so $x' \in B(y, r)$.

  And, as $x$ is a cluster point of $S$ then $B(x, \epsilon)$
  contains infinite points of $S$. But then $B(y, r)$ contains
  infinite points of $S$, hence $y$ is a cluster point. But $S$
  contains all of its cluster points. Contradiction.
\end{proof}

\subsection{Problem 33}
\begin{question}
  Let $E$ be a compact metric space, $\{U_i\}_{i \in I}$ a collection
  of a open subsets of E whose union is E. Show that there exists a
  real number $\epsilon > 0 $ such that any closed ball in $E$ of
  radius $\epsilon$ is entirely contained in at least one set $U_i$
\end{question}

\begin{proof}
  Let's suppose that then that for every $\epsilon > 0$ it exists a
  closed ball such that this ball is not entirely contained in any
  $U_i$. Then, the closed balls $\bar{B}(p_1, 1)$,
  $\bar{B}(p_2, 1/2)$, $\bar{B}(p_3, 1/3)$, \dots such that they are
  not entirely contained in any $U_i$ exist.

  In this manner, if we make the set $S = \{p_1, p_2, p_3, ...\}$. As
  it is a infinite subset of a compact metric space then it has at
  least one cluster point $p$. As $p \in E$, then this point must be
  in $U_j$ for some $j \in I$. And, as $U_j$ is open there exists a
  ball with radius $r$ such that:

  \begin{equation}
    B(p, r) \subset U_j
  \end{equation}

  And as $p$ is a cluster point of $S$ then this ball contains
  infinite points of that set. Because of Bolzano Weirtrass theorem,
  one could argue that a convergent sub-sequence of $S$ exists and
  converges to $p$. Therefore, one could select a $n$ big enough such
  that:

  \begin{equation}
    \frac{1}{n} < \frac{r}{2} \text{ with } d(p, p_n)< \frac{r}{2}
  \end{equation}

  In this manner, it is clear that
  $\bar{B}(p_n, 1/n) \subset B(p, r)$. This happens because for an
  $x \in \bar{B}(p_n, 1/n)$ it occurs that $d(p_n, x) <
  1/n$. Therefore:

  \begin{equation}
    \begin{split}
      d(p, x) &\le d(p, p_n) + d(p_n, x) \\
      &< \frac{r}{2} + \frac{1}{n} \\
      &< \frac{r}{2} + \frac{r}{2} = r
    \end{split}
  \end{equation}

  Therefore, $x \in B(p, r)$. Hence,
  $\bar{B}(p_n, \frac{1}{n}) \subset U_j$. But this ball was not
  entirely contained in any $U_i$. Contradiction.
\end{proof}

\subsection{Problem 36}
\begin{question}
  Call a metric space \textit{totally bounded} if, for every
  $\epsilon > 0$, the metric space is the union of a finite number of
  closed balls of radius $\epsilon$. Prove that a metric space is
  totally bounded if and only if every sequence has a Cauchy
  sub-sequence.
\end{question}

\begin{proof}
  ($\rightarrow$) Let's suppose that the metric space is totally
  bounded and that here exists a sequence that has no Cauchy
  sub-sequence. In this manner for some $\{x_i\}_{i \in \N}$ it must
  happen that for every $\epsilon > 0$ it exists an $N$ such
  that for every $m, n > N$ it occurs that
  $d(x_{i_n}, x_{i_m}) \ge \epsilon$.

  Furthermore, as $E$ is totally bounded one could argue that for
  some $\epsilon > 0$ it exists a finite set of points
  $Y = \{y_1, y_2, \dots, y_k\}$ such that:

  \begin{equation}
    E = \bigcup_{i=1}^kB(y_i, \epsilon)
  \end{equation}

  As there are more points of $x_i$ than $y_i$ and as $E$ is equal to
  the union of all of those open balls, then it must happen that at
  least one of those open balls must have infinite many points of
  $x_i$.

  Then, it can be selected a $N$ such that for every $m, n > N$ it
  occurs that $d(x_{i_m}, x_{i_n}) > 2\epsilon$ that
  $x_{i_m}, x_{i_n}$ are contained in a open ball $B(y_i,
    \epsilon)$. Hence:

  \begin{equation}
    \begin{split}
      d(x_{i_m}, d_{i_n}) &\le d(x_{i_m}, y_i) + d(y_i, x_{i,n}) \\
      &< \epsilon + \epsilon = 2\epsilon
    \end{split}
  \end{equation}

  Therefore, we got a set of two points that happens that
  $d(x_{i_m}, x_{i_n}) > 2\epsilon$ and
  $d(x_{i_m}, x_{i_n}) < 2\epsilon$. Contradiction.

  ($\leftarrow$) Suppose that in a metric space every sequence has a
  Cauchy sub-sequence and that $E$ is not totally bounded. In this
  manner, we can construct a finite set of points such that it occurs
  that $X = \{x_1, x_2, \dots, x_n\}$:

  \begin{equation}
    E \ne \bigcup_{i=1}^nB(x_i, \epsilon)
  \end{equation}

  Then, there must exist a point $y_0 \in X$ such that it occurs that
  $y_0 \notin \bigcup_{i=1}^nB(x_i, \epsilon)$. Then, we could keep
  constructing the distinct points $\{y_i\}_{i \in \N}$ in this way:

  \begin{equation}
    y_i \notin \bigcup_{j=1}^nB(x_j, \epsilon) \cup \bigcup_{j=0}^{i-1}B(y_j, \epsilon)
  \end{equation}

  As it is not totally bounded. Hence, the sequence
  $\{y_i\}_{i \in \N}$ must have a Cauchy sub-sequence. Hence, for
  some $k,m > N$ it must occur that:

  \begin{equation}
    d(y_k, y_m) < \epsilon
  \end{equation}

  Hence, it can be argued that $y_m \in B(y_k, \epsilon)$ and
  $y_k \in B(y_m, \epsilon)$. But, one of $y_m$ or $y_k$ by
  construction cannot be in one of those balls as the sequence is
  constructed by excluding the balls of the previous points in the
  sequence. Contradiction.
\end{proof}

\section{Homework 4}
\subsection{Problem 17}
\begin{question}
  Is the function $x^2$ uniformly continuous on $\R$? The function
  $\sqrt{|x|}?$
\end{question}

\begin{proof}
  \begin{enumerate}
  \item $x^2$ is not UC. Let's suppose that is, then let $\epsilon = 1$, $y = x_0 + \delta / 2$, then:

    \begin{equation*}
      |x_0^2 - (x_0 + \delta/2)^2| = |x_0\delta + \delta^2/4|
    \end{equation*}

    Which, we could then select $x$ arbitrarily large enough and that
    would not be less than $1$. Contradiction.

  \item $\sqrt{|x|}$ is UC. Let $\epsilon > 0$ and
    $\delta = \epsilon^2$ then if $|x-y| < \delta$ then:
    \begin{equation*}
      |\sqrt{|x|} - \sqrt{|y|}| = \sqrt{|x - y|} < \epsilon
    \end{equation*}

    Hence, it is UC.
  \end{enumerate}
\end{proof}

\subsection{Problem 27}
\begin{question}
  Proof that a polynomial with an odd degree, then $f(\R) = \R$.
\end{question}

\begin{proof}
  Let $P_n(x)$ with $n$ odd.

  \begin{equation*}
    P_n(x) = \sum_{i=0}^na_ix^i
  \end{equation*}

  with $a_n > 0$. Clearly:

  \begin{equation*}
    \lim_{x \ri \infty}P_n(x) = \infty \text{ and } \lim_{x \ri -\infty}P_n(x) = -\infty
  \end{equation*}

  Hence, it is always possible to find a value $[-M, M]$ such that
  $f(-M) < y$ and $f(M) > y$. Then, by the middle value theorem it
  exists a value such that $f(x) = y$. Then, for all $y \in \R$ it is
  defined. Hence, $f(\R) = \R$
\end{proof}

\subsection{Problem 32}
\begin{question}
  Show that the sequence of functions:
  \begin{equation*}
    \sqrt{x}, \sqrt{x + \sqrt{x}}, \sqrt{x + \sqrt{x + \sqrt{x}}},
    \dots
  \end{equation*}
  on $\{x \in \R: x \ge 0\}$ is convergent and find the limit
  function.
\end{question}
\begin{proof}

  It is clear that for $x=0$ the sequence is constant with it's limit
  being 0. Furthermore, this sequence of functions can be written by
  the following recursion formula:

  \begin{equation*}
    f_0(x) = \sqrt{x} \quad f_{n+1}(x) = \sqrt{x + f_n(x)}
  \end{equation*}

  Then, for a $x > 0$ let's show that the sequence is increasing and
  bounded.
  \begin{enumerate}
  \item Increasing:
    \begin{itemize}
    \item It is clear that $f_0(x) < f_1(x)$. This happens because:
      \begin{equation*}
        \begin{split}
          \sqrt{x} &< \sqrt{x + \sqrt{x}} \ri x < x + \sqrt(x) \\
          0 &< \sqrt{x}
        \end{split}
      \end{equation*}

      Which is true.
    \item Let's show that if $f_{n-1}(x) < f_n(x)$ then
      $f_n(x) < f_{n+1}(x)$. Let's proceed by direct proof:
      \begin{equation*}
        \begin{split}
          f_{n-1}(x) &< f_n(x) \ri x + f_{n-1}(x) < x + f_n(x) \\
          \sqrt{x + f_{n-1}(x)} &< \sqrt{x + f_n(x)} \ri f_n(x) <
          f_{n+1}(x)
        \end{split}
      \end{equation*}
    \end{itemize}

  \item Bounded by $x + 1$
    \begin{itemize}
    \item Let's show that $f_0(x) < x + 1$
      \begin{equation*}
        \begin{split}
          f_0(x) &< x + 1 \ri \sqrt(x) < x + 1 \ri x < (x+1)^2 \\
          x &< x^2+2x+1 \ri 0 < x^2 + x + 1
        \end{split}
      \end{equation*}
      Which is true, as every element is positive.
    \item Let's show that $f_n(x) < x + 1$ then $f_{n+1}(x) < x + 1$.
      \begin{equation*}
        \begin{split}
          &f_n(x) < x + 1 \ri x + f_n(x) < 2x + 1 \\
          \ri& \sqrt{x + f_n(x)} < \sqrt{2x+1} \ri f_{n+1}(x) <
          \sqrt{2x+1} \\
          & \text{ and now} \\
          &\sqrt{2x + 1} < x + 1 \ri 2x+1 < (x+1)^2 \\
          &2x+1 < x^2+2x+1 \ri 0 < x^2
        \end{split}
      \end{equation*}
      Which is true.
    \end{itemize}
  \end{enumerate}
  Now, as the sequence is increasing and bounded it is convergent and
  has a unique limit. Then, by fix point we obtain that:
  \begin{equation*}
    \begin{split}
      &f(x) = \sqrt{x + f(x)} \ri f^2(x) = x + f(x) \\
      &f^2(x) - f(x) - x = 0 \\
      &f(x) = \frac{1 + \sqrt{1+4x}}{2}
    \end{split}
  \end{equation*}
  As the $x$ value was selected randomly, then the function converges
  for all $x > 0$. Hence, the sequence of function converges to:
  \begin{equation*}
    \lim_{n \ri \infty} f_n(x) =
    \begin{cases}
      0, &x=0 \\
      \frac{1 + \sqrt{1+4x}}{2}, &x > 0
    \end{cases}
  \end{equation*}
\end{proof}

\subsection{Problem 34}
\begin{question}
  Is the sequence of functions $\{f_n\}_{n \in \N}$ on $[0, 1]$
  uniformly convergent if:
  \begin{equation*}
    \begin{split}
      f_n(x) &= \frac{x}{1 + nx^2} \\
      f_n(x) &= \frac{nx}{1 + nx^2} \\
      f_n(x) &= \frac{nx}{1 + n^2x^2}
    \end{split}
  \end{equation*}
\end{question}

\begin{proof}
  Let's proof each case individually.
  \begin{enumerate}
  \item In this case, it is clear that $\lim_{n \ri \infty}
    f_n(x)=0$. Let's show that $|f_n(x) - 0| < \frac{1}{2\sqrt{n}}$.

    \begin{equation*}
      \begin{split}
        |f_n(x) - 0| &< \frac{1}{2\sqrt{n}} \\
        \frac{x}{1+nx^2} &< \frac{1}{2\sqrt{n}} \\
        2x\sqrt{n} &< 1 + nx^2 \\
        0 &< nx^2 - 2x\sqrt{n} + 1 \\
        0 &< (x\sqrt{n} - 1)^2
      \end{split}
    \end{equation*}

    Which is true. Hence, that distance to the limit is bounded by
    the given equation, which then for every $\epsilon > 0$ we could
    choose an $n$ big enough that $\frac{1}{2\sqrt{n}} < \epsilon$.
    In conclusion, these sequence uniformly converges.

  \item In this manner, solving the limit we would obtain that
    $\lim_{n \ri \infty} f_n(x)=x^{-1}$ if $x = 0$; on the last case,
    it is clear that the sequence is always 0 which would not present
    any problem. Then, supposing $x \in (0,1]$ then, we can obtain
    that limit. Let's show that this sequence is not uniformly
    convergent. Hence, it is sufficient to proof that it exists
    $\epsilon >0$, $x \in (0,1]$ such that for every
    $N \in \N$ it exists a $n > N$ such that
    $|f_n(x) - x^{-1}| > \epsilon$.

    Now let's pick a $N \in \N$, $n = N + 1$,
    $x = \frac{1}{\sqrt{n}}$ and $\epsilon = 0.25$. Then:

    \begin{equation*}
      \begin{split}
        \left|\frac{nx}{1 + nx^2} - x^{-1}\right| &=
        \left|\frac{\sqrt{n}}{2} - \sqrt{n}\right| \\
        &= \frac{\sqrt{n}}{2} > \frac{1}{2} > 0.25 = \epsilon
      \end{split}
    \end{equation*}

    Then, it is not uniformly convergent.

  \item As done before, the limit is $\lim_{n \ri \infty}
    f_n(x)=0$. Let's show, as in the previous proof, that these
    function does not uniformly converge. Let $N \in \N$, $n = N + 1$,
    $x = \frac{1}{n}$ and $\epsilon = 0.25$. Then:

    \begin{equation*}
      \begin{split}
        |f_n(x) - 0| &= \left|\frac{nx}{1 + n^2x^2}\right| \\
        &= \frac{1}{2} > \epsilon.
      \end{split}
    \end{equation*}

    Then, it is not uniformly convergent.
  \end{enumerate}
\end{proof}
\subsection{Problem 36}
\begin{question}
  Does the sequence of functions $\{\frac{x}{n}\}_{n \in \N}$ converge
  uniformly in $\R$.
\end{question}
\begin{proof}
  It is clear that these sequence of functions converge to $0$. Let's
  prove it is not uniformly convergent. Then let $N \in \N$,
  $n = N + 1$, $x = n + 1$ and $\epsilon = 1$. Then:
  \begin{equation*}
    \begin{split}
      \left|\frac{x}{n} - 0\right| &= \frac{n + 1}{n} \\
       &> 1 > \epsilon.
    \end{split}
  \end{equation*}

  Hence, it is not uniformly continuous.
\end{proof}

\subsection{Problem 40}
\begin{question}
  Let $a, b \in \R$, $a < b$, and for $n = 1, 2, 3, \dots$ let
  $f_n:[a,b] \ri \R$ be an increasing function. Prove that if the
  sequence $f_1, f_2, \dots$ converges to $f$ then $f$ is increasing,
  and that if $f$ is continuous then the convergence is uniform.
\end{question}

\begin{proof}
  \begin{enumerate}
  \item Let's proof that if the sequence is increasing then the limit
    is increasing. Let $x,y \in [a, b]$ with $x \ge y$. There are
    $\epsilon > 0$ and $N, M$ such that:

    \begin{equation*}
      |f_n(x)-f(x)| < \epsilon \text{ and } |f_m(x) - f(x)| <
      \epsilon
    \end{equation*}

    whenever $n > N$ and $m > M$. If we choose $k > \min(N,M)$ then
    we have:

    \begin{equation*}
      \begin{split}
        f_k(x) - f(x) &<\epsilon \\
        f(y) - f_k(y) &< \epsilon
      \end{split}
    \end{equation*}

    thus, $0 \le f_k(x) - f_k(y) < f(x) - f(y) + 2\epsilon$. Since,
    we can make $\epsilon$ arbitrarily small, $f(x) \ge f(y)$. Hence,
    $f$ is increasing.
  \item Let's proof that if $f$ is continuous then the convergence is
    uniform. Let's proceed by contradiction. Then, it exists
    $\epsilon > 0$, such that for all $N \in \N$, it exists
    $x_N \in [a,b]$, with $n \ge N$ such that
    $|f_m(x_N) - f(x_N)| \ge \epsilon$.

    Note that the sequence $x_N$ must have a converging sub-sequence,
    which as $E$ is compact. Hence, as this converging sub-sequence
    must converge also using each element and applying $f$ to it as
    it is continuous. Hence, for some $x_i$ it must happen that:

    \begin{equation*}
      |f_n(x_i) - f(x_i)| < \epsilon
    \end{equation*}

    Contradiction.
  \end{enumerate}
\end{proof}

\end{document}
