\section{Theoretical Approach}\label{sec:theo}
\subsection{General ARMAX model}
This work uses an ARMAX model presented in the literature; hereby, let the general ARMAX model be presented:
\begin{equation}
    z_{t+1}=\sum_{i=0}^{h_1}a_iz_{t-i}+\sum_{i=0}^{h_2}b_iu_{t-i}+\sum_{i=0}^{h_3}c_i\xi_{t-i}
\end{equation}
for $t=0,1,2,\ldots$, and $u_k$ are external inputs and $\xi_k$ are random noises.

\subsection{R\"ossler System}
The R\"ossler system is a set of ordinary differential equations given by:
\begin{equation}
  \begin{cases}
    \dot{x}=-y-z&\\
    \dot{y}=x+ay&\\
    \dot{z}=b+z(x-c).&
  \end{cases}
\end{equation}
These equations originally proposed by O. R\"ossler in \cite{rossler1976equation}; it is well-known that, under certain parameters, the output of the state variable $z$ presents peaks in a aperiodic manner \cite{canals2014random}. This model is used to emulate the behavior of one of the inputs for the ARMAX model, since it presents quite an unusual behavior and this method gives a decent representation.
