\documentclass[11pt]{article}
\usepackage[spanish]{babel}
\usepackage{geometry}
\usepackage{amsmath}
\usepackage{amsthm}
\usepackage{graphicx}
\usepackage[utf8]{inputenc}

%%%%%%%% MARGIN
\geometry{verbose,letterpaper,tmargin=3cm,bmargin=3cm,lmargin=2.5cm,rmargin=2.5cm}

%%%%%%%% SUB-FIGURE PACKAGE
\usepackage{subcaption}

%%%%%%%% HYPERREF PACKAGE
\usepackage{hyperref}
\hypersetup{linkcolor=blue}
\hypersetup{citecolor=blue}
\hypersetup{urlcolor=blue}
\hypersetup{colorlinks=true}

%%%%%%%% DEFINITION AND THEOREM DEFINITIONS
\theoremstyle{definition}
\newtheorem{definition}{Definición}[section]

\theoremstyle{remark}
\newtheorem{remark}{Anotación}

\theoremstyle{remark}
\newtheorem{question}{Pregunta}

\newtheorem{theorem}{Teorema}[section]

%%%%%%%% MULTI-COLUMNS PACKAGE
\usepackage{multicol}

%%%%%%%% SETS DEFINITIONS
\usepackage{amssymb}
%%%% Important sets
\renewcommand{\O}{\mathbb{O}}
\newcommand{\N}{\mathbb{N}}
\newcommand{\Z}{{\mathbb{Z}}}
\newcommand{\Q}{{\mathbb{Q}}}
\newcommand{\R}{{\mathbb{R}}}

%%%% Statistics
\newcommand{\E}[1]{\mathbb{E}\left[#1 \right]}
\newcommand{\V}[1]{\mathrm{Var}\left[#1 \right]}

%%%% Lambda Calculus Symbols
\newcommand{\dneq}{\,\, \# \,\,}
\renewcommand{\S}{\pmb{\mathrm{S}}}
\newcommand{\I}{\pmb{\mathrm{I}}}
\newcommand{\K}{\pmb{\mathrm{K}}}
\newcommand{\ch}[1]{\ulcorner #1 \urcorner}

%%%% Make optional parameter
% https://tex.stackexchange.com/questions/217757/special-behavior-if-optional-argument-is-not-passed
\usepackage{xparse}
\NewDocumentCommand{\cx}{o}{
  \IfNoValueTF{#1}
  {\left[\quad\right]}
  {\left[\, #1 \,\right]}
}

%%%%%%%% LOGIC TREES
\usepackage{prftree}

%%%%%%%% SPLIT EQUATIONS
% https://tex.stackexchange.com/questions/51682/is-it-possible-to-pagebreak-aligned-equations
\allowdisplaybreaks

%%%%%%%% START DOCUMENT

\title{Tarea Estocásticos}
\author{Juan Sebasti\'an C\'ardenas-Rodríguez \\
  \scalebox{0.7}{Ingeniería Matemática, Universidad EAFIT} \\
  \scalebox{0.7}{201710008101}}
\date{\today}


\begin{document}
\maketitle

\begin{question}
  Usando las aproximaciones de la integral de ITO, probar que:
  \begin{equation*}
    \int_0^tsdB_s = tB_t - \int_0^tB_sds
  \end{equation*}
\end{question}
\begin{proof}
  Procedamos de esta esta forma:
  \begin{align*}
    \Delta (s_iB_i) &= s_iB_i - s_{i-1}B_{i-1} \\
                    &= s_iB_i - s_{i-1}B_{i-1} + s_iB_{i-1} - s_iB_{i-1} \\
                    &= s_i\Delta B_i + B_{i-1}\Delta s_i \\
                    &= (\Delta s_i + s_{i-1})\Delta B_i + B_{i-1} \Delta s_i \\
                    &= \Delta s_i \Delta B_i + s_{i-1}\Delta B_i + B_{i-1} \Delta s_i \\
    \lim_{n \rightarrow \infty}\sum_{i =1}^{n} \Delta (s_i B_i) &=
    \lim_{n \rightarrow \infty}\sum_{i =1}^{n} \Delta s_i \Delta B_i +
   \lim_{n \rightarrow \infty}\sum_{i =1}^{n} s_{i-1} \Delta B_i +
   \lim_{n \rightarrow \infty}\sum_{i =1}^{n} B_{i-1} \Delta s_i \\
    \int_0^td(sB_s) &= \int_0^t ds dB_s + \int_0^t s dB_s + \int_0^tB_s ds \\
    \int_0^t s dB_s &= tB_t - \int_0^t B_s ds
  \end{align*}
\end{proof}

\begin{question}
    Usando las aproximaciones de la integral de ITO, probar que:
  \begin{equation*}
    \int_0^tB_s^2dB_s = \frac{1}{3}B_t^3 - \int_0^tB_sds
  \end{equation*}
\end{question}
\begin{proof}
  Obsérvese que:
  \begin{align*}
    (\Delta B_i)^3 &= B_i^3 - 3B_i^2B_{i-1} + 3B_iB_{i - 1}^2 - B_{i-1}^3 \\
    (\Delta B_i)^3 &= \Delta B_i^3 - 3B_iB_{i - 1}\Delta B_i \\
    (\Delta B_i)^3 &= \Delta B_i^3 - 3(\Delta B_i + B_{i - 1})B_{i - 1} \Delta B_i \\
    3B_{i - 1}^2\Delta B_i &= \Delta B_i^3 - 3B_{i - 1}(\Delta B_i)^2 - (\Delta B_i)^3 \\
    3\lim_{n \rightarrow \infty}\sum_{i =1}^{n} B_{i - 1}^2\Delta B_i &=
    \lim_{n \rightarrow \infty}\sum_{i =1}^{n}\Delta B_i^3 -
   3\lim_{n \rightarrow \infty}\sum_{i =1}^{n}B_{i - 1}(\Delta B_i)^2
    - \lim_{n \rightarrow \infty}\sum_{i =1}^{n} (\Delta B_i)^3 \\
    3 \int_0^tB_s^2dB_s &= B_t^3 - 3\int_0^tB_s (dB_s)^2 - \int_0^t(dB_s)^3 \\
    \int_0^tB_s^2dB_s &= \frac{1}{3}B_t^3 - \int_0^tB_s ds - \int_0^t(dB_s)^2 dB_s \\
    &\text{Dado que } dB_s ds = 0 \text{ obtenemos entonces que } \\
        \int_0^tB_s^2dB_s &= \frac{1}{3}B_t^3 - \int_0^tB_s ds
  \end{align*}
\end{proof}

\begin{question}
    Usando las aproximaciones de la integral de ITO, probar que:
  \begin{equation*}
    \int_0^tB_s^3dB_s = \frac{1}{4}B_t^4 - \int_0^t\frac{3}{2}B_s^2ds
  \end{equation*}
\end{question}
\begin{proof}
  Obsérvese que:
  \begin{align*}
    (\Delta B_i)^4 &= B_i^4 - 4B_i^3B_{i - 1} + 6B_i^2B_{i -1}^2 - 4B_iB_{i-1}^3 +
                      B_{i-1}^4 \\
    (\Delta B_i)^4 &= B_i^4 - B_{i -1}^4 - 4B_i^3B_{i - 1} + 6B_i^2B_{i -1}^2 - 4B_iB_{i-1}^3 +
                      2B_{i-1}^4 \\
    (\Delta B_i)^4 &= \Delta B_i^4 \pmb{+ 2B_iB_{i - 1}^2(3B_i - 2B_{i-1}) -
                      2B_{i-1}(2B_i^3 - B_{i-1}^3)} \\
    (\Delta B_i)^4 &= \Delta B_i^4 + \Gamma_i
  \end{align*}
  Si nos enfocamos en el término en negrilla ($\Gamma_i$) obtenemos:
  \begin{align*}
    \Gamma_i
    &= 2B_iB_{i - 1}^2(3B_i - 2B_{i-1}) -2B_{i-1}(2B_i^3 - B_{i-1}^3) \\
    &= 2B_iB_{i-1}^2(3\Delta B_i + B_{i-1}) -2B_{i-1}(2B_i^3 - B_{i-1}^3) \\
    &= 2(\Delta B_i + B_{i-1})B_{i-1}^2(3\Delta B_i + B_{i-1}) -2B_{i-1}(2B_i^3 - B_{i-1}^3) \\
    &= 6B_{i-1}^2(\Delta B_i)^2 + 2B_{i-1}^3\Delta B_i + 6B_{i-1}^3\Delta B_i
      + 2B_{i-1}^4 -2B_{i-1}(2B_i^3 - B_{i-1}^3) \\
    &= 6B_{i-1}^2(\Delta B_i)^2 + 8B_{i-1}^3\Delta B_i
      \pmb{-4B_{i-1}(B_i^3 - B_{i-1}^3)} \\
    &= 6B_{i-1}^2(\Delta B_i)^2 + 8B_{i-1}^3\Delta B_i + \Sigma_i
  \end{align*}
  Ahora, usando $\Sigma_i$:
  \begin{align*}
    \Sigma_i
    &= -4B_{i-1}(B_i^3 - B_{i-1}^3) \\
    &= -4B_{i-1}\Delta B_i(B_i^2 + B_iB_{i-1} + B_{i-1}^2) \\
    &= -4B_{i-1}\Delta B_i((\Delta B_i)^2 + 3B_iB_{i-1}) \\
    &= -4B_{i-1}(\Delta B_i)^3 - 12B_iB_{i-1}^2\Delta B_i \\
    &= -4B_{i-1}(\Delta B_i)^3 - 12(\Delta B_i + B_{i-1})B_{i-1}^2\Delta B_i \\
    &= -4B_{i-1}(\Delta B_i)^3 - 12B_{i-1}^2(\Delta B_i)^2 - 12B_{i-1}^3\Delta B_i
  \end{align*}
  Luego, aplicando sumatoria y límite a ambos lados de $\Sigma_i$:
  \begin{align*}
    \lim_{n \rightarrow \infty}\sum_{i =1}^{n}\Sigma_i
    &= \lim_{n \rightarrow \infty}\sum_{i =1}^{n} -4B_{i-1}(\Delta B_i)^3 -
      \lim_{n \rightarrow \infty}\sum_{i =1}^{n} 12B_{i-1}^2(\Delta B_i)^3 -
      \lim_{n \rightarrow \infty}\sum_{i =1}^{n} 12B_{i-1}^3\Delta B_i \\
    &= -\int_0^t 4B_s (dB_s)^3 - 12 \int_0^t B_s^2 (dB_s)^2 - 12 \int_0^t B_s^3 dB_s \\
    &= -12\int_0^t B_s^2 ds -12 \int_0^t B_s^3 dB_s
  \end{align*}
    Luego, aplicando sumatoria y límite a ambos lados de $\Gamma_i$:
  \begin{align*}
    \lim_{n \rightarrow \infty}\sum_{i =1}^{n} \Gamma_i
    &= \lim_{n \rightarrow \infty}\sum_{i =1}^{n} 6B_{i-1}^2(\Delta B_i)^2 +
      \lim_{n \rightarrow \infty}\sum_{i =1}^{n} 8B_{i-1}^3\Delta B_i +
      \lim_{n \rightarrow \infty}\sum_{i =1}^{n} \Sigma_i \\
    &= 6\int_0^t B_s^2 (dB_s)^2 + \int_0^t 8B_s^3 dB_s - 12\int_0^t B_s^2 ds
      - 12\int_0^t B_s^3 dB_s \\
    &= -6\int_0^t B_s^2 ds - 4 \int_0^t B_s^3 dB_s
  \end{align*}
  Por último, en la ecuación original:
  \begin{align*}
    \lim_{n \rightarrow \infty}\sum_{i =1}^{n} (\Delta B_i)^4
    &= \lim_{n \rightarrow \infty}\sum_{i =1}^{n} \Delta B_i^4 +
      \lim_{n \rightarrow \infty}\sum_{i =1}^{n} \Gamma_i \\
    \int_0^t (dB_s)^4
    &= \int_0^td(B_s^4) - 6\int_0^t B_s^2 ds - 4 \int_0^t B_s^3 dB_s \\
    0 &= B_t^4 - 6\int_0^t B_s^2 ds - 4\int_0^t B_s^3 dB_s \\
    \int_0^t B_s^3 dB_s &= \frac{1}{4} B_t^4 - \int_0^t \frac{3}{2}B_s^2 ds
  \end{align*}
\end{proof}

\end{document}
