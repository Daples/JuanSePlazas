\documentclass{beamer}
\usefonttheme{professionalfonts}
\usepackage[english]{babel}
\usepackage{geometry}
\usepackage{amsmath}
\usepackage{amsthm}
\usepackage{graphicx}
\usepackage{caption}
\usepackage[utf8]{inputenc}

%%%%%%%% HYPERREF PACKAGE
\hypersetup{colorlinks=true}
\hypersetup{citecolor=orange}
\hypersetup{urlcolor=orange}

%%%%%%%% DEFINITION AND THEOREM DEFINITIONS
\let\definition\relax
\theoremstyle{definition}
\newtheorem{definition}{Definition}[section]

\let\remark\relax
\theoremstyle{remark}
\newtheorem{remark}{Remark}

\theoremstyle{example}
\newtheorem{question}{Question}

%%%%%%%% MULTI-COLUMNS PACKAGE
\usepackage{multicol}

%%%%%%%% THEME SLIDES
\usetheme{default}

%%%%%%%% BEAMER STUFF
\setbeamertemplate{footline}[frame number]
\setbeamertemplate{section in toc}[sections numbered]
\setbeamertemplate{subsection in toc}[subsections numbered]
\setbeamertemplate{navigation symbols}{}

% Not enumerate frame breaks
% https://tex.stackexchange.com/questions/295854/how-to-edit-behaviour-of-frame-titles-during-frame-break-in-beamer
\setbeamertemplate{frametitle continuation}[from second][]

%%%%%%%% FRAME TITLES AND SUBTITLES
% https://stackoverflow.com/questions/35810906/vim-beamer-dynamic-frame-title1
\newif\ifinsection
\newif\ifinsubsection

\let\oldsection\section
\renewcommand{\section}{
  \global\insectiontrue
  \global\insubsectionfalse
  \oldsection}
\let\oldsubsection\subsection
\renewcommand{\subsection}{
  \global\insubsectiontrue
  \oldsubsection}

\newcommand {\aframe}[1] {
  \begin{frame}
    \ifinsection\frametitle{\secname}\fi
    \ifinsubsection\framesubtitle{\subsecname}\fi
  #1
  \end{frame}
}

% Blue line after title
% https://tex.stackexchange.com/questions/343517/beamer-full-width-hrule-below-frame-title
\setbeamertemplate{frametitle}{%
    \usebeamerfont{frametitle}\insertframetitle\strut%
    \vskip.0\baselineskip%
    \leaders\vrule width 0.85\paperwidth\vskip0.4pt%
    \vskip2pt%
    \usebeamerfont{framesubtitle}\insertframesubtitle
}

%%%%%%%% FOOTNOTE STUFF
\renewcommand{\thefootnote}{\fnsymbol{footnote}}
% Footnote without symbol
% https://tex.stackexchange.com/questions/30720/footnote-without-a-marker
\newcommand\blfootnote[1]{%
  \begingroup
  \renewcommand\thefootnote{}\footnote{#1}%
  \addtocounter{footnote}{-1}%
  \endgroup
}

%%%%%%%% CENTER OF SLIDE THANK YOU
\usepackage{tikz}

%%%%%%%% SUB-FIGURE PACKAGE
\usepackage{subcaption}

%%%%%%%% PERSONAL COMMANDS
\usepackage{amssymb}

%%%% Important sets
\renewcommand{\O}{\mathbb{O}}
\newcommand{\N}{\mathbb{N}}
\newcommand{\Z}{{\mathbb{Z}}}
\newcommand{\Q}{{\mathbb{Q}}}
\newcommand{\R}{{\mathbb{R}}}

%%%% Usual operations
\newcommand{\pow}[2]{#1^{#2}}
\newcommand{\expp}[1]{e^{#1}}
\newcommand{\fst}{\mathrm{fst}}
\newcommand{\snd}{\mathrm{snd}}

%%%% Lambda Calculus Symbols
\newcommand{\dneq}{\,\, \# \,\,}
\renewcommand{\S}{\pmb{\mathrm{S}}}
\newcommand{\I}{\pmb{\mathrm{I}}}
\newcommand{\K}{\pmb{\mathrm{K}}}
\newcommand{\ch}[1]{\ulcorner #1 \urcorner}

%%%% Ordinal Lambda Calculus Symbols
\newcommand{\ordAlph}{\Sigma_{\text{Ord}}}
\newcommand{\termOrd}{\text{Term}_\text{Ord}}
\newcommand{\fl}{\mathrm{fl}}
\newcommand{\sk}{\mathrm{sk}}

%%%% Superscript to the left
% https://latex.org/forum/viewtopic.php?t=455
\usepackage{tensor}
\newcommand{\app}[3]{\tensor*[^{#1}]{\left(#2, #3\right)}{}}

%%%% Make optional parameter
% https://tex.stackexchange.com/questions/217757/special-behavior-if-optional-argument-is-not-passed
\usepackage{xparse}

%%%% Statistics
\NewDocumentCommand{\E}{o m}{
  \IfNoValueTF{#1}
  {\mathbb{E}\left[#2\right]}
  {\mathbb{E}^{#1}\left[ #2\right]}
}
\NewDocumentCommand{\V}{o m}{
  \IfNoValueTF{#1}
  {\mathrm{Var}\left[#2\right]}
  {\mathrm{Var}^{#1}\left[ #2\right]}
}
\RenewDocumentCommand{\P}{o o m}{
  \IfNoValueTF{#1}
  {\IfNoValueTF{#2}
    {\mathrm{P}\left(#3\right)}
    {\mathrm{P}^{#2}\left(#3\right)}}
  {\IfNoValueTF{#2}
    {\mathrm{P}_{#1}\left(#3\right)}
    {\mathrm{P}_{#1}^{#2} \left(#3\right)}}
}

%%%% Lambda Calculus
\NewDocumentCommand{\cx}{o}{
  \IfNoValueTF{#1}
  {\left[\quad\right]}
  {\left[\, #1 \,\right]}
}

%%%% Create absolute value function
% https://tex.stackexchange.com/questions/43008/absolute-value-symbols
\usepackage{mathtools}
\DeclarePairedDelimiter\abs{\lvert}{\rvert}%
\DeclarePairedDelimiter\norm{\lVert}{\rVert}%
\makeatletter
\let\oldabs\abs
\def\abs{\@ifstar{\oldabs}{\oldabs*}}
%
\let\oldnorm\norm
\def\norm{\@ifstar{\oldnorm}{\oldnorm*}}
\makeatother

%%%%%%%% LOGIC TREES
\usepackage{prftree}

%%%%%%%% SPLIT EQUATIONS
% https://tex.stackexchange.com/questions/51682/is-it-possible-to-pagebreak-aligned-equations
\allowdisplaybreaks

% Better spacing between display equation
% https://bit.ly/2zLJEXN
\makeatletter
\g@addto@macro\normalsize{%
    \setlength\belowdisplayskip{8pt}
  }

%%%%%%%% TO USE SHORT COMMANDS FOR VECTOR LINES
\usepackage{esvect}

%%%%%%%% START DOCUMENT
\title{Multiplication of Church Numerals}

\author{Juan Sebasti\'an C\'ardenas-Rodríguez \\
  \scalebox{0.7}{Mathematical Engineering, Universidad EAFIT}}

\begin{document}

% Plain so is not numbered
\begin{frame}[plain]
  \titlepage
\end{frame}

%%%%%%%%%% SLIDES %%%%%%%%%%%%%%%
\section{Introduction}
\subsection{Preliminaries}
\aframe{ Composition between two lambda terms:
  \begin{equation*}
    M \circ N \equiv \lambda y. M(Ny)
  \end{equation*} \pause

  Exponentiation:
  \begin{equation}
    \label{eq:exp}
    F^{0}M \equiv M \quad F^{n+1}M \equiv F(F^{n}M)
  \end{equation} \pause

  Church Numerals:
  \begin{equation}
    \label{eq:cn}
    c_{n} \equiv \lambda fx. f^{n}x
  \end{equation}

  $M$ is $\alpha$-congruent to $N$ if $N$ results from M by a series of changes
  of bound variables. Noted by:
  \begin{equation*}
    M \equiv_{\alpha} N
  \end{equation*}
}

\aframe{ Let's prove that: Using Equations \ref{eq:exp}, \ref{eq:cn}:
  \begin{equation*}
    c_{0} = \lambda fx. x \quad c_{n+1} = \lambda fx. f(c_{n}fx)
  \end{equation*}

  \begin{proof}
    The initial proof is straight forward:
    \begin{align*}
      c_{0} &\equiv \lambda fx. f^{0}x \\
            &\equiv \lambda fx. x \\
            &= \lambda fx. x
    \end{align*} \pause
    It is clearly seen that:
    \begin{align*}
      \action<+->{c_{n}fx = f^{n}x} \quad
      \action<+->{c_{n+1} &\equiv \lambda fx. f^{n+1}x \\
                          &\equiv \lambda fx. f(f^{n}x) \\
                          &= \lambda fx. f(c_{n}fx)}
    \end{align*}
  \end{proof}
}

\section{Multiplication}
\aframe{The lambda term proposed is:
  \begin{equation*}
    A_{\times} xy \equiv x \circ y
  \end{equation*} \pause

  And to prove $\lambda$-definability, let's show that:
  \begin{equation*}
    A_{\times} c_{n} c_{m} = c_{n \times m}
  \end{equation*} \pause

  Then, let's prove it inductively using the inductive definition proposed.
  \begin{equation*}
    c_{0} = \lambda fx. x \quad c_{n+1} = \lambda fx. f(c_{n}fx)
  \end{equation*}

  The induction will be performed in the parameter $c_{n}$.
}

\subsection{Base Case}
\aframe{ Let's proof that it satisfies for the base case, $\forall m \in \N$:
  \begin{align*}
    \action<+->{A_{\times} c_{0} c_{m} &\equiv c_{0} \circ c_{m} \\}
    \action<+->{&\equiv \lambda y. c_{0}(c_{m}y) \\}
    \action<+->{&= \lambda y. (\lambda fx. x)(c_{m}y) \\}
    \action<+->{&= \lambda y. (\lambda x. x) \\}
    \action<+->{&\equiv \lambda yx. x \\}
    \action<+->{&\equiv_{\alpha} \lambda fx. x \\
                                       &\equiv c_{0} \\
                                       &\equiv c_{0 \times m}}
  \end{align*}
  \action<+->{Hence, it is concluded that:
    \begin{equation*}
      A_{\times} c_{0} c_{m} = c_{0 \times m}
    \end{equation*}}
}

\subsection{Induction Step}
\aframe{\onslide<1->{Suppose that for some $n \ge 0$ it happens that
    $\forall m \in \N$:
    \begin{equation*}
      A_{\times} c_{n} c_{m} \equiv \lambda y. c_{n}(c_{m}y) = c_{n \times m}
    \end{equation*}
    Then, let's prove that it happens for $c_{n+1}$.}
  \begin{align*}
    \onslide<2->{A_{\times} c_{n+1} c_{m} &\equiv \lambda y. c_{n+1}(c_{m}y) \\}
    \onslide<3->{&\equiv \lambda y. [\lambda fx. f(c_{n}fx)] (c_{m}y) \\}
    \onslide<4->{&= \lambda y. (\lambda x. (c_{m}y)(c_{n}(c_{m}y)x) \\}
    \onslide<5->{&\equiv \lambda yx. (c_{m}y)(c_{n}(c_{m}y)x) \\}
    \onslide<6->{&= \lambda yx. (c_{m}y)(c_{n \times m}yx)}
  \end{align*}
  \onslide<7->{Remember the $\lambda$-term for the sum:
    \begin{align*}
      A_{+} c_{m} c_{n \times m} \equiv \lambda pq. c_{m}p(c_{n \times m} pq)
    \end{align*}}
}

\aframe{With this equation:
  \begin{align*}
    A_{+} c_{m} c_{n \times m} \equiv \lambda pq. c_{m}p(c_{n \times m} pq)
  \end{align*}
  Hence, continuing it is seen that:
  \begin{align*}
    \onslide<1->{&= \lambda yx. (c_{m}y) (c_{n \times m}yx) \\
                 &\equiv \lambda yx. c_{m}y (c_{n \times m} yx) \\}
    \onslide<2->{&\equiv_{\alpha} \lambda pq. c_{m}p(c_{n \times m} pq) \\
                 &\equiv  A_{+} c_{m} c_{n \times m} \\}
    \onslide<3->{&= c_{m + n \times m} \\
                 &= c_{n \times m + m} \\
                 &= c_{(n + 1) \times m}}
  \end{align*}
}

%%%%%%%%%%%%%%%%%%%%%%%%%%%%%%%%%
\begin{frame}
  \begin{tikzpicture}[overlay, remember picture]
    \node[anchor=center] at (current page.center) { \Huge{\emph{Thank you!}}};
  \end{tikzpicture}
  \begin{minipage}[t][.8\textheight]{\textwidth}
    \vfill
    \begin{center}
      \scalebox{0.7}{Juan Sebasti\'an C\'ardenas-Rodríguez} \\
      \scalebox{0.7}{jscardenar@eafit.edu.co} \\
    \end{center}
  \end{minipage}
\end{frame}

\end{document}
