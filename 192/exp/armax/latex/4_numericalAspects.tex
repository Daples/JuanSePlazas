\section{Numerical Aspects}\label{sec:numAsp}
\subsection{ARMAX Model}
In this particular case, the model obtained in \cite{li2014armax} is:
\begin{equation}
\begin{aligned} z_{t}=& 237.565+0.426z_{t-1}+\xi_{t}-0.153 \xi_{t-1}+8.9087u_{1, t} \\ &-1.557 u_{7, t}+31.919 u_{8, t}-2.045u_{9, t}
\end{aligned}
\end{equation}
where $z_t$ is the power output of the PV grid in Watts (W); $u_{1,t}$ is the daily average temperature, $u_{7,t}$ is the precipitation amount, $u_{8,t}$ is the insolation duration and $u_{9,t}$ is the humidity.

\subsection{Input values}
For the inputs, it was used several historical data to simulate their values. In this manner, $u_{1,t}$, $u_{8,t}$ and $u_{9,t}$ was simulated using normal distributions with values for $(\mu, \sigma)$ of $(28.81, 1.10)$, $(8.95, 3.40)$ and $(73.14, 6.99)$ respectively. On the other hand, $u_{7,t}$ was simulated using three times the $z$ value for a simulation of a Rössler circuit. It was simulated using the Runge-Kutta's method with a step of~$0.01$ to a time of $182$ with parameters $(a, b, c) = (0.29, 0.14, 4.52)$ and initial conditions of $(x_0, y_0, z_0) = (0.72, 1.28, 0.21)$, the output was sampled every $1s$.

\subsection{Noise values}
The noise was simulated using different distributions:
\begin{enumerate}
  \item A Normal distribution, with $\mu = 0$ and $\sigma = 1$
  \item A Student t-distribution, with $\nu = 0.7$
  \item A Cauchy distribution, with $x_0 = 0$ and $\gamma = 2$
  \end{enumerate}

This distributions were used to contrast the normal distribution, as they generate outliers.
