\section{Theoretical Approach}\label{sec:theo}
\subsection{General ARMAX model}
This work uses an ARMAX model presented in the literature; hereby, let the general ARMAX model be presented:
\begin{equation}
    z_{t+1}=\sum_{i=0}^{h_1}a_iz_{t-i}+\sum_{i=1}^{m}\sum_{j=0}^{h_2}b_{ij}u_{i,t-j}+\sum_{i=0}^{h_3}c_i\xi_{t-i}
\end{equation}
for $t=0,1,2,\ldots$; $u_{l,k}$ are external inputs (type $l$ and lag $k$), and $\xi_k$ are random noises of lag $k$.

\subsection{R\"ossler System}
The R\"ossler system is a set of ordinary differential equations:
\begin{equation}
  \begin{cases}
    \dot{x}=-y-z&\\
    \dot{y}=x+ay&\\
    \dot{z}=b+z(x-c).&
  \end{cases}
\end{equation}
These equations originally proposed by O. R\"ossler in \cite{rossler1976equation}; it is well-known that, under certain parameters, the output of the state variable $z$ presents peaks in a aperiodic (stochastic) manner \cite{canals2014random}. This model is used to emulate the behavior of one of the inputs for the ARMAX model, since it presents quite an unusual behavior and this method gives a decent representation.

\subsection{Normal Distribution}
Let $\xi$ be a random variable. We say $\xi\sim N(\mu,\sigma)$ if the probability density function (PDF) is given by
\begin{equation}
  f(x)=\dfrac{1}{\sqrt{2\pi}\sigma}e^{-\frac{1}{2}\left(\frac{x-\mu}{\sigma}\right)^2}
\end{equation}
with mean $\mu$ and standard deviation $\sigma$.

\subsection{Cauchy Distribution}
Let $\xi$ be a random variable. We say that $\xi$ has a Cauchy distribution if the PDF is given by
\begin{equation}
  f(x)=\dfrac{1}{\pi \gamma}\left[\dfrac{\gamma^{2}}{\left(x-x_{0}\right)^{2}+\gamma^{2}}\right]
\end{equation}
with location parameter $x_0$ and scale $\gamma$.

\subsection{Student's t-Distribution}
Let $\xi$ be a random variable. We say $\xi\sim t_\nu$ if the PDF is given by
\begin{equation}
  f(x)=\frac{\Gamma\left(\frac{\nu+1}{2}\right)}{\sqrt{\nu \pi} \Gamma\left(\frac{\nu}{2}\right)}\left(1+\frac{x^{2}}{\nu}\right)^{-\frac{\nu+1}{2}}
\end{equation}
with $\nu$ degrees of freedom.
