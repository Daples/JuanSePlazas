\section{CONCLUSIONS}
It was successfully analyzed the received data, using statistical tests which allowed to fit data to use in the simulation, find auto-correlation and test homogeneity between data. It is important to remark that in the fitting of the data the distributions have a left bias as most of the times are positive and close to zero.

It was successfully implemented the simulation model in Python, which partially represents the Neuromédica pharmacy medicine retriveal system. The model did not fully validate with the tests done, which presents an important limitation of the implemented model; even though, the results given are not useful for the real system, this article proposes a methodology for solving this types of problems. 

For further work, a model using continuous simulation could be used as the inter-arrival times are so close to zero that the system could be analyzed supposing a continuous flux of patients. Also, a graphical user interface can be done so it is easily explained without the use of code.