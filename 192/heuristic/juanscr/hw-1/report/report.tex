\documentclass[10pt,twoside]{article}
\usepackage[utf8]{inputenc}
\usepackage{natbib}
\usepackage{graphicx}
\usepackage{booktabs}
\usepackage{adjustbox}
\usepackage[table,xcdraw]{xcolor}
\usepackage{amsmath}
\usepackage{mathtools}
\usepackage[]{todonotes}
\usepackage{fancyhdr}
\usepackage{multirow}

% Idioma
\usepackage[english]{babel}

\usepackage[none]{hyphenat}

\usepackage{algorithm}
\usepackage{algpseudocode}

\newcommand{\ignore}[1]{}


\title{Clark Wright Algorithm Modification for Heterogeneous Cumulatitive Capacitated Vehicle Routing Problem}
\author{\emph{Juan Sebastián Cárdenas}\\
\vspace{0.3cm}
\small{\tt{jscardenar@eafit.edu.co}}\\
Mathematical Science Department\\
School of Science\\
Universidad EAFIT\\
Medellín -- Colombia}
\date{\today}

\usepackage{anysize}
\marginsize{3cm}{2cm}{2cm}{3cm}

% Configurar encabezado y pie de página
\pagestyle{fancy}
\lfoot[\date{\today}]{\date{\today}}
\rfoot[\thepage]{\thepage}
\cfoot[]{}
\rhead[Cárdenas (2019)]{Heterogeneous Clark Wright Algorithm}
\lhead[]{}

\fancypagestyle{firststyle}
{
   \fancyhf{}
   \fancyfoot[R]{Last update: August of 2019}
}

\sloppy



\begin{document}

\maketitle

\thispagestyle{firststyle}

\section{Solution Algorithm}\label{sec_alg}
The \cite{clarke1964} algorithm is a classical method to solve the
Homogeneous Vehicle Routing Problem (VRP), considering an infinite
number of vehicles. This procedure is one of the
better constructive algorithms for this problem, as it has a
polynomial complexity of $O(n^2)$, it solves the problem
effectively and it is extremely flexible. In this manner, adaptations
of this algorithm for different variants of the VRP have been done by
\cite{golden1984}, \cite{caccetta2013} and more. A pseudo-code for this algorithm can be found in Algorithm \ref{a_cw}.

\begin{algorithm}[H]
\caption{Clark And Wright Algorithm}\label{a_cw}
\begin{algorithmic}[1]
  \Procedure{CW}{$b$, nodes, dem, cap}
  \State Initialize $n \gets $ \text{length(nodes)}; \text{routes} $\gets$ [ ]
  \For{$i \gets 0, n$}
    \State Create route with only one node $i$. Add that route to routes.
    \For{$j \gets i + 1, n$} \Comment{Calculate savings}
       \State $s_{ij} \gets d_{bi} + d_{bj} - d_{ij}$ \Comment{$d_{ij}$ is the distance between $i$ and $j$}
    \EndFor
   \EndFor
   \State \text{Sort savings in a decreasing manner.}
   \ForAll{$s_{ij}$}\Comment{Iterate over ordered savings.}
      \State $r1 \gets$ Route that has node $i$.
      \State $r2 \gets$ Routes that has node $j$.
      \State $ed1 \gets$ True if $i$ is first or last element of r1.
      \State $ed2 \gets$ True if $j$ is first or last element of r2.
      \State $d1 \gets$ demand of $(r1) + $demand of$(r2) < $ cap
      \If{$ed1$ and $ed2$ and $d1$}
        \State Join $r1$ and $r2$ by joining nodes $i$ and $j$ and add the new route to list routes.
        \State Remove $r1$ and $r2$ of list routes.
      \EndIf
   \EndFor
   \State \textbf{return} routes
\EndProcedure
\end{algorithmic}
\end{algorithm}

What the Clark and Wright algorithm do is, initially, calculate the
savings for each pair of nodes and generate a route for each node;
this route is the simplest one, as it goes from the base to the node
and back to the base. Then, it iterates over the sorted savings and
joins routes depdending if the nodes associated to that saving are at
the start or end of their respective routes. At the same time, it is
neccessary that the demands of both routes that are joining are less
than the capacity of the bus.

The variation proposed in this work is pretty straight foward. For
each type of bus, run the savings algorithm supposing an infinite
amount of vehicles. Then, select the routes that have more nodes
satisfying the quantity of each type of bus. Furthermore, remove nodes
that are in the chosen routes. If at the end there are nodes not in a
route, they are added to a feasible route. A pseudo-code for this
algorithm can be found in Algorithm \ref{a_mcw}.

\begin{algorithm}[H]
\caption{Modificated Clark Wright}\label{a_mcw}
\begin{algorithmic}[1]
  \Procedure{MCW}{$b$, nodes, dem, buses}
  \State Initialize $n \gets $ \text{length(nodes)}; routes $\gets$ [ ]
  \State Sort the list buses by capacity in an increasing manner.
  \ForAll{buses}
    \State $q \gets$ quantity of the type of bus.
    \State routes1 $\gets$ CW($b$, nodes, dem, capacity of bus)
    \State Sort routes1 by number of nodes in a decreasing manner.
    \State Save in routes the first $q$ elements of routes1.
  \EndFor
  \ForAll{not used nodes}
    \ForAll{routes}
      \If{dem(node) + demand of route $<$ capacity of bus}
        \State Add not used node to start or end of route, depending on which is closer.
        \State Continue to next node.
      \EndIf
    \EndFor
  \EndFor
   \State \textbf{return} routes
\EndProcedure
\end{algorithmic}
\end{algorithm}

This modification does not alter greatly the complexity of the
algorithm, as it has a complexity of $O(n^2m)$. At the same time, it
is important to remark that the buses have to be picked in increasing
manner. This is due to attempt to use the buses with lower capacity
the best way possible as they are the quickest ones. Also, so when the
largest buses are used it is more likely to find a
solution. Furthermore, an improvement to this algorithm could be done
by trying to evaluate the best feasible route for the not used nodes.

\subsection{Lower Bound}\label{sec_lb}
For the lower bound of this problem, it was chosen a similar problem
but changing the distances and bus constraints so that it represents a
problem with an exact solution and with an objective function much
smaller than the one needed.

In this manner, it supposed that all the nodes are distributed equally
on a circunference that has a perimeter equal to the sum of the lowest
$n$ distances of the original problem, concentric to the base. At the
same time, it is supposed that all the nodes have the same demand but
preserving the total demand $td$. Finally, it is supposed that the
problem can only use the type of bus with the smallest capacity
$c_{min}$ and has an amount of $\left \lceil{td/c_{min}} \right \rceil$.

This problem is good for finding a lower bound for several reasons. In
first place, as the demand is equally distributed there is not a real
distinction between any pair of nodes, so constructing the routes is
easy. In second place, the optimal value is easily found as the
geometric shape of the graph allows to find routes that pass through
all nodes. Lastly, as the total distance is the sum of the lowest
distance it assures that the nodes are closer than in the original
problem.

On the other hand, although it seems that the objective function of
the optimal solution for the problem proposed is always lower than the
one for the original problem a formal proof is not known. This is
important for further work, as it would allow to be certain that this
is actually a lower bound. At the same time, with the data set of
problems provided a contradiction was not found therefore it is
purposed for further work.

The objective function for this lower bound can be calculated using the pseudo-code found in Algorithm \ref{a_lb}.

\begin{algorithm}[H]
  \caption{Calculate lower bound} \label{a_lb}
  \begin{algorithmic}[1]
    \Procedure{LB}{nodes, buses, dem}
      \State Initialize $lb \gets$ 0; routes $\gets$ [ ]
      \State bus $\gets$ Select the bus with the lowest capacity.
      \State $p \gets$ Sum the $n$ lowest distances.
      \State $td \gets$ Sum of all the demands in dem.
      \State $n \gets$ length(nodes)
      \State $d \gets \frac{p}{2\pi}\sqrt{2(1 - \cos{\left(\frac{2\pi}{n}\right)})}$
      \State $de \gets \left \lceil{\frac{td}{n}} \right \rceil$
      \ForAll{nodes}
        \If{de + demand of route $\le$ capacity of bus}
          \State Append to route the node.
          \Else
          \State Start new route.
        \EndIf
      \EndFor
      \ForAll{routes}
        \For{node in route}
          \State $lb = lb + de \cdot d \cdot \text{Velocity of bus} \cdot \text{Number of nodes ahead of route}$
        \EndFor
        \State $lb = lb + de \cdot d \cdot \text{Velocity of bus} \cdot \text{Number of nodes in route}$
      \EndFor
    \EndProcedure
  \end{algorithmic}

\end{algorithm}

\section{Computational results}\label{sec_results}
This algorithm was programmed in Python, using the integrated
development evironment PyCharm developed by JetBrains. It was run in a
computer with a processor of $2.2$ GHz and a RAM memory of $8$
Gb. The problem was tested with 21 problems, with different nodes and
types of vehicles. Only one of the 21 problems don't have a feasible
solution. This data set was given by Juan Carlos Rivera.

With the lower bound calculated for each problem, the lower bound
comparison (LBC) was calculated using the equation \ref{eq:er}. This measure allows to calculate how far is the objective function compared to the lower bound percent wise.

\begin{equation}
  \label{eq:er}
  \text{LBC} = \frac{z - lb}{lb} * 100\%
\end{equation}

The results for LBC and the time of computation (TC) in seconds can be
found in table Table \ref{t_cp}.

\begin{table}[]
\centering
\begin{tabular}{llll}
\hline
Problem & Nodes & LBC       & TC      \\ \hline
1       & 50    & 120.66\%  & 0.01613 \\
2       & 50    & 178.723\% & 0.01549 \\
3       & 50    & 152.225\% & 0.01512 \\
4       & 75    & 37.703\%  & 0.03466 \\
5       & 75    & 74.923\%  & 0.03448 \\
6       & 75    & 72.748\%  & 0.03706 \\
7       & 100   & 60.148\%  & 0.06178 \\
8       & 100   & 107.347\% & 0.05721 \\
9       & 100   & 108.602\% & 0.0629  \\
10      & 150   & 31.021\%  & 0.14909 \\
11      & 150   & 22.666\%  & 0.16225 \\
12      & 150   & 37.852\%  & 0.14949 \\
13      & 199   & 134.67\%  & 0.32369 \\
14      & 199   & 40.087\%  & 0.35164 \\
15      & 199   & 35.57\%   & 0.31384 \\
16      & 120   & 422.168\% & 0.09116 \\
17      & 120   & 372.659\% & 0.11881 \\
18      & 120   & 484.049\% & 0.08868 \\
19      & 100   & 102.357\% & 0.05478 \\
20      & 100   & 272.661\% & 0.05941 \\
21      & 100   & 265.137\% & 0.05318 \\ \hline
\end{tabular}
\caption{Lower bound comparison and time of computation (seconds)}
\label{t_cp}
\end{table}

In first place, it is important to notice that the algorithm did had
reasonable time of computations for different size of nodes. In the
same manner, it founded a feasible solution for the problem if it
exists. Furthermore, although for some problems the solution was far
from the lower bound calculated, for problems such as $4$, $11$ and
more the algorithm successfully found a solution considerably
close. In this manner, this algorithm should have some cases in which
it performs better.

In conclusion, it is seen that the algorithm effectively solves the
presented problem as it always returned a feasible solution that can
be sufficiently close to a lower bound. Moreover, it is seen that the
time of computations is reasonable enough compared to the size of the
problem.

\subsection{Algorithm comparison}\label{sec_comp}
The comparison is going to be used a colleague's work, to have a
better perspective of how good of a solution the algorithm presented
works. Specially, it is going to be compared to David Plazas E.'s
work. This work used to contrast the solutions is an adaptation of the
sweep algorithm \citep{renaud2002}. This comparison was done by
comparing the value of the two objective functions.

This comparison, similar to the one made for the lower bound, was made
by equation \ref{eq:pl} with $z_1$ being the value of the objective
function for this article and $z_2$ the value of the objective
function for colleague's work. This equation is used to compare how
distant the solution presented from the solution that David made.

\begin{equation}
  \label{eq:pl}
  \text{Comparison} = \frac{z_2 - z_1}{z_2}
\end{equation}

The results of the comparison for each problem can be found in table
Table \ref{t_pl}.

\begin{table}[]
\centering
\begin{tabular}{ll}
\hline
Problem & Comparison \\ \hline
1       & 17.616\%   \\
2       & 24.181\%   \\
3       & 36.222\%   \\
4       & 25.055\%   \\
5       & 29.911\%   \\
6       & 28.051\%   \\
7       & 45.344\%   \\
8       & 38.774\%   \\
9       & 37.36\%    \\
10      & 41.864\%   \\
11      & 46.271\%   \\
12      & 32.55\%    \\
13      & 34.026\%   \\
14      & 30.06\%    \\
15      & 38.768\%   \\
16      & 52.372\%   \\
17      & 51.792\%   \\
18      & 40.852\%   \\
19      & 47.827\%   \\
20      & 39.114\%   \\
21      & 42.589\%   \\ \hline
\end{tabular}
\caption{Comparison between objective functions between David Plazas work and the algorithm presented}
\label{t_pl}
\end{table}

In first place, it is seen that the algorithm presented in this
article outperformed in results the colleague's work, as a positive
percentage means a better solution as it shows that the obtained value
is smaller. Furthermore, it presents solutions at least $15\%$ smaller
than the ones presented in David's work. This shows that this
algorithm successfully outperforms algorithms with similar purposes.

On the other hand, time comparisons weren't made. As it is known,
Sweep algorithm has a smaller algorithmic complexity so perhaps, that
algorithm outperforms in time of computation the one
presented. Furthermore, even if time of computation is bigger than the
comparison, it does not mean that this algorithm is slow as it has
some really small times of computations.

\section{Conclusions}\label{sec_conclusions}
It was successfully designed a Clark and Wright algorithm modification
for the Heterogeneous Cumulative Capacitated Vehicle Routing Problem,
that solves in polynomial complexity. Furthermore, it was found that
the results by this algorithm are good compared to a colleague's work
and a defined lower bound. Finally, the time of computation for this
algorithm is good enough for the size of the problem as it allowed for
solving the problem in less than a second.

For further work, it is important to formally proof that the lower
bound used in this article is a lower bound for the problem. Moreover,
it is important to find improvements to the time of computation of
this algorithm as it would be desired to be smaller.
\section*{Acknowledgments}
The author thanks his colleague David Plazas E. for giving the results
of his algorithm for comparison.


{\small
\bibliographystyle{authordate1}

\bibliography{ref.bib}}


\end{document}
