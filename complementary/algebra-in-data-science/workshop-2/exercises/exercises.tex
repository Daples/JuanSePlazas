\documentclass[11pt]{article}
\usepackage[english]{babel}
\usepackage{geometry}
\usepackage{amsmath}
\usepackage{amsthm}
\usepackage{graphicx}
\usepackage[utf8]{inputenc}

%%%%%%%% MARGIN
\geometry{verbose, letterpaper, tmargin=2.5cm,
  bmargin=3cm,lmargin=2.5cm,rmargin=2.5cm}

%%%%%%%% PARAGRAPH SETTINGS
% https://tex.stackexchange.com/questions/27802/set-noindent-for-entire-file
\setlength\parindent{0pt}

% https://tex.stackexchange.com/questions/49188/how-to-insert-vertical-space-between-paragraphs
\setlength{\parskip}{5pt}

%%%%%%%% SUB-FIGURE PACKAGE
\usepackage{subcaption}

%%%%%%%% HYPERREF PACKAGE
\usepackage{hyperref}
\hypersetup{linkcolor=blue}
\hypersetup{citecolor=blue}
\hypersetup{urlcolor=blue}
\hypersetup{colorlinks=true}

%%%%%%%% DEFINITION AND THEOREM DEFINITIONS
\theoremstyle{definition}
\newtheorem{definition}{Definition}[section]

\theoremstyle{remark}
\newtheorem{remark}{Remark}

\theoremstyle{remark}
\newtheorem{example}{Example}

\theoremstyle{remark}
\newtheorem*{question}{Question}

\newtheorem{theorem}{Theorem}

\theoremstyle{proof}
\newtheorem{lemma}{Lemma}

%%%%%%%% MULTI-COLUMNS PACKAGE
\usepackage{multicol}

%%%%%%%% PERSONAL COMMANDS
\usepackage{amssymb}

%%%% Important sets
\renewcommand{\O}{\mathbb{O}}
\newcommand{\N}{\mathbb{N}}
\newcommand{\Z}{{\mathbb{Z}}}
\newcommand{\Q}{{\mathbb{Q}}}
\newcommand{\R}{{\mathbb{R}}}

%%%% Usual operations
\newcommand{\pow}[2]{#1^{#2}}
\newcommand{\expp}[1]{e^{#1}}
\newcommand{\fst}{\mathrm{fst}}
\newcommand{\snd}{\mathrm{snd}}

%%%% Lambda Calculus
\newcommand{\dneq}{\,\, \# \,\,}
\renewcommand{\S}{\pmb{\mathrm{S}}}
\newcommand{\I}{\pmb{\mathrm{I}}}
\newcommand{\K}{\pmb{\mathrm{K}}}
\newcommand{\ch}[1]{\ulcorner #1 \urcorner}

%%%% Ordinal Lambda Calculus
\newcommand{\ordAlph}{\Sigma_{\text{Ord}}}
\newcommand{\termOrd}{\text{Term}_\text{Ord}}
\newcommand{\fl}{\mathrm{fl}}
\newcommand{\sk}{\mathrm{sk}}

%% Superscript to the left
% https://latex.org/forum/viewtopic.php?t=455
\usepackage{tensor}
\newcommand{\app}[3]{\tensor*[^{#1}]{\left(#2, #3\right)}{}}

%%%% Make optional parameter
% https://tex.stackexchange.com/questions/217757/special-behavior-if-optional-argument-is-not-passed
\usepackage{xparse}

%%%% Statistics
\NewDocumentCommand{\E}{o m}{
  \IfNoValueTF{#1}
  {\mathbb{E}\left[#2\right]}
  {\mathbb{E}^{#1}\left[ #2\right]}
}
\NewDocumentCommand{\V}{o m}{
  \IfNoValueTF{#1}
  {\mathrm{Var}\left[#2\right]}
  {\mathrm{Var}^{#1}\left[ #2\right]}
}
\NewDocumentCommand{\cov}{o m}{
  \IfNoValueTF{#1}
  {\mathrm{cov}\left(#2\right)}
  {\mathrm{cov}^{#1}\left( #2\right)}
}
\RenewDocumentCommand{\P}{o o m}{
  \IfNoValueTF{#1}
  {\IfNoValueTF{#2}
    {\mathrm{P}\left(#3\right)}
    {\mathrm{P}^{#2}\left(#3\right)}}
  {\IfNoValueTF{#2}
    {\mathrm{P}_{#1}\left(#3\right)}
    {\mathrm{P}_{#1}^{#2} \left(#3\right)}}
}

%%%% Lambda Calculus
\NewDocumentCommand{\cx}{o}{
  \IfNoValueTF{#1}
  {\left[\quad\right]}
  {\left[\, #1 \,\right]}
}

%%%% Create absolute value function
% https://tex.stackexchange.com/questions/43008/absolute-value-symbols
\usepackage{mathtools}
\DeclarePairedDelimiter\abs{\lvert}{\rvert}%
\DeclarePairedDelimiter\norm{\lVert}{\rVert}%
\makeatletter
\let\oldabs\abs
\def\abs{\@ifstar{\oldabs}{\oldabs*}}
%
\let\oldnorm\norm
\def\norm{\@ifstar{\oldnorm}{\oldnorm*}}
\makeatother

%%%%%%%% LOGIC TREES
\usepackage{prftree}

%%%%%%%% SPLIT EQUATIONS
% https://tex.stackexchange.com/questions/51682/is-it-possible-to-pagebreak-aligned-equations
\allowdisplaybreaks

%%%%%%%% FLOAT SPECIFIER
% https://www.overleaf.com/learn/latex/Errors/LaTeX_Error:_Unknown_float_option_%60H%27
\usepackage{float}

%%%%%%%% TO USE SHORT COMMANDS FOR VECTOR LINES
\usepackage{esvect}

%%%%%%%% ENUMERATE LABEL
% https://www.latex-tutorial.com/tutorials/lists/
\usepackage{enumitem}

%%%%%%%% DIFFERENT FONTS FOR MATH
\usepackage{mathrsfs}

%%%%%%%% CODE RENDERING !!! UNCOMMENT IF NEEDED !!!
% Compile with flag -shell-escape
%\usepackage{minted}

%%%%%%%% START DOCUMENT
\title{Theoretic Exercises}
\author{Andrés Felipe Tamayo \\
  \scalebox{0.7}{aftamayoa@eafit.edu.co} \and
  David Plazas Escudero \\
  \scalebox{0.7}{dplazas@eafit.edu.co} \and
  Juan Sebasti\'an C\'ardenas-Rodríguez \\
  \scalebox{0.7}{jscardenar@eafit.edu.co} \\ \\
  \scalebox{0.7}{Mathematical Engineering, Universidad EAFIT}}

\date{\today}


\begin{document}
\maketitle

\section*{Exercise 1}

\begin{question}
  Prove computationally or theoretically that for all $x \in \mathbb{R}^{n}$ it
  is given that $x^{T}Cx \geq 0$.
\end{question}

\begin{proof}

  Let $Y = (Y_{1},\ldots,Y_{p})^{T}$ be a p-dimensional random vector with the
  mean vector
  \[
    \mu = \E{Y} = (\mu_{1},\ldots,\mu_{p})^{T},
  \]
  where $\mu_{i} = \E{Y_{i}}$ is the mean of the \textit{i}-th component. Its
  covariance matrix is defined as the $p \times p$ matrix
  \[
    \Sigma = \E{(Y - \mu)(Y - \mu)^{T}} = (\sigma_{ij})
  \]

  where $\sigma_{ij} = \E{(Y_{i} - \mu_{i})(Y_{j} - \mu_{j})}$ is, indeed, the
  covariance between two random variables $Y_{i}$ and $Y_{j}$ for $i \neq j$,
  and $\sigma_{ii}$ is simply the variance of the \textit{i}-th component of
  $Y$.

  It's going to be shown in Excercise 3 that, a matrix $\Sigma$ is a covariance
  matrix if and only if it is of the form $A^{T}*A$ for some matrix $A$. In the
  present exercise, we have that $C = A^{T}*A$ and by so, the covariance matrix
  is nonnegative in the following sense:

  Let's consider any vector $x = (x_{1},\ldots,x_{n})T$. Then
  \[
    x^{T}Cx = \sum_{i = 1}^{n}\sum_{j = 1}^{n} x_{i}x_{j}\sigma_{ij} = \E{x^{T}(Y - \mu)}^{2} = \V{x^{T}(Y - \mu)} \geq 0
  \]
\end{proof}

%%%%%%%%%%%%%%%%%%%%%%%%
\section*{Exercise 3} \label{Ex3}
\begin{question}
  Investigate if given any matrix $A$, then $C = A^{T}*A$ can be considered as a
  covariance matrix. Also, see if given a covariance matrix $C$, there exists
  $A$ such that $C = A^{T} * A$.
\end{question}

Let's consider the following lemma

\begin{lemma}
  For $Y$ a $p \times 1$ random vector with covariance matrix $\cov{Y}$, and
  suitable (nonrandom) matrices $A$, $B$, $C$ and $D$, we have
\begin{enumerate}
  \item $\E{AY + B} = A\E{Y} +B$,
  \item $\cov{AY + B} = A\cov{Y}A^{T}$,
  \item $\cov{AY + B, \,CY + D} = A\cov{Y}C^{T}$
\end{enumerate}
\end{lemma}

The result in part $2$ suggests a way to construct more elaborate covariance
matrices from the simpler ones. For example, starting from a $Y$ with the
identity covariance matrix, \textit{it follows that $AA^{T}$ in a genuine
  covariace matrix for any matrix $A$}. In fact, all covariance matrices turn
out to hace this form. In other words, a matrix $\Sigma$ is a covariance matrix,
if and only if it is of the form $AA^{T}$ for some matrix $A$.

Now, consider the fact that the covariance matrix is a symmetrix and nonnegative
definite matrix. With this fact, the following theorem guarantee that, given a
covariance matrix $C$, there exists $A$ such that $C = A^{T}*A$.

\begin{theorem}
  For a $p \times p$ symmetrix matrix $\Sigma = (\sigma_{ij})$, the following
  are equivalent:
\begin{enumerate}
  \item $\Sigma$ is nonnegative definite
  \item there exists a matrix $A$ such that
    \[
    \Sigma = A*A^{T}
    \]
\end{enumerate}
\end{theorem}

%%%%%%%%%%%%%%%%%%%%%%%%
\section*{Exercise 8}
\subsection*{Exercise a}

\begin{question}
  The rank of the data matrix is $p$.
\end{question}
\begin{proof}
  This statement is not completely accurate and we believe it should be changed
  to ``the rank of the data matrix is at most $p$''. First, the rank of a matrix
  $A\in\mathbb{R}^{n\times p}$ satisfies
  \[
    \mathrm{rank}(A)\leq \min\{n, p\}.
  \]
  Additionally, it is well known that most of the statistical analysis and
  methods are based on the premise of the number of observations (samples) being
  significantly larger than the number of variables in the dataset ($n\gg p$).
  One good example of the issue while working under $n<p$ is the covariance
  matrix: Let $S$ be the covariance matrix of the dataset; if $n<p$ then
  $\mathrm{rank}(S)<p$, thus the covariance matrix is singular and it is clear
  that a lot of statistical methods relay on the inverse of the covariance
  matrix. Therefore, is always desirable that $n\gg p$.

  Finally, it is possible to have an unnoticed multicollinearity situation
  between the variables of the dataset; in this scenario, one (or more)
  variables can be, approximately, written as a linear combination of the other
  variables which directly implies that the matrix is not full-rank and it will
  be ill-conditioned. Hence, with these arguments, the rank of the data matrix
  should be at most $p$.
\end{proof}


%%%%%%%%%%%%%%%%%%%%%%%%
\subsection*{Exercise b}
\begin{question}
  The operation $A^{T}1_{n}$ where $1_{n}$ is a vector $n \times 1$ with all its
  components equal to one provides a vector $p \times 1$ whose components are
  the sum of the values of each variable.
\end{question}

\begin{proof}
  This is correct, indeed, if $A$ is a matrix $(n \times p)$, then $A^{T}$ is a
  matrix $(p \times n)$, then $A^{T}1_{n}$ is a vector $p \times 1$. Now, all
  the components of this matrix are the sum of the values of each variable. This
  is,
  \[
    A^{T}1_{n} = \begin{bmatrix}
      a_{11} & a_{21} & a_{31} & \ldots & a_{n1}\\
      a_{12}& a_{22}  & a_{32} & \ldots & a_{n2} \\
      \vdots & \vdots & \vdots & \ddots & \vdots \\
      a_{1p} & a_{2p} & a_{3p} & \cdots & a_{np}
    \end{bmatrix} * \begin{bmatrix}
      1\\
      1 \\
      \vdots\\
      1
    \end{bmatrix} = \begin{bmatrix}
      a_{11} + a_{21} + a_{31} + \ldots + a_{n1}\\
      a_{12} + a_{22} + a_{32} + \ldots + a_{n2} \\
      \vdots \\
      a_{1p} + a_{2p} + a_{3p} + \ldots + a_{np}
    \end{bmatrix}
  \]
  Therefore
  \begin{equation*}
    A^{T}1_{n} =
    \begin{bmatrix}
      \displaystyle\sum_{i=1}^{n}a_{i1} \\
      \displaystyle\sum_{i=1}^{n}a_{i2} \\
      \vdots \\
      \displaystyle\sum_{i=1}^{n}a_{ip} \\
    \end{bmatrix}
  \end{equation*}
\end{proof}

%%%%%%%%%%%%%%%%%%%%%%%%
\subsection*{Exercise c}
\begin{question}
  Show that $\frac{1}{n}A^{T}1_{n}$ is a vector with the means of the $p$
  variables.
\end{question}

\begin{proof}
  As seen in the previous exercise:
  %
  \begin{equation*}
    A^{T}1_{n} =
    \begin{bmatrix}
      \displaystyle\sum_{i=1}^{n}a_{i1} \\
      \displaystyle\sum_{i=1}^{n}a_{i2} \\
      \vdots \\
      \displaystyle\sum_{i=1}^{n}a_{ip} \\
    \end{bmatrix}
  \end{equation*}
  %
  Therefore
  \begin{equation*}
    \frac{1}{n}A^{T}1_{n} =
    \begin{bmatrix}
      \displaystyle\frac{1}{n}\sum_{i=1}^{n}a_{i1} \\
      \displaystyle\frac{1}{n}\sum_{i=1}^{n}a_{i2} \\
      \vdots \\
      \displaystyle\frac{1}{n}\sum_{i=1}^{n}a_{ip} \\
    \end{bmatrix}
    =
    \begin{bmatrix}
      \mu_{1} \\
      \mu_{2} \\
      \vdots \\
      \mu_{p}
    \end{bmatrix}
  \end{equation*}
\end{proof}
%%%%%%%%%%%%%%%%%%%%%%%%
\subsection*{Exercise d}
\begin{question}
  Show that $\frac{1}{n}1_{n}1_{n}^{T}A$ is a matrix for which the columns
  contains the mean of each variable.
\end{question}

\begin{proof}
  It is seen that
  %
  \begin{align*}
    1_{n}1_{n}^{T}A
    &=\begin{bmatrix}
      1 \\
      \vdots \\
      1
    \end{bmatrix}
    \begin{bmatrix}
      1 & \ldots & 1
    \end{bmatrix}
                   \begin{bmatrix}
                     a_{11} & \ldots & a_{1p} \\
                     \vdots & \ddots & \vdots \\
                     a_{n1} & \ldots & a_{np}
                   \end{bmatrix} \\
    &=
      \begin{bmatrix}
        1 & \ldots & 1 \\
        \vdots & \ddots & \vdots \\
        1 & \ldots & 1
      \end{bmatrix}
                     \begin{bmatrix}
                       a_{11} & \ldots & a_{1p} \\
                       \vdots & \ddots & \vdots \\
                       a_{n1} & \ldots & a_{np}
                     \end{bmatrix} \\
    &=
      \begin{bmatrix}
        a_{11} + \ldots + a_{n1} & \ldots & a_{1p} + \ldots + a_{np} \\
        \vdots & \ddots & \vdots \\
        a_{11} + \ldots + a_{n1} & \ldots & a_{1p} + \ldots + a_{np}
      \end{bmatrix} \\
    &=
      \begin{bmatrix}
        \displaystyle\sum_{i=1}^{n}a_{i1} & \ldots & \displaystyle\sum_{i=1}^{n}a_{ip} \\
        \vdots & \ddots & \vdots \\
        \displaystyle\sum_{i=1}^{n}a_{i1} & \ldots & \displaystyle\sum_{i=1}^{n}a_{ip}
      \end{bmatrix}
  \end{align*}
  %
  Therefore, it is obtained that:
  \begin{equation*}
    \frac{1}{n}1_{n}1_{n}^{T}A =
      \begin{bmatrix}
        \displaystyle\frac{1}{n}\sum_{i=1}^{n}a_{i1} & \ldots & \displaystyle\frac{1}{n}\sum_{i=1}^{n}a_{ip} \\
        \vdots & \ddots & \vdots \\
        \displaystyle\frac{1}{n}\sum_{i=1}^{n}a_{i1} & \ldots & \displaystyle\frac{1}{n}\sum_{i=1}^{n}a_{ip}
      \end{bmatrix}
      =
      \begin{bmatrix}
        \mu_{1} & \ldots & \mu_{p} \\
        \vdots & \ddots & \vdots \\
        \mu_{1} & \ldots & \mu_{p}
      \end{bmatrix}
  \end{equation*}
\end{proof}

%%%%%%%%%%%%%%%%%%%%%%%%
\subsection*{Exercise e}
\begin{question}
  Explain the meaning of the matrix $\tilde{A}=A-\frac{1_n1_n^TA}{n}$ and the
  matrix $S=\frac{\tilde{A}^T\tilde{A}}{n}$.
\end{question}
\begin{proof}
  The matrix $\tilde{A}$ can be rewritten as
  $\tilde{A}=\left(I_n-\frac{1_n1_n^T}{n}\right)A$ and the matrix inside the
  parenthesis is known as the centering matrix, which subtracts the mean vector
  to the each of the observations in the dataset $A$. Hence, $\tilde{A}$ is the
  centered dataset.

  Furthermore, the matrix $S$ is an equivalent definition of the Maximum
  Likelihood Estimation (MLE) of the covariance matrix (since it is divided by
  $n$). The product of $\tilde{A}^T$ and $\tilde{A}$ corresponds to the pairwise
  sum of centered data points for two variables and it is divided by $n$, which
  is the exact MLE for the covariance.
\end{proof}

%%%%%%%%%%%%%%%%%%%%%%%%
\subsection*{Exercise g}
\begin{question}
  Explain how the eigenvalues of $\widetilde{A}$ are related to those of
  $\widetilde{A} + I$, being $I$ the identity of $p \times p$. Explain now the
  relation with $\widetilde{A} +kI$, $k >0$ and related it to the result of
  Ledoit and Wolf.
\end{question}

\begin{proof}
  If $\lambda$ is an eigenvalue of $\widetilde{A}$ and $x$ is the corresponding
  eigenvector, then $1 + \lambda$ is an eigenvalue of $I + \widetilde{A}$ and
  $1 - \lambda$ is an eigenvalue of $I - \widetilde{A}$. In either case, $x$ is
  the corresponding eigenvetor. Thus, for $I + \widetilde{A}$ we have
  \begin{align*}
    \widetilde{A}x &= \lambda x \\
    x + \widetilde{A}x &= x + \lambda x \\
    (I + \widetilde{A})x &= (1 + \lambda)x
  \end{align*}

  Now, if $\lambda_{1},\lambda_{2}, \ldots$ are the eigenvalues of the
  $n \times n$ matrix $\widetilde{A}$, $I$ is the $n \times n$ identity matrix
  and $k$ is a single numbers, them the eigenvalues of the matrix
  $\widetilde{A} + kI$ are $\lambda_{1 + k}, \lambda_{2 + k}, \ldots$. To see
  this, If $\lambda$ is any eigenvalie of $\widetilde{A}$, then
  $\widetilde{A}X = \lambda X$, hence
  \[
    (\widetilde{A} +kI)X = AX + KX = \lambda X + kX = (\lambda + k) X
  \]
\end{proof}

%%%%%%%%%%%%%%%%%%%%%%%%
\section*{Exercise 9}
\begin{question}
  See what the Kronecker product consists of and verify computationally that if
  $A$ and $B$ are squares of orders $n$ and $p$, the $np$ eigenvalues of its
  Kronecker product, $A \otimes B$ are the Kronecker product of the eigenvalues
  of $A$ and $B$. Show two applications of the Kronecker product.
\end{question}

\begin{proof}
  Given an $m \times n$ matrix $A$ and a $p \times q$ matrix $B$, their
  kronecker product $C = A \otimes B$ is an $(m \cdot n) \times (n \cdot q)$
  matrix with elements defined by
  \[
    c_{\alpha \beta} = a_{ij}b_{kl}
  \]
  where
  \begin{align*}
    \alpha &= p(i-j)+k \\
    \beta &= q(i-j) + l
  \end{align*}

  Kronecker product can be used in order to solve the Lyapunov equation and
  verify stability of discrete time systems. Kronecker product is also useful
  for representing 2D image processing operations in matrix-vector form.
\end{proof}

%%%%%%%%%%%%%%%%%%%%%%%%
\section*{Exercise 10}
\begin{question}
  Discuss how the spectral decomposition of $A$ is related to the spectral
  decomposition for $A^{-1}$ and $A^{T}$.
\end{question}

\begin{proof}
  The spectral decomposition for $A$ is given by the matrix that contains in
  it's columns the eigen vectors of $A$ noted by $Q$ and a diagonal matrix that
  contains in the diagonal the eigen values $\Lambda$. This decomposition
  satisfies that:
  %
  \begin{equation*}
    A = Q\Lambda Q^{-1}
  \end{equation*}
  %
  In this manner, it is seen that:
  %
  \begin{align*}
    A^{-1} &= \left(Q \Lambda Q^{-1}\right)^{-1} \\
           &= \left(\Lambda Q^{-1}\right)^{-1} Q^{-1} \\
           &= Q \Lambda^{-1} Q^{-1}
  \end{align*}
  %
  As $\Lambda$ is a diagonal matrix, then the inverse of the matrix is given by
  each of the diagonal values inverted. Therefore, it is seen that the
  decomposition of the inverse matrix is given by the same matrix $Q$ and the
  inverse of $\Lambda$ given by:
  \begin{equation*}
    \Lambda^{-1} =
    \begin{bmatrix}
      \frac{1}{\lambda_{1}} & 0 & \ldots & 0 \\
      0 & \frac{1}{\lambda_{2}} & \ldots & 0 \\
      \vdots & \vdots & \ddots & \vdots \\
      0 & 0 & \ldots & \frac{1}{\lambda_{n}}
    \end{bmatrix}
  \end{equation*}
  %
  On the other hand, for $A^{T}$ it is seen that
  \begin{align*}
    A^{T} &= \left(Q \Lambda Q^{-1}\right)^{T} \\
          &= \left(\Lambda Q^{-1}\right)^{T} Q^{T} \\
          &= \left(Q^{-1}\right)^{T} \Lambda^{T} Q^{T} \\
          &= \left(Q^{T}\right)^{-1} \Lambda Q^{T}
  \end{align*}
  %
  Therefore, the spectral decomposition for the transpose is given by the same
  eigen values matrix $\Lambda$ and their eigen vector matrix $\hat{Q}$ such
  that:
  \begin{equation*}
    \hat{Q}^{-1} = Q^{T}
  \end{equation*}
\end{proof}
\end{document}
