\documentclass[11pt]{article}
\usepackage[english]{babel}
\usepackage{geometry}
\usepackage{amsmath}
\usepackage{amsthm}
\usepackage{graphicx}
\usepackage{caption}
\usepackage[utf8]{inputenc}


%%%%%%%% SUB-FIGURE PACKAGE
\usepackage{subcaption}

%%%%%%%% MULTI-COLUMNS PACKAGE
\usepackage{multicol}

%%%%%%%% PERSONAL COMMANDS
\usepackage{amssymb}

%%%% Important sets
\renewcommand{\O}{\mathbb{O}}
\newcommand{\N}{\mathbb{N}}
\newcommand{\Z}{{\mathbb{Z}}}
\newcommand{\Q}{{\mathbb{Q}}}
\newcommand{\R}{{\mathbb{R}}}

%%%% Usual operations
\newcommand{\pow}[2]{#1^{#2}}
\newcommand{\expp}[1]{e^{#1}}
\newcommand{\fst}{\mathrm{fst}}
\newcommand{\snd}{\mathrm{snd}}

%%%% Lambda Calculus
\newcommand{\dneq}{\,\, \# \,\,}
\newcommand{\prm}[1]{\pmb{\mathrm{#1}}}
\renewcommand{\S}{\prm{S}}
\newcommand{\I}{\prm{I}}
\newcommand{\K}{\prm{K}}
\newcommand{\ch}[1]{\ulcorner #1 \urcorner}

%%%% Ordinal Lambda Calculus
\newcommand{\ordAlph}{\Sigma_{\text{Ord}}}
\newcommand{\termOrd}{\text{Term}_\text{Ord}}
\newcommand{\fl}{\mathrm{fl}}
\newcommand{\sk}{\mathrm{sk}}

%% Superscript to the left
% https://latex.org/forum/viewtopic.php?t=455
\usepackage{tensor}
\newcommand{\app}[3]{\tensor*[^{#1}]{\left(#2, #3\right)}{}}

%%%% Make optional parameter
% https://bit.ly/3jVGRwQ
\usepackage{xparse}

%%%% Statistics
\NewDocumentCommand{\E}{o m}{
  \IfNoValueTF{#1}
  {\mathbb{E}\left[#2\right]}
  {\mathbb{E}^{#1}\left[ #2\right]}
}
\NewDocumentCommand{\V}{o m}{
  \IfNoValueTF{#1}
  {\mathrm{Var}\left[#2\right]}
  {\mathrm{Var}^{#1}\left[ #2\right]}
}
\RenewDocumentCommand{\P}{o o m}{
  \IfNoValueTF{#1}
  {\IfNoValueTF{#2}
    {\mathrm{P}\left(#3\right)}
    {\mathrm{P}^{#2}\left(#3\right)}}
  {\IfNoValueTF{#2}
    {\mathrm{P}_{#1}\left(#3\right)}
    {\mathrm{P}_{#1}^{#2} \left(#3\right)}}
}

%%%% Lambda Calculus
\NewDocumentCommand{\cx}{o}{
  \IfNoValueTF{#1}
  {\left[\quad\right]}
  {\left[\, #1 \,\right]}
}

%%%% Create absolute value function
% https://bit.ly/33Rkq6H
\usepackage{mathtools}
\DeclarePairedDelimiter\abs{\lvert}{\rvert}%
\DeclarePairedDelimiter\norm{\lVert}{\rVert}%
\makeatletter
\let\oldabs\abs
\def\abs{\@ifstar{\oldabs}{\oldabs*}}
%
\let\oldnorm\norm
\def\norm{\@ifstar{\oldnorm}{\oldnorm*}}
\makeatother

%%%%%%%% LOGIC TREES
\usepackage{prftree}

%%%%%%%% SPLIT EQUATIONS
% https://bit.ly/33P1OUM
\allowdisplaybreaks

%%%%%%%% FLOAT SPECIFIER
% https://bit.ly/30Wi4BC
\usepackage{float}

%%%%%%%% TO USE SHORT COMMANDS FOR VECTOR LINES
\usepackage{esvect}

%%%%%%%% ENUMERATE LABEL
% https://www.latex-tutorial.com/tutorials/lists/
\usepackage{enumitem}

%%%%%%%% DIFFERENT FONTS FOR MATH
\usepackage{mathrsfs}


%%%%%%%% MARGIN
\geometry{verbose, letterpaper, tmargin=3cm,
  bmargin=3cm,lmargin=2.5cm,rmargin=2.5cm}

%%%%%%%% PARAGRAPH SETTINGS
% https://bit.ly/36WrtN4
\setlength\parindent{0pt}

% https://bit.ly/371dvto
\setlength{\parskip}{5pt}

%%%%%%%% HYPERREF PACKAGE
\usepackage{hyperref}
\hypersetup{linkcolor=blue}
\hypersetup{citecolor=blue}
\hypersetup{urlcolor=blue}
\hypersetup{colorlinks=true}


%%%%%%%% DEFINITION AND THEOREM DEFINITIONS
\theoremstyle{definition}
\newtheorem{definition}{Definition}[section]

\theoremstyle{remark}
\newtheorem{remark}{Remark}

\theoremstyle{remark}
\newtheorem{question}{Question}

\newtheorem{theorem}{Theorem}[section]

%%%%%%%% ENUMERATE LABEL
% https://www.latex-tutorial.com/tutorials/lists/
\usepackage{enumitem}

%%%%%%%% CODE RENDERING !!! UNCOMMENT IF NEEDED !!!
% Compile with flag -shell-escape
%\usepackage{minted}

%%%%%%%% START DOCUMENT

\title{SAUL Grammar}
\author{Juan Sebasti\'an C\'ardenas-Rodríguez \\
  \scalebox{0.7}{jscardenar@eafit.edu.co} \and
  Manuela Gallego Gómez \\
  \scalebox{0.7}{mgalle41@eafit.edu.co} \and
  \scalebox{0.7}{Mathematical Engineering, Universidad EAFIT}}

\date{\today}


\begin{document}
\maketitle

\section{Grammar}

The grammar accepted by SAUL is:
%
\begin{align*}
  \texttt{<program>}
  \equiv& \texttt{ codigo <funDefinitionList> fincodigo} \\
  \texttt{<funDefinitionList>}
  \equiv& \texttt{ <funDefinition> <funDefinitionList>} \mid \epsilon \\
  \texttt{<funDefinition>}
  \equiv& \texttt{ funcion <variable> ( <varDefList> )} \\
        & \texttt{ <varDefList> <statementList> fincodigo} \\
  \texttt{<varDefList>}
  \equiv& \texttt{ <variable> <varDefList>} \mid \epsilon \\
  \texttt{<variable>}
  \equiv& \texttt{ entero <variableName> } \mid \texttt{ real <variableName>} \\
  \texttt{<variableName>}
  \equiv& \texttt{ <letter> <characters>} \\
  \texttt{<letter>}
  \equiv&\,\,\, A \mid \ldots \mid Z \mid a \mid \ldots \mid z \\
  \texttt{<characters>}
  \equiv& \texttt{ <letter> <characters> } \mid
          \texttt{ <number> <characters> } \mid \epsilon \\
  \texttt{<number>}
  \equiv& \,\,\, 0 \mid \ldots \mid 9 \\
  \texttt{<statementList>}
  \equiv& \texttt{ <statement> <statementList>} \mid \epsilon \\
  \texttt{<statement>}
  \equiv& \texttt{ leer <variableName> } \\
  \mid& \texttt{ mostrar <variableName> } \\
  \mid& \texttt{ llamar <variableName> ( <variableList> ) } \\
  \mid& \texttt{ <variableName> = <mathExpression> } \\
  \texttt{<variableList>}
  \equiv& \texttt{ <variableName> <variableList> } \\
  \mid& \texttt{ <constant> <variableList>} \mid \epsilon \\
  \texttt{ <mathExpression> }
  \equiv& \texttt{ <variable>} \mid \texttt{ <constant> } \\
  \mid& \texttt{ ( \texttt{<mathExpression>} ) } \\
  \mid& \texttt{ <variable> <op> <mathExpression> } \\
  \mid& \texttt{ <constant> <op> <mathExpression> } \\
  \mid& \texttt{ ( <mathExpression> ) <op> <mathExpression> } \\
  \texttt{<constant>}
  \equiv& \texttt{ <numbers> } \mid \texttt{ <numbers>.<numbers> } \\
  \texttt{<numbers>}
  \equiv& \texttt{ <number><numbers> } \mid \texttt{ <number> } \\
  \texttt{ <op> }
  \equiv& \,\,\, + \mid - \mid * \mid /
\end{align*}

\section{Limitations}

The program must have a main function called ``principal'' that receives no
arguments.

The math operations must be written with a space between each term.

The grammar accepts only the four basic mathematical operations: addition,
subtraction, multiplication and division. Operations with parentheses are
performed first. Then, the standard order of operations is followed:
multiplication, division, addition and subtraction.

The math expressions only accept positive constants. In order to work with
negative values, the "read" statement or a math operation must be executed.

If a double value is assigned to an integer variable, an error is displayed.

There are no global variables. Local variables have a limited scope and cannot
be accessed outside the block in which they were declared.

\end{document}
