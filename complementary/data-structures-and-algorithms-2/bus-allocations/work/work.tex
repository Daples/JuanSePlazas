\documentclass[11pt]{article}
\usepackage[english]{babel}
\usepackage{geometry}
\usepackage{amsmath}
\usepackage{amsthm}
\usepackage{graphicx}
\usepackage{caption}
\usepackage[utf8]{inputenc}


%%%%%%%% SUB-FIGURE PACKAGE
\usepackage{subcaption}

%%%%%%%% MULTI-COLUMNS PACKAGE
\usepackage{multicol}

%%%%%%%% PERSONAL COMMANDS
\usepackage{amssymb}

%%%% Important sets
\renewcommand{\O}{\mathbb{O}}
\newcommand{\N}{\mathbb{N}}
\newcommand{\Z}{{\mathbb{Z}}}
\newcommand{\Q}{{\mathbb{Q}}}
\newcommand{\R}{{\mathbb{R}}}

%%%% Usual operations
\newcommand{\pow}[2]{#1^{#2}}
\newcommand{\expp}[1]{e^{#1}}
\newcommand{\fst}{\mathrm{fst}}
\newcommand{\snd}{\mathrm{snd}}

%%%% Lambda Calculus
\newcommand{\dneq}{\,\, \# \,\,}
\newcommand{\prm}[1]{\pmb{\mathrm{#1}}}
\renewcommand{\S}{\prm{S}}
\newcommand{\I}{\prm{I}}
\newcommand{\K}{\prm{K}}
\newcommand{\ch}[1]{\ulcorner #1 \urcorner}

%%%% Ordinal Lambda Calculus
\newcommand{\ordAlph}{\Sigma_{\text{Ord}}}
\newcommand{\termOrd}{\text{Term}_\text{Ord}}
\newcommand{\fl}{\mathrm{fl}}
\newcommand{\sk}{\mathrm{sk}}

%% Superscript to the left
% https://latex.org/forum/viewtopic.php?t=455
\usepackage{tensor}
\newcommand{\app}[3]{\tensor*[^{#1}]{\left(#2, #3\right)}{}}

%%%% Make optional parameter
% https://bit.ly/3jVGRwQ
\usepackage{xparse}

%%%% Statistics
\NewDocumentCommand{\E}{o m}{
  \IfNoValueTF{#1}
  {\mathbb{E}\left[#2\right]}
  {\mathbb{E}^{#1}\left[ #2\right]}
}
\NewDocumentCommand{\V}{o m}{
  \IfNoValueTF{#1}
  {\mathrm{Var}\left[#2\right]}
  {\mathrm{Var}^{#1}\left[ #2\right]}
}
\RenewDocumentCommand{\P}{o o m}{
  \IfNoValueTF{#1}
  {\IfNoValueTF{#2}
    {\mathrm{P}\left(#3\right)}
    {\mathrm{P}^{#2}\left(#3\right)}}
  {\IfNoValueTF{#2}
    {\mathrm{P}_{#1}\left(#3\right)}
    {\mathrm{P}_{#1}^{#2} \left(#3\right)}}
}

%%%% Lambda Calculus
\NewDocumentCommand{\cx}{o}{
  \IfNoValueTF{#1}
  {\left[\quad\right]}
  {\left[\, #1 \,\right]}
}

%%%% Create absolute value function
% https://bit.ly/33Rkq6H
\usepackage{mathtools}
\DeclarePairedDelimiter\abs{\lvert}{\rvert}%
\DeclarePairedDelimiter\norm{\lVert}{\rVert}%
\makeatletter
\let\oldabs\abs
\def\abs{\@ifstar{\oldabs}{\oldabs*}}
%
\let\oldnorm\norm
\def\norm{\@ifstar{\oldnorm}{\oldnorm*}}
\makeatother

%%%%%%%% LOGIC TREES
\usepackage{prftree}

%%%%%%%% SPLIT EQUATIONS
% https://bit.ly/33P1OUM
\allowdisplaybreaks

%%%%%%%% FLOAT SPECIFIER
% https://bit.ly/30Wi4BC
\usepackage{float}

%%%%%%%% TO USE SHORT COMMANDS FOR VECTOR LINES
\usepackage{esvect}

%%%%%%%% ENUMERATE LABEL
% https://www.latex-tutorial.com/tutorials/lists/
\usepackage{enumitem}

%%%%%%%% DIFFERENT FONTS FOR MATH
\usepackage{mathrsfs}


%%%%%%%% MARGIN
\geometry{verbose, letterpaper, tmargin=3cm,
  bmargin=3cm,lmargin=2.5cm,rmargin=2.5cm}

%%%%%%%% PARAGRAPH SETTINGS
% https://bit.ly/36WrtN4
\setlength\parindent{0pt}

% https://bit.ly/371dvto
\setlength{\parskip}{5pt}

%%%%%%%% HYPERREF PACKAGE
\usepackage{hyperref}
\hypersetup{linkcolor=blue}
\hypersetup{citecolor=blue}
\hypersetup{urlcolor=blue}
\hypersetup{colorlinks=true}


%%%%%%%% DEFINITION AND THEOREM DEFINITIONS
\theoremstyle{definition}
\newtheorem{definition}{Definition}[section]

\theoremstyle{remark}
\newtheorem{remark}{Remark}

\theoremstyle{remark}
\newtheorem{question}{Question}

\newtheorem{theorem}{Theorem}[section]

%%%%%%%% ENUMERATE LABEL
% https://www.latex-tutorial.com/tutorials/lists/
\usepackage{enumitem}

%%%%%%%% CODE RENDERING !!! UNCOMMENT IF NEEDED !!!
% Compile with flag -shell-escape
\usepackage{minted}
\usemintedstyle{vs}

%%%%%%%% START DOCUMENT

\title{UVa: 11389 - The Bus Driver Problem}
\author{Juan Sebasti\'an C\'ardenas-Rodríguez \\
  \scalebox{0.7}{Mathematical Engineering, Universidad EAFIT}}
\date{\today}


\begin{document}
\maketitle

It is important to remark that this algorithm was written in \texttt{C++},
compiled using \texttt{g++ 10.2.0} in Arch Linux x86\_64. Furthermore, the time
of execution given by the online judge is 0.000 and it passes all of the tests
that uDebug gives.

To test the code write the input in a file and pipe it to the executable.

\section{Problem Solving}

The minimum distance for which the travel is payed is given by $d$, the rate of
pay is given by $r$ and the number of buses is $n$. The distance of each travel
in the morning is given by $d_{i}$ and the distance of each travel in the
afternoon is given by $b_{i}$ for $i = 0,\ldots, n-1$.

\subsection{Data Structures}

Both $d_{i}$ and $b_{i}$ are represented by the data type \texttt{vector}. This
data type represents an array that can be modified in size. Furthermore, each
instance of the problem is represented by an custom class \texttt{BusProblem}.

\subsection{Algorithm}
The algorithm to solving this exercise is straightforward.
\begin{enumerate}
  \item The problem is read by standard input. Each of the $d_{i}$ and $b_{i}$
        are stored in the vector sorted. Both vectors where sorted in the same
        way to simplify coding.
  \item Then the following is calculated:
        \begin{equation*}
          c_i = d_i + b_{n - i + 1}, \text{ for } i=0,\ldots, n-1
        \end{equation*}
  \item Lastly, the minimum cost to realize the travels $c$ is given by the
        following formula:
        \begin{equation*}
          c = \sum_{i=0}^{n-1} r\cdot c_i H(c_i - d)
        \end{equation*}
        with $H(\cdot)$ the Heaviside function.
\end{enumerate}

The implementation of this algorithm is seen in the following code.
\inputminted[frame=lines, linenos, firstline=51,
lastline=61]{cpp}{../src/bus-allocations.cpp}

\subsection{Why Does it Work?}

The solution is found by grouping the smaller morning travels with the largest
afternoon travels. In this manner, let's show that any other grouping would
preserve or worsen the objective function. For this proof it will be shown that
any pairwise exchange does not improve the objective function; this is enough to
show that any other exchange is not good as it is a multiple of pairwise
exchanges.

Let $d_{i}$ and $b_{i}$ be the morning and afternoon travels such that for all
$j > i$ it occurs that $d_{j} \ge d_{i}$ and $b_{j} \le b_{i}$. Let's show that
if an exchange of afternoon travels between $i^{*}$ and $j^{*}$ does not
improve the objective function with $i^{*} < j^{*}$.

In first place, the extra cost for driver $i^{*}$ by the original
order $\lambda_{1}$ and the exchanged order $\lambda_{2}$ are given by:
%
\begin{align*}
  \lambda_1 &= r(d_{i^*} + b_{i^*})H(d_{i^*} + b_{i^*} - d) \\
  \lambda_2 &= r(d_{i^*} + b_{j^*})H(d_{i^*} + b_{j^*} - d)
\end{align*}
%
Similarly, for $j^{*}$ the costs are given by:
%
\begin{align*}
  \sigma_1 &= r(d_{j^*} + b_{j^*})H(d_{j^*} + b_{j^*} - d) \\
  \sigma_2 &= r(d_{j^*} + b_{i^*})H(d_{j^*} + b_{i^*} - d)
\end{align*}
%
Let's suppose that $d_{i^{*}} + b_{j^{*}} - d > 0$. In this manner:
%
\begin{align*}
  d_{i^{*}} + b_{j^{*}} - d > 0 &\Rightarrow d_{j^*} + b_{j^*} - d > 0
                                  \text{, as } d_{i^*} \le d_{j^*} \\
  d_{i^{*}} + b_{j^{*}} - d > 0 &\Rightarrow d_{i^*} + b_{i^*} - d > 0
                                  \text{, as } b_{j^*} \le b_{i^*} \\
  d_{i^*} + b_{i^*} - d > 0 &\Rightarrow d_{j^*} + b_{i^*} - d > 0
                              \text{, as } d_{i^*} \le d_{j^*}
\end{align*}
%
Therefore its obtained that:
%
\begin{align*}
  \lambda_1 &= r(d_{i^*} + b_{i^*}) \\
  \lambda_2 &= r(d_{i^*} + b_{j^*}) \\
  \sigma_1 &= r(d_{j^*} + b_{j^*}) \\
  \sigma_2 &= r(d_{j^*} + b_{i^*})
\end{align*}
%
In this manner it is clear that
$\lambda_{1} + \sigma_{1} = \sigma_{2} + \lambda_{2}$. Therefore, the objective
function stays the same. Now let's suppose the other case for which
$d_{i^{*}} + b_{j^{*}} - d < 0$. In this case it is seen that:
%
\begin{equation*}
  \lambda_2 = 0
\end{equation*}
%
Therefore for we obtain both extra costs as:
%
\begin{align*}
  c_1 &= r(d_{i^*} + b_{i^*})H(d_{i^*} + b_{i^*} - d) +
         r(d_{j^*} + b_{j^*})H(d_{j^*} + b_{j^*} - d) \\
  c_2 &= r(d_{j^*} + b_{i^*})H(d_{j^*} + b_{i^*} - d)
\end{align*}
%
If $d_{j^{*}} + b_{i^{*}} - d < 0$ therefore both $c_{1}$ and $c_{2}$ will be
zero using a similar reasoning as before. In this manner, in this case the
objective function is not affected.

Lastly, suppose that $d_{j^{*}} + b_{i^{*}} - d > 0$. In this case, note that it
is impossible that $d_{i^{*}} + b_{i^{*}} - d > 0$ and
$d_{j^{*}} + b_{j^{*}} - d > 0$ because $d_{i^{*}} + b_{j^{*}} < 0$. Therefore,
only one condition, at most, can be satisfied. Therefore it trivially shown that
the cost function gets worst in this case.


\end{document}
