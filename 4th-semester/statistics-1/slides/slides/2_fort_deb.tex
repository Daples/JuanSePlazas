\section{FORTALEZAS Y DEBILIDADES}
\begin{frame}{FORTALEZAS Y DEBILIDADES\footnote{\bibentry{stat}}}
\begin{multicols}{2}
\textbf{Fortalezas}
\begin{itemize}
    \item No importan los tamaños de las muestras o si no son iguales.
    \item Permite encontrar dos estimadores insesgados para $\sigma^2$.
    \item Permite tener en cuenta cuantas muestras se desee.
    \item Permite hallar (o no) evidencia suficiente para verificar si $\mu_1=\mu_2=...=\mu_k$.
\end{itemize}
\columnbreak
\textbf{Debilidades}
\begin{itemize}
    \item Se debe suponer normalidad en las poblaciones ($n<30$).
    \item Se debe suponer que las varianzas poblacionales para todas las muestras deben ser iguales.
    \item Se debe comprobar primero que las muestras sean independientes (podrían utilizarse métodos como el test de Levene\footnote{\bibentry{levene}}).
\end{itemize}
\end{multicols}
\end{frame}