\section{Conclusions}
It can be concluded that the graphs coincide in this paper and the Chen's paper using the Adams-Bashforth-Moulton predictor-corrector. Each discrepancy are due to the time-step and the interval of simulation; we suggest that in further work this is more clear so that the results are easier to replicate. In second place, we found that Adams' method can be, to a certain extent, equivalent to the Runge-Kutta method; taking into account, that the orders of both algorithms are different the similitude is a very good approximation. It was also shown that the generalized equations conserve the properties of memory and randomness that are fully necessary to simulate the macro-financial system of China. 

On the other hand, it was seen that the system has three equilibrium points: one of them ($p_1$) will always be unstable, and the other ones ($p_2,p_3$) depend on the system's order ($\alpha$). Additionally, it was found that the system has a critical value for $\alpha = 0.8436$, which is when the stability changes.

A successful control for both equilibrium points and periodic orbits was achieved; showing that the system, regardless of its chaotic behavior, can be stabilized.

Finally, a sensitivity analysis was developed successfully, based on the methods described on literature; based on this results, it could be observed that the sensitivity is highly dependent on the order of the system, yielding different results for each order.

In future work, it could be analyzed the sensitivity for the incommensurate system; furthermore, the system could be analyzed for $\alpha>1$, verifying if the Adams-Bashforth-Moulton predictor-corrector converges for these orders and developing a methodology for both control and sensitivity.








